%
%\usepackage{amsfonts}
%\usepackage{amssymb}
%\usepackage{amsmath}
%%\usepackage{txfonts}
%\usepackage{mathrsfs}
%\usepackage{mathdots}
%\usepackage{lmodern}
%%\usepackage[classicReIm]{kpfonts}
%\usepackage{physics}    %  partial and total derivatives  
%
%
%%___________ MATH THEOREMS
%
%%\newtheorem{definition}{a}

\newtheorem{proposition}{Proposition}
\newtheorem{example}{Example}
\newtheorem{definition}{Definition}
\theoremstyle{remark}

%\newcommand{\definicion}[1]
%{
%	\begin{definition}
%		\normalfont #1
%	\end{definition}
%}

\newcommand{\proposicion}[1]
{
	\begin{proposition}
		\normalfont #1
	\end{proposition}
}

\DeclareMathOperator{\sgn}{sgn}
\DeclareMathOperator{\supp}{supp}
\renewcommand{\vec}{\mbox{vec}}
\newcommand{\Mod}[1]{\ \mathrm{mod}\ #1}

\DeclarePairedDelimiter\floor{\lfloor}{\rfloor}

\renewcommand{\i}{\mathrm{i}}
\newcommand{\ejemplo}[1]
{
	\begin{example}
		\normalfont #1
	\end{example}
}

\newcommand{\definicion}[1]
{
	\begin{definition}
		\normalfont #1
	\end{definition}
}



\newcommand{\Lorentz}
{
	\mathcal{L}
}

\newcommand{\Vp}
{
	\mathcal{V}_{\parallel}
}

\newcommand{\Ad}
{
	\mathcal{A}_{d}
}



\newcommand{\Laguerre}[2]
{
	{L}^{(#1)}_{#2}
}


\newcommand{\Eqn}[2]
{
	\begin{align}
	#1
	\label{#2}
	\end{align}
}

\newcommand{\Set}[2]
{
	\left\lbrace #1  \,\middle|\, #2 \right\rbrace
}

\newcommand{\scalar}[2]{
	\langle#1,#2\rangle
}

\newcommand{\R}[1]
{ 
	\mathbb{R}^{#1}
}

\newcommand{\Q}[1]
{ 
	\mathbb{Q}^{#1}
}

\newcommand{\M}[2]
{
	\mathcal{M}_{#1 \times #2}
}

\newcommand{\application}[3]
{
	#1 : #2 \longrightarrow #3
}

\newcommand{\applicationr}[5]
{
	\begin{tikzcd}[row sep = 0ex, ampersand replacement=\&]
		#1: #2  \arrow[r]         \& #3 \\
		#4  \arrow[r, mapsto] \& #5  
	\end{tikzcd}
}
%&{#2}\longrightarrow {#3}\\
%&{#4}\longmapsto {#5}
%\underset{#4}{#2}  \underset{\longmapsto }{ \longrightarrow }  \underset{#5}{#3}




\newcommand{\vp}{v_\parallel}

%\newcommand{\C}[1]
%{ 
%	\mathbb{C}^{#1}
%}

\newcommand{\N}[1]
{ 
	\mathbb{N}^{#1}
}

\newcommand{\Z}[1]
{ 
	\mathbb{Z}^{#1}
}

\newcommand{\K}[1]
{ 
	\mathbb{K}^{#1}
}

\newcommand{\B}
{ 
	\vect{B}
}
\renewcommand{\b}
{ 
	\vect{b}
}

\renewcommand{\P}
{ 
	\mathcal{P}
}

\newcommand{\T}
{ 
	\mathcal{T}
}

\renewcommand{\L}
{ 
	\mathcal{L}
}


\DeclarePairedDelimiterX{\mean}[1]{\langle}{\rangle}{#1}

\newcommand{\Eval}[1]
{
	#1|
}
\newcommand{\Frac}[2]
{
	\left.{#1} \right/ {#2}
}
\newcommand{\dotprod}[2]
{
	\mean*{#1,#2}
}


\newcommand{\Matrix}[2]
{
	\mleft[
	\begin{array}{#1}
		#2
	\end{array}
	\mright]
}

\newcommand{\Determinant}[2]
{
	\left|
	\begin{array}{#1}
		#2
	\end{array}
	\right|
}

\newcommand\ddfrac[2]{\frac{\displaystyle #1}{\displaystyle #2}}




% NUMBERING OF EQUATIONS DEPENDING ON SECTION: e.g: (2.3)
% (For articles)

\numberwithin{equation}{section}

%\newcommand{\mean}[1]
%{
%	\langle {#1} \rangle
%}










\newcommand{\qmarks}[1]
{
	`` #1 ''
}
 

\newcommand{\btable}
{
  \begin{table}[H]
      \begin{center}
          \begin{spacing}{1.2}
              \begin{tabular}{| p{2.5cm} |  p{2.8cm}  | p{1.1cm} | p{4.5cm} |}
                  \hline
 \bf Argument & \bf Type & \bf Intent & \bf Description \\ \hline \hline   
}

\newcommand{\etable}[1]
{        \end{tabular}
      \end{spacing}
   \end{center}
   \vspace*{-0.8cm}
   \caption{#1}
\end{table}     
}


