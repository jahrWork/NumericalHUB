



\pgfplotstableset{
    every head row/.style={before row=\toprule,  after row=\midrule},
    every last row/.style={after row=\bottomrule},
    col sep=space, 
     columns/T_s/.style={
        column name=$\theta_s$(stagnation),
    },
    columns/qi_off/.style={
    column name=$q_i^{OFF}$(W/m$^2$),
    },
    columns/qi_on/.style={
    column name=$q_i^{ON}$(W/m$^2$),
    },
    columns/T_w/.style={
    column name=$\theta_w$,
    },
    columns/TI_b/.style={
    column name=$T I_b$(W/m$^2$),
    },
    columns/P/.style={
    column name=$P$(W/m$^2$),
    },
    columns/T_e/.style={
    column name=$\theta_e$,
    },
    columns/T_se/.style={
    column name=$\theta_{sc}$,
    },
    columns/T_si/.style={
    column name=$\theta_{si}$,
    },
    columns/T_i/.style={
    column name=$\theta_i$,
    },
    columns/h_i/.style={
    column name=$h_i(W/m^2 K)$,
    },
    columns/h_e/.style={
    column name=$h_e(W/m^2 K)$,
    },
    columns/h_g/.style={
    column name=$h_g(W/m^2 K)$,
    },
    columns/h_w/.style={
    column name=$h_w(W/m^2 K)$,
    },
    columns/c/.style={
    column name=$c(J/Kg K)$,
    },
    columns/T_inlet/.style={
    column name=$\theta_{INLET}$,
    },
    columns/I_beam/.style={
    column name=$I_{b}(W/m^2)$,
    },
    columns/I_diffuse/.style={
    column name=$I_{d}(W/m^2)$,
    },
    columns/Angle_of_incidence/.style={
    column name=$\gamma$,
    },
    columns/flow_rate/.style={
    column name=$\dot m (l/(min \ m^2))$,
    },
    columns/T_a/.style={
    column name=$T_a$,
    },
    columns/alpha_NIR/.style={
    column name=$\alpha_{NIR}$,
    },
    columns/alpha_FIR/.style={
    column name=$\alpha_{FIR}$,
    },
    columns/U_wall/.style={
    column name=$U_{wall}(W/m^2 K)$,
    },
    header=true, 
    string type}




\graphicspath{ {images/} }
\topmargin 0.0cm
\oddsidemargin 0.2cm
\textwidth 16cm 
\textheight 21cm
\footskip 1.0cm
\definecolor{mygreen}{rgb}{0,0.6,0}
\definecolor{mygray}{rgb}{0.5,0.5,0.5}
\definecolor{mymauve}{rgb}{0.58,0,0.82}

\lstset{ %
  backgroundcolor=\color{white},   % choose the background color; you must add \usepackage{color} or \usepackage{xcolor}
  basicstyle=\footnotesize,        % the size of the fonts that are used for the code
  breakatwhitespace=false,         % sets if automatic breaks should only happen at whitespace
  breaklines=true,                 % sets automatic line breaking
  captionpos=b,                    % sets the caption-position to bottom
  commentstyle=\color{mygreen},    % comment style
  deletekeywords={...},            % if you want to delete keywords from the given language
  escapeinside={\%*}{*)},          % if you want to add LaTeX within your code
  extendedchars=true,              % lets you use non-ASCII characters; for 8-bits encodings only, does not work with UTF-8
  frame=single,                    % adds a frame around the code
  keepspaces=true,                 % keeps spaces in text, useful for keeping indentation of code (possibly needs columns=flexible)
  keywordstyle=\color{blue}\textbf,       % keyword style
  language=Fortran,                 % the language of the code
  otherkeywords={*,source},            % if you want to add more keywords to the
  % set
  numbers=none,                    % where to put the line-numbers; possible
  % values are (none, left, right)
  numbersep=5pt,                   % how far the line-numbers are from the code
  numberstyle=\tiny\color{mygray}, % the style that is used for the line-numbers
  rulecolor=\color{black},         % if not set, the frame-color may be changed on line-breaks within not-black text (e.g. comments (green here))
  showspaces=false,                % show spaces everywhere adding particular underscores; it overrides 'showstringspaces'
  showstringspaces=false,          % underline spaces within strings only
  showtabs=false,                  % show tabs within strings adding particular underscores
  stepnumber=2,                    % the step between two line-numbers. If it's 1, each line will be numbered
  stringstyle=\color{mymauve},     % string literal style
  tabsize=2,                       % sets default tabsize to 2 spaces
  title=\lstname                   % show the filename of files included with \lstinputlisting; also try caption instead of title
}

\lstset{language=[90]Fortran,
    basicstyle=\tiny, %\footnotesize,
    keywordstyle=\color{blue},
    commentstyle=\color{red},
    stringstyle=\color{green},
    morecomment=[l]{!\ }% Comment only with space after !
}

\par\vspace{\baselineskip}
\lstset{language=Fortran}


\lstset{
    numbers=left,            
    numbersep=5pt,                 
    numberstyle=\tiny\color{mygray}
}

%Allow us to make the rules larger
\newlength\FHoffset
\setlength\FHoffset{1cm}


\fancyheadoffset{\FHoffset}
\renewcommand{\headrulewidth}{0.5pt} % No header rule
\renewcommand{\footrulewidth}{0.5pt} % Thin footer rule
% Headers - all currently empty
\lhead{}
\chead{}
\rhead{}
% Footers
\lfoot{}
\cfoot{}
\rfoot{\footnotesize Page \thepage\ of \pageref{LastPage}} % "Page 1 of 2"


\definecolor{DarkGoldenrod}{rgb}{0.8,0.6,0.1}
\definecolor{DarkRed}{rgb}{0.5,0.1,0.1}
\definecolor{DarkRedPart}{rgb}{0.3,0.1,0.2}

\usepackage{lettrine} % Package to accentuate the first letter of the text
\newcommand{\initialgold}[1]{ % Defines the command and style for the first
% letter
\lettrine[lines=3,lhang=0.3,nindent=0em]{
\color{DarkGoldenrod}
{\textsf{#1}}}{}}
\newcommand{\initialred}[1]{ % Defines the command and style for the first
% letter
\lettrine[lines=3,lhang=0.3,nindent=0em]{
\color{DarkRed}
{\textsf{#1}}}{}}
%----------------------------------------------------------------------------------------
%	TITLE SECTION
%----------------------------------------------------------------------------------------
\usepackage{titling} % Allows custom title configuration

\newcommand{\ON}{^{\textsc{\tiny ON}}}
\newcommand{\OFF}{^{\textsc{\tiny OFF}}}
\newcommand{\INLET}{_{\textsc{\tiny INLET}}}
\newcommand{\OUTLET}{_{\textsc{\tiny OUTLET}}}
