%%%%%%%%%%%%%%%%%%%%%%%%%%%%%%%%%%%%%%%%%%%%%%%%%%%%%%%%%%%%%%%%%%%%%%%%%%%%%%%%%%%%%%%%%%%%%%%%%%%%%%%%%%%%%%%%%%%%%%%%%%%%%%%%%%%%%%%%%%%%%%%%%%%%%%

% OVERVIEW

%%%%%%%%%%%%%%%%%%%%%%%%%%%%%%%%%%%%%%%%%%%%%%%%%%%%%%%%%%%%%%%%%%%%%%%%%%%%%%%%%%%%%%%%%%%%%%%%%%%%%%%%%%%%%%%%%%%%%%%%%%%%%%%%%%%%%%%%%%%%%%%%%%%%%%

\section{Overview}

This library has been designed to apply lagrangian interpolation in order to carry out different computations. There are two modules, defined as \textbf{Interpolation} and \textbf{Lagrange\_interpolation}. In spite of this, the API is contained only in the \textbf{Interpolation} module. On the whole, this module contains two main functions: \textbf{interpolated\_value} and \textbf{Integral}. Each of them is based on a lagrangian interpolation, which is permormed in the support module \textbf{Lagrange\_interpolation}. 

There are two main objectives of this API. The first one, attained by the function \textbf{interpolated\_value}, computes the value of a function at a certain point taking into account values of that function at other points. The second purpose, carried out by the function \textbf{Integral}, is related to the computation of the integral of a function in a certain interval.

\section{Example using the API}

For the sake of clarity, a file called \textbf{API\_Example\_Lagrange\_Interpolation.f90} contains an example of how to use this library. For using the API it is necessary to write the sentence \textbf{use Interpolation}.

The first subroutine, called \textbf{interpolated\_Solution}, is devoted to clarify the usage of the function \textbf{interpolated\_value}. Through this function, one could obtain the value of a function at a certain point by means of the values of that function at other points. For instance, the chosen function is a simple cosine, and the inputs of \textbf{interpolated\_value} are four values of the function at other points. The final result is the interpolated value of the function at \textbf{xp}, which is denoted as \textbf{yp}. Since the order of the interpolation has not been defined, it acquires the predefined value of two. 

Subsequently, the second subroutine, coined as \textbf{Integral\_Solution}, carries out the computation of the integral of a certain function in an interval defined by the given points. Again the function is set to be a cosine, and five values of it are known at different points. The function \textbf{Integral} enables to perform the integral computation. Since the degree of the interpolation has not been defined, it acquires the predefined value of two.

\newpage

\par\vspace{\baselineskip}
\lstset{backgroundcolor=\color{Goldenrod!10}}
\lstset{language=Fortran}

\lstset{
	numbers=left,            
	numbersep=5pt,                 
	numberstyle=\tiny\color{mygray}
}


\lstinputlisting[language=Fortran]{../sources/Interpolation_example.f90}
\newpage


\section{Interpolation module}


\subsection*{interpolated\_value}


\lstset{backgroundcolor=\color{Goldenrod!10}}
\lstset{language=Fortran}
\begin{lstlisting}[frame=trBL]
yp = interpolated_value( x, y, xp, degree )
\end{lstlisting}

The function \textbf{interpolated\_value} is devoted to conduct a piecewise polynomial interpolation of the value of a certain function $y(x)$ in $x=x_p$. The data provided to carry out the interpolation is the value of that function $y(x)$ in a group of nodes.


The result of the function is the following:

\begin{table}[H]
	\begin{center}
		\begin{spacing}{1.2}
			\begin{tabular}{| l | l | p{6cm} |}
				
				\hline
				
				\bf Function result & \bf Type & \bf Description \\ \hline \hline
				
				yp & real & Interpolated value of the function $y(x)$ in $x=x_p$.\\ \hline
				
			\end{tabular}
		\end{spacing}
	\end{center}
	\caption{Output of \textbf{interpolated\_value}}
\end{table}

The arguments of the function are described in the following table.


\begin{table}[H]
	\begin{center}
		\begin{spacing}{1.2}
			\begin{tabular}{| l | l | l | p{6cm} |}
				
				\hline
				
				\bf Argument & \bf Type & \bf Intent & \bf Description \\ \hline \hline
				
				x & vector of reals & in & Points in which the value of the function $y(x)$ is provided.\\ \hline
				
				y & vector of reals & in & Values of the function $y(x)$ in the group of points denoted by $x$. \\ \hline
				
				xp & real & in & Point in which the value of the function $y$ will be interpolated. \\ \hline
				
				degree & integer & in (optional) & Degree of the polynomial used in the interpolation. If it is not presented, it takes the value 2. \\ \hline
				
			\end{tabular}
		\end{spacing}
	\end{center}
	\caption{Description of \textbf{interpolated\_value} arguments}
\end{table}





\newpage

%%%%%%%%%%%%%%%%%%%%%%%%%%%%%%%%%%%%%%%%%%%%%%%%%%%%%%%%%%%%%%%%%%%%%%%%%%%%%%%%%%%%%%%%%%%%%%%%%%%%%%%%%%%%%%%%%%%%%%%%%%%%%%%%%%%%%%%%%%%%%%%%%%%%%%

% FUNCTION Integral

%%%%%%%%%%%%%%%%%%%%%%%%%%%%%%%%%%%%%%%%%%%%%%%%%%%%%%%%%%%%%%%%%%%%%%%%%%%%%%%%%%%%%%%%%%%%%%%%%%%%%%%%%%%%%%%%%%%%%%%%%%%%%%%%%%%%%%%%%%%%%%%%%%%%%%

\subsection*{Integral}


\lstset{backgroundcolor=\color{Goldenrod!10}}
\lstset{language=Fortran}
\begin{lstlisting}[frame=trBL]
I = Integral( x, y, degree )
\end{lstlisting}


The function \textbf{Integral} is devoted to conduct a piecewise polynomial integration of a certain function $y(x)$. The data provided to carry out the interpolation is the value of that function $y(x)$ in a group of nodes. The limits of the integral correspond to the minimum and maximum values of the nodes.


The result of the function is the following:

\begin{table}[H]
	\begin{center}
		\begin{spacing}{1.2}
			\begin{tabular}{| l | l | p{6cm} |}
				
				\hline
				
				\bf Function result & \bf Type & \bf Description \\ \hline \hline
				
				I & real & Value of the piecewise polynomial integration of $y(x)$.\\ \hline
				
			\end{tabular}
		\end{spacing}
	\end{center}
	\caption{Output of \textbf{Integral}}
\end{table}



The arguments of the function are described in the following table.


\begin{table}[H]
	\begin{center}
		\begin{spacing}{1.2}
			\begin{tabular}{| l | l | l | p{6cm} |}
				
				\hline
				
				\bf Argument & \bf Type & \bf Intent & \bf Description \\ \hline \hline
				
				x & vector of reals & in & Points in which the value of the function $y(x)$ is provided.\\ \hline
				
				y & vector of reals & in & Values of the function $y(x)$ in the group of points denoted by $x$. \\ \hline
				
				degree & integer & in (optional) & Degree of the polynomial used in the interpolation. If it is not presented, it takes the value 2. \\ \hline
				
			\end{tabular}
		\end{spacing}
	\end{center}
	\caption{Description of \textbf{Integral} arguments}
\end{table}





%%%%%%%%%%%%%%%%%%%%%%%%%%%%%%%%%%%%%%%%%%%%%%%%%%%%%%%%%%%%%%%%%%%%%%%%%%%%%%%%%%%%%%%%%%%%%%%%%%%%%%%%%%%%%%%%%%%%%%%%%%%%%%%%%%%%%%%%%%%%%%%%%%%%%%

% FUNCTION weighted_average

%%%%%%%%%%%%%%%%%%%%%%%%%%%%%%%%%%%%%%%%%%%%%%%%%%%%%%%%%%%%%%%%%%%%%%%%%%%%%%%%%%%%%%%%%%%%%%%%%%%%%%%%%%%%%%%%%%%%%%%%%%%%%%%%%%%%%%%%%%%%%%%%%%%%%%



%\subsection*{weighted\_average}
%
%
%\lstset{backgroundcolor=\color{Goldenrod!10}}
%\lstset{language=Fortran}
%\begin{lstlisting}[frame=trBL]
%f_average = weighted_average( f, w, x, dg )
%\end{lstlisting}
%
%
%The function \textbf{weighted\_average} computes the average of the function $f(x)$ weighted by the function $w(x)$:
%
%
%
%\begin{equation*}
% f_{average}  =\frac{\int f(x) w(x) dx}{\int w(x)}
%\end{equation*}
%
%
%
%
% The data provided to carry out the interpolation is the value of those functions $f(x)$ and $w(x)$ in a group of nodes. The result of the function \textbf{weighted\_average} is the following:
%
%\begin{table}[H]
%	\begin{center}
%	\begin{spacing}{1.2}
%		\begin{tabular}{| l | l | p{6cm} |}
%			
%			\hline
%			
%			\bf Function result & \bf Type & \bf Description \\ \hline \hline
%			
%			f\_average & real & Average of the function $f(x)$ weighted by the function $w(x)$.\\ \hline
%			
%		\end{tabular}
%		\end{spacing}
%	\end{center}
%	\caption{Output of weighted\_average}
%\end{table}
%
%
%The arguments of the function are described in the following table.
%
%
%\begin{table}[H]
%	\begin{center}
%	\begin{spacing}{1.2}
%		\begin{tabular}{| l | l | l | p{6cm} |}
%\hline
%			
%\bf Argument & \bf Type & \bf Intent & \bf Description \\ \hline \hline
%
%f & vector of reals & in & Values of the function $f(x)$ in the group of points denoted by $x$. \\ \hline
%
%w & vector of reals & in & Values of the function $w(x)$ in the group of points denoted by $x$. \\ \hline
%			
%x & vector of reals & in & Points in which the value of the function $y(x)$ is provided.\\ \hline
%
%degree & integer & in (optional) & Degree of the polynomial used in the interpolation. If it is not presented it takes the value 2. \\ \hline
%			
%		\end{tabular}
%		\end{spacing}
%	\end{center}
%	\caption{Description of weighted\_average arguments}
%\end{table}