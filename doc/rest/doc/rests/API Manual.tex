\documentclass{report}

\usepackage{float}
\usepackage[usenames,dvipsnames]{pstricks}
\usepackage{epsfig}
\usepackage{pst-grad} % For gradients
\usepackage{pst-plot} % For axes
\usepackage{graphicx}
\usepackage{listings}
\usepackage[english]{babel} % English language/hyphenation
\usepackage[protrusion=true,expansion=true]{microtype} % Better typography
\usepackage{amsmath,amsfonts,amsthm} % Math packages
%\usepackage[svgnames]{xcolor} % Enabling colors by their 'svgnames'
%\usepackage[usenames,dvipsnames,svgnames,table]{xcolor}
\usepackage{array}

\usepackage[hang, small,labelfont=bf,up,textfont=it,up]{caption} % Custom
% captions under/above floats in tables or figures
\usepackage{booktabs} % Horizontal rules in tables
\usepackage{fix-cm}	 % Custom font sizes - used for the initial letter in the document
\usepackage{color}
\usepackage{sectsty} % Enables custom section titles
\allsectionsfont{\usefont{OT1}{phv}{b}{n}} % Change the font of all section commands
\usepackage{titling} % Allows custom title configuration

\usepackage{fancyhdr} % Needed to define custom headers/footers
\pagestyle{fancyplain} % Enables the custom headers/footers
\usepackage{lastpage} % Used to determine the number of pages in the document (for "Page X of Total")

\graphicspath{ {images/} }
\topmargin 0.0cm
\oddsidemargin 0.2cm
\textwidth 16cm 
\textheight 21cm
\footskip 1.0cm
\definecolor{mygreen}{rgb}{0,0.6,0}
\definecolor{mygray}{rgb}{0.5,0.5,0.5}
\definecolor{mymauve}{rgb}{0.58,0,0.82}


\setlength{\parskip}{2mm} 
\usepackage{mathrsfs}

\lstset{ %
	backgroundcolor=\color{white},   % choose the background color; you must add \usepackage{color} or \usepackage{xcolor}
	basicstyle=\footnotesize,        % the size of the fonts that are used for the code
	breakatwhitespace=false,         % sets if automatic breaks should only happen at whitespace
	breaklines=true,                 % sets automatic line breaking
	captionpos=b,                    % sets the caption-position to bottom
	commentstyle=\color{mygreen},    % comment style
	deletekeywords={...},            % if you want to delete keywords from the given language
	escapeinside={\%*}{*)},          % if you want to add LaTeX within your code
	extendedchars=true,              % lets you use non-ASCII characters; for 8-bits encodings only, does not work with UTF-8
	frame=single,                    % adds a frame around the code
	keepspaces=true,                 % keeps spaces in text, useful for keeping indentation of code (possibly needs columns=flexible)
	keywordstyle=\color{blue}\textbf,       % keyword style
	language=Fortran,                 % the language of the code
	otherkeywords={*,source},            % if you want to add more keywords to the
	% set
	numbers=none,                    % where to put the line-numbers; possible
	% values are (none, left, right)
	numbersep=5pt,                   % how far the line-numbers are from the code
	numberstyle=\tiny\color{mygray}, % the style that is used for the line-numbers
	rulecolor=\color{black},         % if not set, the frame-color may be changed on line-breaks within not-black text (e.g. comments (green here))
	showspaces=false,                % show spaces everywhere adding particular underscores; it overrides 'showstringspaces'
	showstringspaces=false,          % underline spaces within strings only
	showtabs=false,                  % show tabs within strings adding particular underscores
	stepnumber=2,                    % the step between two line-numbers. If it's 1, each line will be numbered
	stringstyle=\color{mymauve},     % string literal style
	tabsize=2,                       % sets default tabsize to 2 spaces
	title=\lstname,                   % show the filename of files included with \lstinputlisting; also try caption instead of title
	linewidth=17cm,
	xleftmargin=-0.5cm
}

%Allow us to make the rules larger
\newlength\FHoffset
\setlength\FHoffset{1cm}


\fancyheadoffset{\FHoffset}
\renewcommand{\headrulewidth}{0.5pt} % No header rule
\renewcommand{\footrulewidth}{0.5pt} % Thin footer rule
% Headers - all currently empty
\lhead{}
\chead{}
\rhead{}
% Footers
\lfoot{}
\cfoot{}
\rfoot{\footnotesize Page \thepage\ of \pageref{LastPage}} % "Page 1 of 2"


\definecolor{DarkGoldenrod}{rgb}{0.8,0.6,0.1}
\definecolor{DarkRed}{rgb}{0.5,0.1,0.1}
\definecolor{DarkRedPart}{rgb}{0.3,0.1,0.2}

\usepackage{lettrine} % Package to accentuate the first letter of the text
\newcommand{\initialgold}[1]{ % Defines the command and style for the first
	% letter
	\lettrine[lines=3,lhang=0.3,nindent=0em]{
		\color{DarkGoldenrod}
		{\textsf{#1}}}{}}
\newcommand{\initialred}[1]{ % Defines the command and style for the first
	% letter
	\lettrine[lines=3,lhang=0.3,nindent=0em]{
		\color{DarkRed}
		{\textsf{#1}}}{}}


%----------------------------------------------------------------------------------------
%	TITLE SECTION
%----------------------------------------------------------------------------------------
\usepackage{titling} % Allows custom title configuration

\newcommand{\HorRule}{\color{DarkGoldenrod} \rule{\linewidth}{1pt}} % Defines the gold horizontal rule around the title

\pretitle{\vspace{-150pt} \begin{flushleft} \HorRule \fontsize{50}{50}
		\usefont{OT1}{phv}{b}{n} \color{DarkRed} \selectfont} % Horizontal rule before the title
	
	\title{How to learn Applied Mathematics trough modern Fortran} % Your article title
	
	\posttitle{\par\end{flushleft}\vskip 0.5em} % Whitespace under the title

\preauthor{\begin{flushleft}\large \lineskip 0.5em \usefont{OT1}{phv}{b}{sl} \color{DarkRed}} % Author font configuration
	
	
	%\author{John Smith, } % Your name
\postauthor{\footnotesize \usefont{OT1}{phv}{m}{sl} \color{Black} % Configuration for the institution name
	Department of Applied Mathematics \\
	School of Aeronautical and Space Engineering \\
	Technical University of Madrid (UPM)
	\par\end{flushleft}\HorRule} % Horizontal rule after the title

\date{} % Add a date here if you would like one to appear underneath the title block

%----------------------------------------------------------------------------------------

\usepackage{setspace}
\usepackage[hidelinks]{hyperref}
 
 
\begin{document}

\maketitle

\newpage

\tableofcontents

\newpage



\chapter{Some basic programs}
\input{first_course}


\chapter{System of Equations}
\section{Overview}

This is a library designed to solve systems of equations.

It has three modules: \textbf{Linear\_systems}, \textbf{Non\_Linear\_Systems} and \textbf{Jacobian\_module}. In spite of this, the API is contained only in the \textbf{Linear\_systems} and in the \textbf{Non\_Linear\_Systems} modules. With the \textbf{Linear\_systems} module the user must be able to solve a linear system of equations. With the \textbf{Non\_Linear\_Systems} module the user must be able to solve a linear system of equations. 

\section{Example using the API}

For the sake of clarity, a file called \textbf{API\_Example\_Systems\_of\_Equations.f90} contains an example of how to use this library. For using the API it is necessary to write the sentence \textbf{use Linear\_systems} and \textbf{use Non\_Linear\_Systems}.

The first example consists of a linear system of equations of four unknowns with four equations. First of all, it is defined the matrix which contains the terms of the equation, and after the solution. In this example:

 \begin{equation*} 
    \left[ \begin{array}{cccc}
    4 & 3 & 6 & 9 \\
    2 & 5 & 4 & 2 \\
    1 & 3 & 2 & 7 \\
    2 & 4 & 3 & 8
    \end{array}\right]  
     \left( \begin{array}{c}
    x  \\
    y  \\
    z  \\
    w 
    \end{array}\right) 
    = 
    \left( \begin{array}{c}
    3  \\
    1  \\
    5  \\
    2 
    \end{array}\right) 
 \end{equation*}
 
 The second example consists in the solution of a nonlinear system of equations defined as follows:
 
  \begin{equation*}
  F_{1}=x^{2}-y{3}-2
  \end{equation*}
  \begin{equation*}
  F_{2}=3xy-z
  \end{equation*}
 \begin{equation*}
 F_{3}=z^{2}-x
 \end{equation*}
 
 \newpage
 
\par\vspace{\baselineskip}
\lstset{backgroundcolor=\color{Goldenrod!10}}
\lstset{language=Fortran}

\lstset{
	numbers=left,            
	numbersep=5pt,                 
	numberstyle=\tiny\color{mygray}
}



\lstinputlisting[language=Fortran]{../sources/Systems_of_equations_example.f90}
\newpage


\section{Linear\_systems module}


\subsection*{LU\_factorization}

\lstset{backgroundcolor=\color{Goldenrod!10}}
\lstset{language=Fortran}
\begin{lstlisting}[frame=trBL]
call LU_factorization( A )
\end{lstlisting}

The subroutine \textbf{LU\_factorization} returns the inlet matrix which has been factored by the LU method. The arguments of the subroutine are described in the following table.

\begin{table}[H]
	\begin{center}
	\begin{spacing}{1.2}
		\begin{tabular}{| l | l | l | p{5cm} |}
			
\hline
			
\bf Argument & \bf Type & \bf Intent & \bf Description \\ \hline \hline
			
A & two-dimensional array of reals & inout & Square matrix to be factored by the LU method.\\ \hline
			
		\end{tabular}
	\end{spacing}
	\end{center}
	\caption{Description of \textbf{LU\_factorization} arguments}
\end{table}

\subsection*{Solve\_LU}

\lstset{backgroundcolor=\color{Goldenrod!10}}
\lstset{language=Fortran}
\begin{lstlisting}[frame=trBL]
x = Solve_LU( A, b )
\end{lstlisting}

The function \textbf{Solve\_LU} finds the solution to the linear system of equations:

\begin{equation*}
	\textbf{A}\cdot\vec{x}=\vec{b}
\end{equation*}

$\textbf{A}$ and $\vec{b}$ are the given values. The result of the function is:

\begin{table}[H]
	\begin{center}
	\begin{spacing}{1.2}
		\begin{tabular}{| l | l | p{6cm} |}
			
			\hline
			
			\bf Function result & \bf Type & \bf Description \\ \hline \hline
			
			x & vector of reals & Solution ($\vec{x}$) of the linear system of equations.\\ \hline
			
		\end{tabular}
	\end{spacing}
	\end{center}
	\caption{Output of \textbf{Solve\_LU}}
\end{table}

The arguments of the function are described in the following table.

\begin{table}[H]
	\begin{center}		
	\begin{spacing}{1.2}
		\begin{tabular}{| l | l | l | p{6cm} |}
			
\hline
			
\bf Argument & \bf Type & \bf Intent & \bf Description \\ \hline \hline
			
A & two-dimensional array of reals & inout & Square matrix $\textbf{A}$ in the previous equation, but it must be facotred \underline{before} using the LU method.\\ \hline
			
b & vector of reals & in & Vector $\vec{b}$ in the previous equation. \\ \hline
			
		\end{tabular}		
	\end{spacing}
	\end{center}
	\caption{Description of \textbf{Solve\_LU} arguments}
\end{table}

The dimensions of $\textbf{A}$ and $\vec{b}$ must match.

\section{Non\_Linear\_Systems module}

\subsection*{Newton}

\lstset{backgroundcolor=\color{Goldenrod!10}}
\lstset{language=Fortran}
\begin{lstlisting}[frame=trBL]
call Newton( F, x0 )
\end{lstlisting}

The subroutine \textbf{Newton} returns the solution of a non-linear system of equations. The arguments of the subroutine are described in the following table.

\begin{table}[H]
	\begin{center}
	\begin{spacing}{1.2}
		\begin{tabular}{| l | l | l | p{7cm} |}
			
			\hline
			
			\bf Argument & \bf Type & \bf Intent & \bf Description \\ \hline \hline
			
			F & vector function: $\mathbb{R}^{N} \rightarrow \mathbb{R}^{N}$ & in & System of equations that wants to be solved. \\ \hline
			
			x0 & vector of reals & inout & Initial point to start the iteration. Besides, this vector will contain the solution of the problem after the call. Its dimension must be $N$. \\ \hline
			
		\end{tabular}		
	\end{spacing}
	\end{center}
	\caption{Description of \textbf{Newton} arguments}
\end{table}

\chapter{Lagrange Interpolation}
%%%%%%%%%%%%%%%%%%%%%%%%%%%%%%%%%%%%%%%%%%%%%%%%%%%%%%%%%%%%%%%%%%%%%%%%%%%%%%%%%%%%%%%%%%%%%%%%%%%%%%%%%%%%%%%%%%%%%%%%%%%%%%%%%%%%%%%%%%%%%%%%%%%%%%

% OVERVIEW

%%%%%%%%%%%%%%%%%%%%%%%%%%%%%%%%%%%%%%%%%%%%%%%%%%%%%%%%%%%%%%%%%%%%%%%%%%%%%%%%%%%%%%%%%%%%%%%%%%%%%%%%%%%%%%%%%%%%%%%%%%%%%%%%%%%%%%%%%%%%%%%%%%%%%%

\section{Overview}

This library has been designed to apply lagrangian interpolation in order to carry out different computations. There are two modules, defined as \textbf{Interpolation} and \textbf{Lagrange\_interpolation}. In spite of this, the API is contained only in the \textbf{Interpolation} module. On the whole, this module contains two main functions: \textbf{interpolated\_value} and \textbf{Integral}. Each of them is based on a lagrangian interpolation, which is permormed in the support module \textbf{Lagrange\_interpolation}. 

There are two main objectives of this API. The first one, attained by the function \textbf{interpolated\_value}, computes the value of a function at a certain point taking into account values of that function at other points. The second purpose, carried out by the function \textbf{Integral}, is related to the computation of the integral of a function in a certain interval.

\section{Example using the API}

For the sake of clarity, a file called \textbf{API\_Example\_Lagrange\_Interpolation.f90} contains an example of how to use this library. For using the API it is necessary to write the sentence \textbf{use Interpolation}.

The first subroutine, called \textbf{interpolated\_Solution}, is devoted to clarify the usage of the function \textbf{interpolated\_value}. Through this function, one could obtain the value of a function at a certain point by means of the values of that function at other points. For instance, the chosen function is a simple cosine, and the inputs of \textbf{interpolated\_value} are four values of the function at other points. The final result is the interpolated value of the function at \textbf{xp}, which is denoted as \textbf{yp}. Since the order of the interpolation has not been defined, it acquires the predefined value of two. 

Subsequently, the second subroutine, coined as \textbf{Integral\_Solution}, carries out the computation of the integral of a certain function in an interval defined by the given points. Again the function is set to be a cosine, and five values of it are known at different points. The function \textbf{Integral} enables to perform the integral computation. Since the degree of the interpolation has not been defined, it acquires the predefined value of two.

\newpage

\par\vspace{\baselineskip}
\lstset{backgroundcolor=\color{Goldenrod!10}}
\lstset{language=Fortran}

\lstset{
	numbers=left,            
	numbersep=5pt,                 
	numberstyle=\tiny\color{mygray}
}


\lstinputlisting[language=Fortran]{../sources/Interpolation_example.f90}
\newpage


\section{Interpolation module}


\subsection*{interpolated\_value}


\lstset{backgroundcolor=\color{Goldenrod!10}}
\lstset{language=Fortran}
\begin{lstlisting}[frame=trBL]
yp = interpolated_value( x, y, xp, degree )
\end{lstlisting}

The function \textbf{interpolated\_value} is devoted to conduct a piecewise polynomial interpolation of the value of a certain function $y(x)$ in $x=x_p$. The data provided to carry out the interpolation is the value of that function $y(x)$ in a group of nodes.


The result of the function is the following:

\begin{table}[H]
	\begin{center}
		\begin{spacing}{1.2}
			\begin{tabular}{| l | l | p{6cm} |}
				
				\hline
				
				\bf Function result & \bf Type & \bf Description \\ \hline \hline
				
				yp & real & Interpolated value of the function $y(x)$ in $x=x_p$.\\ \hline
				
			\end{tabular}
		\end{spacing}
	\end{center}
	\caption{Output of \textbf{interpolated\_value}}
\end{table}

The arguments of the function are described in the following table.


\begin{table}[H]
	\begin{center}
		\begin{spacing}{1.2}
			\begin{tabular}{| l | l | l | p{6cm} |}
				
				\hline
				
				\bf Argument & \bf Type & \bf Intent & \bf Description \\ \hline \hline
				
				x & vector of reals & in & Points in which the value of the function $y(x)$ is provided.\\ \hline
				
				y & vector of reals & in & Values of the function $y(x)$ in the group of points denoted by $x$. \\ \hline
				
				xp & real & in & Point in which the value of the function $y$ will be interpolated. \\ \hline
				
				degree & integer & in (optional) & Degree of the polynomial used in the interpolation. If it is not presented, it takes the value 2. \\ \hline
				
			\end{tabular}
		\end{spacing}
	\end{center}
	\caption{Description of \textbf{interpolated\_value} arguments}
\end{table}





\newpage

%%%%%%%%%%%%%%%%%%%%%%%%%%%%%%%%%%%%%%%%%%%%%%%%%%%%%%%%%%%%%%%%%%%%%%%%%%%%%%%%%%%%%%%%%%%%%%%%%%%%%%%%%%%%%%%%%%%%%%%%%%%%%%%%%%%%%%%%%%%%%%%%%%%%%%

% FUNCTION Integral

%%%%%%%%%%%%%%%%%%%%%%%%%%%%%%%%%%%%%%%%%%%%%%%%%%%%%%%%%%%%%%%%%%%%%%%%%%%%%%%%%%%%%%%%%%%%%%%%%%%%%%%%%%%%%%%%%%%%%%%%%%%%%%%%%%%%%%%%%%%%%%%%%%%%%%

\subsection*{Integral}


\lstset{backgroundcolor=\color{Goldenrod!10}}
\lstset{language=Fortran}
\begin{lstlisting}[frame=trBL]
I = Integral( x, y, degree )
\end{lstlisting}


The function \textbf{Integral} is devoted to conduct a piecewise polynomial integration of a certain function $y(x)$. The data provided to carry out the interpolation is the value of that function $y(x)$ in a group of nodes. The limits of the integral correspond to the minimum and maximum values of the nodes.


The result of the function is the following:

\begin{table}[H]
	\begin{center}
		\begin{spacing}{1.2}
			\begin{tabular}{| l | l | p{6cm} |}
				
				\hline
				
				\bf Function result & \bf Type & \bf Description \\ \hline \hline
				
				I & real & Value of the piecewise polynomial integration of $y(x)$.\\ \hline
				
			\end{tabular}
		\end{spacing}
	\end{center}
	\caption{Output of \textbf{Integral}}
\end{table}



The arguments of the function are described in the following table.


\begin{table}[H]
	\begin{center}
		\begin{spacing}{1.2}
			\begin{tabular}{| l | l | l | p{6cm} |}
				
				\hline
				
				\bf Argument & \bf Type & \bf Intent & \bf Description \\ \hline \hline
				
				x & vector of reals & in & Points in which the value of the function $y(x)$ is provided.\\ \hline
				
				y & vector of reals & in & Values of the function $y(x)$ in the group of points denoted by $x$. \\ \hline
				
				degree & integer & in (optional) & Degree of the polynomial used in the interpolation. If it is not presented, it takes the value 2. \\ \hline
				
			\end{tabular}
		\end{spacing}
	\end{center}
	\caption{Description of \textbf{Integral} arguments}
\end{table}





%%%%%%%%%%%%%%%%%%%%%%%%%%%%%%%%%%%%%%%%%%%%%%%%%%%%%%%%%%%%%%%%%%%%%%%%%%%%%%%%%%%%%%%%%%%%%%%%%%%%%%%%%%%%%%%%%%%%%%%%%%%%%%%%%%%%%%%%%%%%%%%%%%%%%%

% FUNCTION weighted_average

%%%%%%%%%%%%%%%%%%%%%%%%%%%%%%%%%%%%%%%%%%%%%%%%%%%%%%%%%%%%%%%%%%%%%%%%%%%%%%%%%%%%%%%%%%%%%%%%%%%%%%%%%%%%%%%%%%%%%%%%%%%%%%%%%%%%%%%%%%%%%%%%%%%%%%



%\subsection*{weighted\_average}
%
%
%\lstset{backgroundcolor=\color{Goldenrod!10}}
%\lstset{language=Fortran}
%\begin{lstlisting}[frame=trBL]
%f_average = weighted_average( f, w, x, dg )
%\end{lstlisting}
%
%
%The function \textbf{weighted\_average} computes the average of the function $f(x)$ weighted by the function $w(x)$:
%
%
%
%\begin{equation*}
% f_{average}  =\frac{\int f(x) w(x) dx}{\int w(x)}
%\end{equation*}
%
%
%
%
% The data provided to carry out the interpolation is the value of those functions $f(x)$ and $w(x)$ in a group of nodes. The result of the function \textbf{weighted\_average} is the following:
%
%\begin{table}[H]
%	\begin{center}
%	\begin{spacing}{1.2}
%		\begin{tabular}{| l | l | p{6cm} |}
%			
%			\hline
%			
%			\bf Function result & \bf Type & \bf Description \\ \hline \hline
%			
%			f\_average & real & Average of the function $f(x)$ weighted by the function $w(x)$.\\ \hline
%			
%		\end{tabular}
%		\end{spacing}
%	\end{center}
%	\caption{Output of weighted\_average}
%\end{table}
%
%
%The arguments of the function are described in the following table.
%
%
%\begin{table}[H]
%	\begin{center}
%	\begin{spacing}{1.2}
%		\begin{tabular}{| l | l | l | p{6cm} |}
%\hline
%			
%\bf Argument & \bf Type & \bf Intent & \bf Description \\ \hline \hline
%
%f & vector of reals & in & Values of the function $f(x)$ in the group of points denoted by $x$. \\ \hline
%
%w & vector of reals & in & Values of the function $w(x)$ in the group of points denoted by $x$. \\ \hline
%			
%x & vector of reals & in & Points in which the value of the function $y(x)$ is provided.\\ \hline
%
%degree & integer & in (optional) & Degree of the polynomial used in the interpolation. If it is not presented it takes the value 2. \\ \hline
%			
%		\end{tabular}
%		\end{spacing}
%	\end{center}
%	\caption{Description of weighted\_average arguments}
%\end{table}

\chapter{Finite Differences}
\section{Overview}

This is a library is designed to prepare a PDE problem for a future resolution. The finite differences library obtains the discretization, interpolation and derivative of a function and boundary conditions needed to solve a PDE problem. This will be achieved with the subroutines \textbf{Grid\_initialization}, \textbf{Derivative}, \textbf{Dirichlet} and \textbf{Neumann}. 


The library has two modules: \textbf{Finite\_differences} and \textbf{Non\_uniform\_grid}. But the API is contained only in the \textbf{Finite\_differences} module.


\section{Example using the API}

For the sake of clarity, a file called \textbf{API\_Example\_Finite\_Differences.f90} contains an example of how to use this library. For using the API it is necessary to write the sentence \textbf{use Finite\_differences}.

In the following subroutine, denoted as \textbf{Derivative\_example}, two derivatives of a certain function $y(r,\theta)$ are obtained by means of the function \textbf{Derivative}. Firstly, through the function \textbf{Grid\_Initialization}, a discretization of the domain is carried out, both in the variables $r$ and $\theta$. Afterwards, and taking into account the values of the function $y(r,\theta)$ at the nodes of the discretized domain, some derivatives are computed by the function \textbf{Derivative}:

\begin{itemize}
	\item The second derivative with respect to $r$.
	\item The first derivative with respect to $\theta$.
	\item The derivative with restpect to $\theta$ and with respect to $r$.
\end{itemize}

\newpage

\par\vspace{\baselineskip}
\lstset{backgroundcolor=\color{Goldenrod!10}}
\lstset{language=Fortran}

\lstset{
	numbers=left,            
	numbersep=5pt,                 
	numberstyle=\tiny\color{mygray}
}

\lstinputlisting[language=Fortran]{../sources/Derivative_example.f90}


\section{Finite\_differences module}


\subsection*{Grid\_Initalization}

\lstset{backgroundcolor=\color{Goldenrod!10}}
\lstset{language=Fortran}
\begin{lstlisting}[frame=trBL]
call Grid_Initialization( grid_spacing , direction , q , grid_d )
\end{lstlisting}

This subroutine will calculate a set of points within the space domain defined; $[-1,1]$ by default. The arguments of the routine are described in the following table.

\begin{table}[H]
	\begin{center}
		\begin{spacing}{1.2}
			\begin{tabular}{| l | l | l | p{6cm} |}
				
				\hline
				
				\bf Argument & \bf Type & \bf Intent & \bf Description \\ \hline \hline
				
				grid\_spacing & character & in &   Here the grid structure must be chosen. It can be \textbf{'uniform'}   (equally-spaced) or \textbf{'nonuniform'}.  \\ \hline
				
				direction &  character  & in & Selected by user. If the name of the direction has already been used along the program, it will be overwritten. \\ \hline
				
				q & integer & in &   This is the order chosen for the interpolating polynomials. This label is for the software to be sure that the number of nodes ($N$) is greater than the polynomials order (at least $ N = \text{order} +1 $).  \\ \hline
				
				grid\_d & vector of reals  & inout &  Contains the mesh nodes.  \\ \hline
				
				
			\end{tabular}
		\end{spacing}
	\end{center}
	\caption{Description of \textbf{Grid\_Initalization} arguments}
\end{table}

If \textbf{grid\_spacing} is \textbf{'nonuniform'}, the nodes are calculated by obtaining the extrema of the polynomial error associated to the polynomial of degree $N-1$ that the unknown nodes form.

\subsection*{Derivative for 1D grids}

\lstset{backgroundcolor=\color{Goldenrod!10}}
\lstset{language=Fortran}
\begin{lstlisting}[frame=trBL]
call Derivative ( direction , derivative_order , W , Wxi )
\end{lstlisting}

The subroutine \textbf{Derivative} approximates the derivative of a function by using finite differences. It performs the operation:

\begin{equation*}
\frac{\partial^k W}{\partial x^k}= W_{xk}
\end{equation*}

The arguments of the subroutine are described in the following table.

\begin{table}[H]
	\begin{center}
		\begin{spacing}{1.2}
			\begin{tabular}{| l | l | l | p{6cm} |}
				
				\hline
				
				\bf Argument & \bf Type & \bf Intent & \bf Description \\ \hline \hline
				
				direction & character & in &  It selects the direction which composes the grid from the ones that have already been defined.  \\ \hline
				
				derivative\_order &  integer & in & Order of derivation ($k=1$ first derivate, $k=2$ second derivate and so on).\\ \hline
				
				W &  vector of reals & in & Values that the function has at the given points.\\ \hline
				
				Wxi & vector of reals & out & Result. Value of the k-derivate of the given function.\\ \hline
				
			\end{tabular}
		\end{spacing}
	\end{center}
	\caption{Description of \textbf{Derivative} arguments for 1D grids}
\end{table}

\subsection*{Derivative for 2D and 3D grids}

\lstset{backgroundcolor=\color{Goldenrod!10}}
\lstset{language=Fortran}
\begin{lstlisting}[frame=trBL]
call Derivative ( direction , coordinate , derivative_order , W , Wxi )
\end{lstlisting}

The subroutine \textbf{Derivative} approximates the derivative of a function by using finite differences. It performs the operation:

\begin{equation*}
\frac{\partial^k W}{\partial x^k}= W_{xk}
\end{equation*}

The arguments of the subroutine are described in the following table.

\begin{table}[H]
	\begin{center}
		\begin{spacing}{1.2}
			\begin{tabular}{| l | l | l | p{6cm} |}
				
				\hline
				
				\bf Argument & \bf Type & \bf Intent & \bf Description \\ \hline \hline
				
				direction & vector of characters & in & It selects the directions  which compose the grid from the ones that have already been defined. The first component of the vector will be the first coordinete and so on. \\ \hline
				
				coordinate & integer & in & Coordinate at which the derivate is calculated. It can be 1 or 2 for 2D grids and 1, 2 or 3 for 3D grids.  \\ \hline
				
				derivative\_order &  integer & in & Order of derivation ($k=1$ first derivate, $k=2$ second derivate and so on).\\ \hline
				
				W &  N-dimensional array of reals & in & Values that the function has at the given points.\\ \hline
				
				Wxi & N-dimensional array of reals & out & Result. Value of the k-derivate of the given function.\\ \hline
				
			\end{tabular}
		\end{spacing}
	\end{center}
	\caption{Description of \textbf{Derivative} arguments for 2D and 3D grids}
\end{table}

The subroutine is prepared to be called equally in 2D and 3D problems ($N = 2$ or $3$). 

\subsection{Boundary conditions}

%When dealing with a PDE problem, the boundary conditions, BC, should be discretized as well. Sometimes %the BC can be as complex as the main equations, however, there are some coomon conditions that can be %written in an universal notations, so the library can solve them.

This library is capable of discretizating two type of boundary conditions: Dirichlet and Neumann. Previously to calling this libraries, \textbf{Grid\_Initializartion} must be used. The way they are used will be explained below.

\subsection*{Dirichlet }

\lstset{backgroundcolor=\color{Goldenrod!10}}
\lstset{language=Fortran}
\begin{lstlisting}[frame=trBL]
call Dirichlet( coordinate , N , W , f )\end{lstlisting}

A boundary condition type Dirichlet is defined as:

\begin{equation*}
W(\vec{x}_{0},t)= f(\vec{x}_{0}, t)   \hspace{1cm}   \vec{x}_{0} \; \in \; \partial \Omega
\end{equation*}

The subroutine \textbf{Dirichlet} imposes the Dirichlet condition. The arguments of the subroutine are described in the following table.

\begin{table}[H]
	\begin{center}
		\begin{spacing}{1.2}
			\begin{tabular}{| l | l | l | p{6cm} |}
				
				\hline
				
				\bf Argument & \bf Type & \bf Intent & \bf Description \\ \hline \hline
				
				coordinate & integer & in &  It can be 1 or 2. If 1, the boundary condition will be imposed along the coordinate 2 with the coordinate 1 fixed and vece versa. \\ \hline
				
				N &  integer & in & Boundary point at which the codition is imposed.\\ \hline
				
				W &  two-dimensional array of reals & inout & It will contain the solution. After entering the subroutine it will have imposed the boundary condition determined by \textbf{f}. \\ \hline
				
				f &  vector of reals & in & Value of the boundary condition. \\ \hline
				
			\end{tabular}
		\end{spacing}
	\end{center}
	\caption{Description of \textbf{Dirichlet} arguments}
\end{table}

This subroutine only can work with 2D grids.

\subsection*{Neumann for 1D grids}

\lstset{backgroundcolor=\color{Goldenrod!10}}
\lstset{language=Fortran}
\begin{lstlisting}[frame=trBL]
call Neumann( direction , N , W , f )\end{lstlisting}

A boundary condition type Neumann is defined as:

\begin{equation*}
\frac{dW}{dn}(\vec{x}_{0},t)= f(\vec{x}_{0},t)   \hspace{1cm}   \vec{x}_{0} \; \epsilon \; \partial \Omega
\end{equation*}

The subroutine \textbf{Neumann} imposes the Neumann condition. The arguments of the subroutine are described in the following table.

%igual que en el anterior no tengo claro como definir f y W
\begin{table}[H]
	\begin{center}
		\begin{spacing}{1.2}
			\begin{tabular}{| l | l | l | p{5cm} |}
				
				\hline
				
				\bf Argument & \bf Type & \bf Intent & \bf Description \\ \hline \hline
				
				direction & character & in &   It selects the direction which composes the grid from the ones that have already been defined. \\ \hline
				
				N &  integer & in & Boundary point at which the condition is imposed.\\ \hline
				
				W &  vector of reals & inout & It will contain the solution. After entering the subroutine it will have imposed the boundary condition determined by \textbf{f}.\\ \hline
				
				f & vector of reals & in & Value of the boundary condition.\\ \hline
				
			\end{tabular}
		\end{spacing}
	\end{center}
	\caption{Description of \textbf{Neumann} arguments for 1D grids}
\end{table}

\subsection*{Neumann for 2D grids}

\lstset{backgroundcolor=\color{Goldenrod!10}}
\lstset{language=Fortran}
\begin{lstlisting}[frame=trBL]
call Neumann( direction , coordinate , N , W , f )\end{lstlisting}

A boundary condition type Neumann is defined as:

\begin{equation*}
\frac{dW}{dn}(\vec{x}_{0},t)= f(\vec{x}_{0},t)   \hspace{1cm}   \vec{x}_{0} \; \epsilon \; \partial \Omega
\end{equation*}

The subroutine \textbf{Neumann} imposes the Neumann condition. The arguments of the subroutine are described in the following table.

%igual que en el anterior no tengo claro como definir f y W
\begin{table}[H]
	\begin{center}
		\begin{spacing}{1.2}
			\begin{tabular}{| l | l | l | p{6cm} |}
				
				\hline
				
				\bf Argument & \bf Type & \bf Intent & \bf Description \\ \hline \hline
				
				direction & vector of characters & in & It selects the directions  which compose the grid from the ones that have already been defined. The first component of the vector will be the first coordinate and so on.  \\ \hline
				
				coordinate & integer & in &  It can be 1 or 2. If 1, the boundary condition will be imposed along the coordinate 2 with the coordinate 1 fixed and vice versa. \\ \hline
				
				N &  integer & in & Boundary point at which the condition is imposed.\\ \hline
				
				W &  vector of reals & inout & It will contain the solution. After entering the subroutine it will have imposed the boundary condition determined by \textbf{f}.\\ \hline
				
				f & vector of reals & in & Value of the boundary condition.\\ \hline
				
			\end{tabular}
		\end{spacing}
	\end{center}
	\caption{Description of \textbf{Neumann} arguments for 2D grids}
\end{table}




\chapter{Boundary Value Problem}
\section{Overview}

This library is designed to solve both linear and non linear boundary value problems. 
A boundary value problem appears when a equation  in partial derivatives is to be solved inside a region (space domain) according to some constraints which applies to the frontier of this domain (boundary conditions).
The library has a module:  \textbf{Boundary\_value\_problems}, where the API is contained. 
The API consists of 2 subroutines: one to solve linear problems and the other to solve non linear problems. Finally, depending on the inputs of the subroutines, a 1D problem or a  2D problem is solved.

\section{Example using the API}

For the sake of clarity, a file called \textbf{API\_Example\_Boundary\_Value\_Problem.f90} contains an example of how to use this library. For using the API it is necessary to write the sentence \textbf{use Boundary\_value\_problems}. 

This example consists of two boundary value problems: a 1D linear problem and a 2D non linear problem. The 1D linear problem is the Legendre differential equation:

\begin{equation*}      	
(1 - x^2) \frac{\text{d}^2 y}{\text{d} x^2} - 2x \frac{\text{d} y}{\text{d} x} + n (n + 1) y = 0
\end{equation*}

Where $n = 3$ and the boundary conditions are: $y(-1) = - 1$ and $y(1) = 1$. The 2D non linear problem is:

\begin{equation*}      	
\left( \frac{\partial^2 u}{\partial x^2} +  \frac{\partial^2 u}{\partial y^2} \right) u = 0
\end{equation*}

Where the boundary conditions are:

\begin{equation*}      	
u(0,y) = 0 \quad ; \quad 
u(1,y) = y  \quad ; \quad 
\frac{\partial u}{\partial y}(x,0) = 0 \quad ; \quad 
u(x,1) = x  
\end{equation*}


\newpage

\par\vspace{\baselineskip}
\lstset{backgroundcolor=\color{Goldenrod!10}}
\lstset{language=Fortran}

\lstset{
	numbers=left,            
	numbersep=5pt,                 
	numberstyle=\tiny\color{mygray}
}



\lstinputlisting[lastline=118]{../sources/BVP_example.f90}

\newpage
\lstinputlisting[firstline=120]{../sources/BVP_example.f90}


\section{Boundary\_value\_problems  module}

\subsection*{Linear\_Boundary\_Value\_Problem for 1D problems}

\lstset{backgroundcolor=\color{Goldenrod!10}}
\lstset{language=Fortran}
\begin{lstlisting}[frame=trBL]
call Linear_Boundary_Value_Problem( x_nodes, Order, Differential_operator, & 
                                    Boundary_conditions, Solution )  
 \end{lstlisting}   

The subroutine \textbf{Linear\_Boundary\_Value\_Problem} calculates the solution to a linear boudary value problem such as:

\begin{equation*}
\mathscr{L}\left(x,\ U, \ \frac{\partial U}{\partial x}, \ \frac{\partial^2 U}{\partial x^2} \right) = 0
\end{equation*}
\begin{equation*}
f_a\left(U, \ \frac{\partial U}{\partial x}\right)=0 \hspace{1cm} x=a
\end{equation*}
\begin{equation*}
f_b\left(U, \ \frac{\partial U}{\partial x}\right)=0 \hspace{1cm} x=b
\end{equation*}



The arguments of the subroutine are described in the following table.

\begin{table}[H]
	\begin{center}
		\begin{spacing}{1.2}
			\begin{tabular}{| l | p{3cm}| l | p{5cm} |}
				
				\hline
				
				\bf Argument & \bf Type & \bf Intent & \bf Description \\ \hline \hline
				
				x\_nodes & vector of reals & inout &  Contains the mesh nodes.  \\ \hline
				
				Order &  integer  & in & It indicates the order of the finitte differences.  \\ \hline
				
				Differential\_operator & \raggedright real function: $\mathscr{L}\left(x, U,  U_x,  U_{xx} \right)$ & in  & This function is the differential operator of the boundary value problem.   \\ \hline
				
				Boundary\_conditions & \raggedright real function: $f\left(x, U,  U_x \right)$  & in &  In this function, the boudary conditions are fixed. The user must include a conditional sentence which sets $f\left(a,\ U, \ U_x \right) = f_a$ and $f\left(b,\ U, \ U_x \right) = f_b$.  \\ \hline
				
				Solution & vector of reals  & out &  Contains the solution, $U = U(x)$, of the boundary value problem. \\ \hline
				
				
			\end{tabular}
		\end{spacing}
	\end{center}
	\caption{Description of \textbf{Linear\_Boundary\_Value\_Problem} arguments for 1D problems}
\end{table}

\newpage

\subsection*{Linear\_Boundary\_Value\_Problem for 2D problems}

\lstset{backgroundcolor=\color{Goldenrod!10}}
\lstset{language=Fortran}
\begin{lstlisting}[frame=trBL]
call Linear_Boundary_Value_Problem( x_nodes, y_nodes, Order, Differential_operator,  &
                                    Boundary_conditions, Solution ) 
\end{lstlisting}   

The subroutine \textbf{Linear\_Boundary\_Value\_Problem} calculates the solution to a linear boudary value problem in a rectangular domain $[a,b] \times [c,d]$:

\begin{equation*}
\mathscr{L}\left(x,\ y,\ U, \ \frac{\partial U}{\partial x}, \ \frac{\partial U}{\partial y}, \ \frac{\partial^2 U}{\partial x^2}, \ \frac{\partial^2 U}{\partial y^2}, \ \frac{\partial^2 U}{\partial x \partial y} \right) = 0
\end{equation*}
\begin{equation*}
f_{x=a}\left(U, \ \frac{\partial U}{\partial x}\right)=0  \quad ; \quad f_{x=b}\left(U, \ \frac{\partial U}{\partial x}\right)=0  
\end{equation*}
\begin{equation*}
f_{y=c}\left(U, \ \frac{\partial U}{\partial y}\right)=0  \quad ; \quad f_{y=d}\left(U, \ \frac{\partial U}{\partial y}\right)=0  
\end{equation*}



The arguments of the subroutine are described in the following table.

\begin{table}[H]
	\begin{center}
		\begin{spacing}{1.2}
			\begin{tabular}{| l | p{5cm}| l | p{4.5cm} |}
				
				\hline
				
				\bf Argument & \bf Type & \bf Intent & \bf Description \\ \hline \hline
				
				x\_nodes & vector of reals & inout &  Contains the mesh nodes in the first direction of the mesh.  \\ \hline
				
				y\_nodes & vector of reals & inout &  Contains the mesh nodes in the second direction of the mesh.  \\ \hline
				
				Order &  integer  & in & It indicates the order of the finitte differences.  \\ \hline
				
				Differential\_operator & \raggedright real function: $\mathscr{L}\left(x, y, U,  U_x,  U_y,  U_{xx},  U_{yy},  U_{xy} \right)$ & in  & This function is the differential operator of the boundary value problem.   \\ \hline
				
				Boundary\_conditions & \raggedright real function: $f\left(x, y, U,  U_x,  U_y \right)$  & in &  In this function, the boudary conditions are fixed. The user must use a conditional sentence to do it.  \\ \hline
				
				Solution & two-dimensional array of reals  & out &  Contains the solution, $U = U(x,  y)$, of the boundary value problem. \\ \hline
				
				
			\end{tabular}
		\end{spacing}
	\end{center}
	\caption{Description of \textbf{Linear\_Boundary\_Value\_Problem} arguments for 2D problems}
\end{table}


\newpage


\subsection*{Non\_Linear\_Boundary\_Value\_Problem for 1D problems}

\lstset{backgroundcolor=\color{Goldenrod!10}}
\lstset{language=Fortran}
\begin{lstlisting}[frame=trBL]
call Non_Linear_Boundary_Value_Problem( x_nodes, Order, Differential_operator,  & 
                                        Boundary_conditions, Solver, Solution )   
\end{lstlisting}   

The subroutine \textbf{Non\_Linear\_Boundary\_Value\_Problem} calculates the solution to a non linear boudary value problem  in a rectangular domain $[a,b] \times [c,d]$:

\begin{equation*}
\mathscr{L}\left(x,\ U, \ \frac{\partial U}{\partial x}, \ \frac{\partial^2 U}{\partial x^2} \right) = 0
\end{equation*}
\begin{equation*}
f_a\left(U, \ \frac{\partial U}{\partial x}\right)=0 \hspace{1cm} x=a
\end{equation*}
\begin{equation*}
f_b\left(U, \ \frac{\partial U}{\partial x}\right)=0 \hspace{1cm} x=b
\end{equation*}



The arguments of the subroutine are described in the following table.

\begin{table}[H]
	\begin{center}
		\begin{spacing}{1.2}
			\begin{tabular}{| l | p{3cm}| l | p{5cm} |}
				
				\hline
				
				\bf Argument & \bf Type & \bf Intent & \bf Description \\ \hline \hline
				
				x\_nodes & vector of reals & inout &  Contains the mesh nodes.  \\ \hline
				
				Order &  integer  & in & It indicates the order of the finitte differences.  \\ \hline
				
				Differential\_operator & \raggedright real function: $\mathscr{L}\left(x, U,  U_x,  U_{xx} \right)$ & in  & This function is the differential operator of the boundary value problem.   \\ \hline
				
				Boundary\_conditions & \raggedright real function: $f\left(x, U,  U_x \right)$  & in &  In this function, the boudary conditions are fixed. The user must include a conditional sentence which sets $f\left(a,\ U, \ U_x \right) = f_a$ and $f\left(b,\ U, \ U_x \right) = f_b$.  \\ \hline
				
				Solution & vector of reals  & out &  Contains the solution, $U = U(x)$, of the boundary value problem. \\ \hline
				
				
			\end{tabular}
		\end{spacing}
	\end{center}
	\caption{Description of \textbf{Non\_Linear\_Boundary\_Value\_Problem} arguments for 1D problems}
\end{table}


\newpage

\subsection*{Non\_Linear\_Boundary\_Value\_Problem for 2D problems}

\lstset{backgroundcolor=\color{Goldenrod!10}}
\lstset{language=Fortran}
\begin{lstlisting}[frame=trBL]
call Non_Linear_Boundary_Value_Problem( x_nodes, y_nodes, Order, Differential_operator,  
                                        Boundary_conditions, Solver, Solution )
 \end{lstlisting}   

The subroutine \textbf{Non\_Linear\_Boundary\_Value\_Problem} calculates the solution to a non linear boudary value problem such as:

\begin{equation*}
\mathscr{L}\left(x,\ y,\ U, \ \frac{\partial U}{\partial x}, \ \frac{\partial U}{\partial y}, \ \frac{\partial^2 U}{\partial x^2}, \ \frac{\partial^2 U}{\partial y^2}, \ \frac{\partial^2 U}{\partial x \partial y} \right) = 0
\end{equation*}
\begin{equation*}
f_{x=a}\left(U, \ \frac{\partial U}{\partial x}\right)=0  \quad ; \quad f_{x=b}\left(U, \ \frac{\partial U}{\partial x}\right)=0  
\end{equation*}
\begin{equation*}
f_{y=c}\left(U, \ \frac{\partial U}{\partial y}\right)=0  \quad ; \quad f_{y=d}\left(U, \ \frac{\partial U}{\partial y}\right)=0  
\end{equation*}



The arguments of the subroutine are described in the following table.

\begin{table}[H]
	\begin{center}
		\begin{spacing}{1.2}
			\begin{tabular}{| l | p{5cm}| l | p{4.5cm} |}
				
				\hline
				
				\bf Argument & \bf Type & \bf Intent & \bf Description \\ \hline \hline
				
				x\_nodes & vector of reals & inout &  Contains the mesh nodes in the first direction of the mesh.  \\ \hline
				
				y\_nodes & vector of reals & inout &  Contains the mesh nodes in the second direction of the mesh.  \\ \hline
				
				Order &  integer  & in & It indicates the order of the finitte differences.  \\ \hline
				
				Differential\_operator & \raggedright real function: $\mathscr{L}\left(x, y, U,  U_x,  U_y,  U_{xx},  U_{yy},  U_{xy} \right)$ & in  & This function is the differential operator of the boundary value problem.   \\ \hline
				
				Boundary\_conditions & \raggedright real function: $f\left(x, y, U,  U_x,  U_y \right)$  & in &  In this function, the boudary conditions are fixed. The user must use a conditional sentence to do it.  \\ \hline
				
				Solution & two-dimensional array of reals  & out &  Contains the solution, $U = U(x,  y)$, of the boundary value problem. \\ \hline
				
				
			\end{tabular}
		\end{spacing}
	\end{center}
	\caption{Description of \textbf{Non\_Linear\_Boundary\_Value\_Problem} arguments for 2D problems}
\end{table}

\chapter{Cauchy Problem}
\section{Overview}

This library is designed to solve the Cauchy problem. The Cauchy problem is defined as:

\begin{equation*}
\frac{\text{d}\vec{U}}{\text{d}t}=\vec{f}\ (\vec{U},\ t) 
\end{equation*}
\begin{equation*}
\vec{U}=\vec{U}_0
\end{equation*}

The library has two modules: \textbf{Cauchy\_problem} and \textbf{Temporal\_Schemes}. However, the API is contained only in the \textbf{Cauchy\_problem} module.

\section{Example using the API}

For the sake of clarity, a file called \textbf{API\_Example\_Cauchy\_Problem.f90} contains an example of how to use this library. For using the API it is necessary to write the sentence \textbf{use Cauchy\_Problem}. 

This example consists of a trajectory. This problem needs to solve a second degree equation. The problem approach is:

\begin{equation*}
\frac{\text{d}}{\text{d}t}\begin{pmatrix}
U_{1}\\
U_{2}
\end{pmatrix}
=
\begin{bmatrix}
0 & 1 \\
-a \cdot t & 0
\end{bmatrix}
\begin{pmatrix}
U_{1} \\
U_{2}
\end{pmatrix}
+
\begin{pmatrix}
0 \\
b
\end{pmatrix}
\end{equation*}

It is necessary to give an initial condition of position and velocity. In this example:

\begin{equation*}
\begin{pmatrix}
U_{1}(0)\\
U_{2}(0)
\end{pmatrix}
=
\begin{pmatrix}
5 \\
0
\end{pmatrix}
\end{equation*}

Where $U_{1}(t)$ is referred to the position and $U_{2}(t)$ is referred to the velocity.

\newpage

\par\vspace{\baselineskip}
\lstset{backgroundcolor=\color{Goldenrod!10}}
\lstset{language=Fortran}

\lstset{
	numbers=left,            
	numbersep=5pt,                 
	numberstyle=\tiny\color{mygray}
}

\lstinputlisting[language=Fortran]{../sources/Cauchy_example.f90}
\newpage

\section{Cauchy\_problem module}

\subsection*{Cauchy\_ProblemS}

\lstset{backgroundcolor=\color{Goldenrod!10}}
\lstset{language=Fortran}
\begin{lstlisting}[frame=trBL]
call Cauchy_ProblemS ( Time_Domain , Differential_operator , Scheme , Solution )   \end{lstlisting}   

The subroutine \textbf{Cauchy\_ProblemS} calculates the solution to a Cauchy problem. Previously to using it, the initial conditions must be imposed. The arguments of the subroutine are described in the following table.

\begin{table}[H]
	\begin{center}
		\begin{spacing}{1.2}
			\begin{tabular}{| l | p{2.5cm}| l | p{6cm} |}
				
				\hline
				
				\bf Argument & \bf Type & \bf Intent & \bf Description \\ \hline \hline
				
				Time\_Domain & vector of reals & in &  Time domain where the solution wants to be calculated.  \\ \hline
				
				Differential\_operator &  vector function: $\mathbb{R}^{N} \times\mathbb{R} \rightarrow \mathbb{R}^{N}$  & in & It is the funcition $\vec{f}\ (\vec{U},\ t) $ described in the overview.  \\ \hline
				
				Scheme & temporal scheme  & in (optional) & Defines the scheme used to solve the problem. If it is not specified it uses a Runge Kutta of four steps by default.    \\ \hline
				
				Solution & vector of reals  & out &  Contains the solution $\vec{U}(t)$. The first index represents the time, the second index contains the components of the solution.  \\ \hline
				
				
			\end{tabular}
		\end{spacing}
	\end{center}
	\caption{Description of \textbf{Cauchy\_ProblemS} arguments}
\end{table}

\newpage

\subsection{Temporal schemes}

The schemes that are available in the library are listed below. $h$ denotes the time step.

\begin{table}[H]
	\begin{center}
		\begin{spacing}{1.5}
			\begin{tabular}{| l | l | l |}
				
				\hline
				
				\bf Scheme & \bf Name (in the code) & \bf Formula  \\ \hline \hline
				
				Euler & Euler & $  U_{n+1} = U_{n}+ h f(t_{n}, U_{n})$ \\ \hline
				
				Runge Kutta 2  &  Runge\_Kutta2   &  
				$ U_{n+1} = U_{n}+ h(\frac{k_{1}+k_{2}} {2} ) $ \\
				& & $\quad k_{1}= f(t_{n},U_{n}) $ \\
				& & $\quad k_{2}=f(t_{n}+h, U_{n}+hk_{1}) $ \\ \hline
				
				Runge Kutta 4 & Runge\_Kutta4 &  $ U_{n+1} = U_{n}+ \frac{1}{6}h (k_{1}+2k_{2}+2k_{3}+k_{4}) $ \\
				& & $\quad k_{1}= f(t_{n},U_{n}) $ \\
				& & $\quad k_{2}=f(t_{n}+\frac{1}{2}h, U_{n}+\frac{1}{2}hk_{1}) $ \\
				& & $\quad k_{3}=f(t_{n}+\frac{1}{2}h, U_{n}+\frac{1}{2}hk_{2}) $\\
				& & $\quad k_{4}= f(t_{n}+h,U_{n}+hk_{3}) $  \\ \hline
				
				Leap Frog & Leap\_Frog  & $ U_{n+2}=U_{n}+2f(x_{n+1},U_{n+1})$ \\ \hline
				
				Adams Bashforth 2 & Adams\_Bashforth  & $ U_{n+2}=U_{n+1}+ h \left( \frac{3}{2} f(t_{n+1},U_{n+1}) - \frac{1}{2} f(t_{n},U_{n}) \right) $  \\ \hline
				
				Adams Bashforth 3  & Adams\_Bashforth3  & $  U_{n+3} = U_{n+2}+ h \left( \frac{23}{12} f(t_{n+2},U_{n+2}) - \right.$ \\
				& & $\qquad\left.-\frac{4}{3} f(t_{n+1},U_{n+1}) + \frac{5}{12} f(t_{n},U_{n}) \right) $ \\ \hline
				
				Predictor Corrector & Predictor\_Corrector1  & $ \bar{U}_{n+1}=U_{n}+hf(t_{n},U_{n}) $ \\
				& & $ U_{n+1}=U_{n}+\frac{1}{2}h(f(t_{n+1},\bar{U}_{n+1})+f(t_{n},U_{n} )) $\\ \hline
				
				Euler Inverso & Inverse\_Euler & $ U_{n+1}=U_{n}+hf(t_{n+1},U_{n+1}) $\\ \hline
				
				Crank Nicolson & Crank\_Nicolson  & $  U_{n+1} = U_{n}+\frac{1}{2} h( f(t_{n+1}, U_{n+1})+f(t_{n},U_{n})) $ \\ \hline
			\end{tabular}
		\end{spacing}
	\end{center}
	\caption{Description of the available schemes }
\end{table}


\chapter{Initial Value Boundary Problem}
\section{Overview}

This library is designed to solve a boundary initial value problem. 
The initial value boundary problem is composed by equations in partial derivatives which change with time. 
Then, the complexity of this problem mixes the resolution scheme of a Cauchy problem (in order to solve the temporal evolution) with the procedure for solving a boundary value problem whose unknowns
change in every time iteration.
The library has a module:  \textbf{Initial\_Value\_Boundary\_Problem}, where the API is contained. 

\section{Example using the API}

For the sake of clarity, a file called \textbf{Test\_advection\_diffusion\_equation.f90} contains an example of how to use this library. For using the API it is necessary to write the sentence \textbf{use Initial\_Value\_Boundary\_Problem}. 

This example consists of two boundary initial value problems: a 1D problem and a 2D problem. The advection diffusion equation is being solved in both cases. The advection diffusion equation for a 1D grid is:

\begin{equation*}      	
\frac{\partial u}{\partial t} = - u \frac{\partial u}{\partial x} + \nu \frac{\partial^2 U}{\partial x^2}
\end{equation*}

The value given for $\nu$ is $0.01$. The boundary conditions choosen are: $u(-1)=1$ and $\frac{\partial u}{\partial x}(1)=0$ and the initial condition: $u(x,t=0)=0$. For a 2D grid:


\begin{equation*}
\frac{\partial U}{\partial t} =  - u \frac{\partial u}{\partial x} + \nu \left( \frac{\partial^2 U}{\partial x^2} + \frac{\partial^2 U}{\partial y^2} \right)
\end{equation*}

Where $\nu=0.02$, the initial condition is $u(x,y,t=0)=\exp(-25x^2-25y^2)$ and the boundary conditions are:

\begin{equation*}      	
u(-1,y) = 0 \quad ; \quad 
u(1,y) = 0  \quad ; \quad 
u(x,-1) = 0 \quad ; \quad 
u(x,1) = 0  
\end{equation*}


\newpage

\par\vspace{\baselineskip}
\lstset{backgroundcolor=\color{Goldenrod!10}}
\lstset{language=Fortran}

\lstset{
	numbers=left,            
	numbersep=5pt,                 
	numberstyle=\tiny\color{mygray}
}

\lstinputlisting[lastline=123]{../sources/IVBP_example.f90}
\newpage

\lstinputlisting[firstline=125]{../sources/IVBP_example.f90}

%Note: The part of the code concerning the graphics has been removed.




\section{Initial\_Value\_Boundary\_Problem  module}

\subsection*{Initial\_Value\_Boundary\_ProblemS for 1D problems}

\lstset{backgroundcolor=\color{Goldenrod!10}}
\lstset{language=Fortran}
\begin{lstlisting}[frame=trBL]
call Initial_Value_Boundary_ProblemS( Time_Domain, x_nodes, Order, 
                                      Differential_operator, Boundary_conditions,  
                                      Solution )  
                                        \end{lstlisting}   

The subroutine \textbf{Initial\_Value\_Boundary\_ProblemS} calculates the solution to a boudary initial value problem such as:

\begin{equation*}
\mathscr{L}\left(x,\ t,\ U, \ \frac{\partial U}{\partial x}, \ \frac{\partial^2 U}{\partial x^2} \right) = \frac{\partial U}{\partial t}
\end{equation*}
\begin{equation*}
f_a\left(U,\ t,\ \frac{\partial U}{\partial x}\right)=0 \hspace{1cm} x=a
\end{equation*}
\begin{equation*}
f_b\left(U, \ t, \ \frac{\partial U}{\partial x}\right)=0 \hspace{1cm} x=b
\end{equation*}


Besides, an initial condition must be established: $U(x,t = t_0) = U_0 (x)$.
The arguments of the subroutine are described in the following table.

\begin{table}[H]
	\begin{center}
		\begin{spacing}{1.2}
			\begin{tabular}{| l | p{3cm}| l | p{5.5cm} |}
				
				\hline
				
				\bf Argument & \bf Type & \bf Intent & \bf Description \\ \hline \hline
				
				Time\_Domain & vector of reals & in &  Time domain where the solution wants to be calculated.  \\ \hline
				
				x\_nodes & vector of reals & inout &  Contains the mesh nodes.  \\ \hline
				
				Order &  integer  & in & It indicates the order of the finitte differences.  \\ \hline
				
				Differential\_operator & \raggedright real function: $\mathscr{L}\left(x, t, U,  U_x,  U_{xx} \right)$ & in  & This function is the differential operator of the boundary value problem.   \\ \hline
				
				Boundary\_conditions & \raggedright real function: $f\left(x, t, U,  U_x \right)$  & in &  In this function, the boundary conditions are fixed. The user must include a conditional sentence which sets $f\left(a, \ t, \ U, \ U_x \right) = f_a$ and $f\left(b,\ t, \ U, \ U_x \right) = f_b$.  \\ \hline
				
				Scheme & temporal scheme  & in (optional) & Defines the scheme used to solve the problem. If it is not specified it uses a Runge Kutta of four steps by default.    \\ \hline
				
				Solution & two-dimensional array of reals  & out &  Contains the solution, $U = U(x,t)$, of the boundary value problem. \\ \hline
				
				
			\end{tabular}
		\end{spacing}
	\end{center}
	\caption{Description of \textbf{Initial\_Value\_Boundary\_ProblemS} arguments for 1D problems}
\end{table}



\subsection*{Initial\_Value\_Boundary\_ProblemS for 2D problems}

\lstset{backgroundcolor=\color{Goldenrod!10}}
\lstset{language=Fortran}
\begin{lstlisting}[frame=trBL]
call Initial_Value_Boundary_ProblemS( Time_Domain, x_nodes, y_nodes, Order,       & 
                                      Differential_operator, Boundary_conditions, & 
                                      Solution ) 
\end{lstlisting}   

The subroutine \textbf{Initial\_Value\_Boundary\_ProblemS} calculates the solution to a boundary initial value problem in a rectangular domain $[a,b] \times [c,d]$:

\begin{equation*}
\mathscr{L}\left(x,\ y,\ t,\  U, \ \frac{\partial U}{\partial x}, \ \frac{\partial U}{\partial y}, \ \frac{\partial^2 U}{\partial x^2}, \ \frac{\partial^2 U}{\partial y^2}, \ \frac{\partial^2 U}{\partial x \partial y} \right) = \frac{\partial U}{\partial t}
\end{equation*}
\begin{equation*}
f_{x=a}\left(U, \  t, \  \frac{\partial U}{\partial x}\right)=0  \quad ; \quad f_{x=b}\left(U, \ t, \ \frac{\partial U}{\partial x}\right)=0  
\end{equation*}
\begin{equation*}
f_{y=c}\left(U, \ t, \ \frac{\partial U}{\partial y}\right)=0  \quad ; \quad f_{y=d}\left(U, \ t, \ \frac{\partial U}{\partial y}\right)=0  
\end{equation*}


Besides, an initial condition must be established: $U(x, y ,t = t_0) = U_0 (x,y)$.
The arguments of the subroutine are described in the following table.

\begin{table}[H]
	\begin{center}
		\begin{spacing}{1.2}
			\begin{tabular}{| l | p{5cm}| l | p{4cm} |}
				
				\hline
				
				\bf Argument & \bf Type & \bf Intent & \bf Description \\ \hline \hline
				
				Time\_Domain & vector of reals & in &  Time domain where the solution wants to be calculated.  \\ \hline
				
				x\_nodes & vector of reals & inout &  Contains the mesh nodes in the first direction of the mesh.  \\ \hline
				
				y\_nodes & vector of reals & inout &  Contains the mesh nodes in the second direction of the mesh.  \\ \hline
				
				Order &  integer  & in & It indicates the order of the finitte differences.  \\ \hline
				
				Differential\_operator & \raggedright real function: $\mathscr{L}\left(x, y, t, U,  U_x,  U_y,  U_{xx},  U_{yy},  U_{xy} \right)$ & in  & This function is the differential operator of the boundary value problem.   \\ \hline
				
				Boundary\_conditions & \raggedright real function: $f\left(x, y, t, U,  U_x,  U_y \right)$  & in &  In this function, the boudary conditions are fixed. The user must use a conditional sentence to do it.  \\ \hline
				
				Scheme & temporal scheme  & in (optional) & Defines the scheme used to solve the problem. If it is not specified it uses a Runge Kutta of four steps by default.    \\ \hline
				
				Solution & three-dimensional array of reals  & out &  Contains the solution, $U = U(x, y, t)$, of the boundary value problem. \\ \hline
				
				
			\end{tabular}
		\end{spacing}
	\end{center}
	\caption{Description of \textbf{Initial\_Value\_Boundary\_ProblemS} arguments for 2D problems}
\end{table}



\newpage

\subsection{Temporal schemes}

The schemes that are available in the library for both, 1D and 2D problems, are listed below. $h$ denotes the time step.

\begin{table}[H]
	\begin{center}
		\begin{spacing}{1.5}
			\begin{tabular}{| l | l | l |}
				
				\hline
				
				\bf Scheme & \bf Name (in the code) & \bf Formula  \\ \hline \hline
				
				Euler & Euler & $  U_{n+1} = U_{n}+ h f(t_{n}, U_{n})$ \\ \hline
				
				Runge Kutta 2  &  Runge\_Kutta2   &  
				$ U_{n+1} = U_{n}+ h(\frac{k_{1}+k_{2}} {2} ) $ \\
				& & $\quad k_{1}= f(t_{n},U_{n}) $ \\
				& & $\quad k_{2}=f(t_{n}+h, U_{n}+hk_{1}) $ \\ \hline
				
				Runge Kutta 4 & Runge\_Kutta4 &  $ U_{n+1} = U_{n}+ \frac{1}{6}h (k_{1}+2k_{2}+2k_{3}+k_{4}) $ \\
				& & $\quad k_{1}= f(t_{n},U_{n}) $ \\
				& & $\quad k_{2}=f(t_{n}+\frac{1}{2}h, U_{n}+\frac{1}{2}hk_{1}) $ \\
				& & $\quad k_{3}=f(t_{n}+\frac{1}{2}h, U_{n}+\frac{1}{2}hk_{2}) $\\
				& & $\quad k_{4}= f(t_{n}+h,U_{n}+hk_{3}) $  \\ \hline
				
				Leap Frog & Leap\_Frog  & $ U_{n+2}=U_{n}+2f(x_{n+1},U_{n+1})$ \\ \hline
				
				Adams Bashforth 2 & Adams\_Bashforth  & $ U_{n+2}=U_{n+1}+ h \left( \frac{3}{2} f(t_{n+1},U_{n+1}) - \frac{1}{2} f(t_{n},U_{n}) \right) $  \\ \hline
				
				Adams Bashforth 3  & Adams\_Bashforth3  & $  U_{n+3} = U_{n+2}+ h \left( \frac{23}{12} f(t_{n+2},U_{n+2}) - \right.$ \\
				& & $\qquad\left.-\frac{4}{3} f(t_{n+1},U_{n+1}) + \frac{5}{12} f(t_{n},U_{n}) \right) $ \\ \hline
				
				Predictor Corrector & Predictor\_Corrector1  & $ \bar{U}_{n+1}=U_{n}+hf(t_{n},U_{n}) $ \\
				& & $ U_{n+1}=U_{n}+\frac{1}{2}h(f(t_{n+1},\bar{U}_{n+1})+f(t_{n},U_{n} )) $\\ \hline
				
				Euler Inverso & Inverse\_Euler & $ U_{n+1}=U_{n}+hf(t_{n+1},U_{n+1}) $\\ \hline
				
				Crank Nicolson & Crank\_Nicolson  & $  U_{n+1} = U_{n}+\frac{1}{2} h( f(t_{n+1}, U_{n+1})+f(t_{n},U_{n})) $ \\ \hline
			\end{tabular}
		\end{spacing}
	\end{center}
	\caption{Description of the available schemes }
\end{table}




\end{document}


