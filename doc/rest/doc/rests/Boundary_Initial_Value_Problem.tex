\section{Overview}

This library is designed to solve a boundary initial value problem. 
The initial value boundary problem is composed by equations in partial derivatives which change with time. 
Then, the complexity of this problem mixes the resolution scheme of a Cauchy problem (in order to solve the temporal evolution) with the procedure for solving a boundary value problem whose unknowns
change in every time iteration.
The library has a module:  \textbf{Initial\_Value\_Boundary\_Problem}, where the API is contained. 

\section{Example using the API}

For the sake of clarity, a file called \textbf{Test\_advection\_diffusion\_equation.f90} contains an example of how to use this library. For using the API it is necessary to write the sentence \textbf{use Initial\_Value\_Boundary\_Problem}. 

This example consists of two boundary initial value problems: a 1D problem and a 2D problem. The advection diffusion equation is being solved in both cases. The advection diffusion equation for a 1D grid is:

\begin{equation*}      	
\frac{\partial u}{\partial t} = - u \frac{\partial u}{\partial x} + \nu \frac{\partial^2 U}{\partial x^2}
\end{equation*}

The value given for $\nu$ is $0.01$. The boundary conditions choosen are: $u(-1)=1$ and $\frac{\partial u}{\partial x}(1)=0$ and the initial condition: $u(x,t=0)=0$. For a 2D grid:


\begin{equation*}
\frac{\partial U}{\partial t} =  - u \frac{\partial u}{\partial x} + \nu \left( \frac{\partial^2 U}{\partial x^2} + \frac{\partial^2 U}{\partial y^2} \right)
\end{equation*}

Where $\nu=0.02$, the initial condition is $u(x,y,t=0)=\exp(-25x^2-25y^2)$ and the boundary conditions are:

\begin{equation*}      	
u(-1,y) = 0 \quad ; \quad 
u(1,y) = 0  \quad ; \quad 
u(x,-1) = 0 \quad ; \quad 
u(x,1) = 0  
\end{equation*}


\newpage

\par\vspace{\baselineskip}
\lstset{backgroundcolor=\color{Goldenrod!10}}
\lstset{language=Fortran}

\lstset{
	numbers=left,            
	numbersep=5pt,                 
	numberstyle=\tiny\color{mygray}
}

\lstinputlisting[lastline=123]{../sources/IVBP_example.f90}
\newpage

\lstinputlisting[firstline=125]{../sources/IVBP_example.f90}

%Note: The part of the code concerning the graphics has been removed.




\section{Initial\_Value\_Boundary\_Problem  module}

\subsection*{Initial\_Value\_Boundary\_ProblemS for 1D problems}

\lstset{backgroundcolor=\color{Goldenrod!10}}
\lstset{language=Fortran}
\begin{lstlisting}[frame=trBL]
call Initial_Value_Boundary_ProblemS( Time_Domain, x_nodes, Order, 
                                      Differential_operator, Boundary_conditions,  
                                      Solution )  
                                        \end{lstlisting}   

The subroutine \textbf{Initial\_Value\_Boundary\_ProblemS} calculates the solution to a boudary initial value problem such as:

\begin{equation*}
\mathscr{L}\left(x,\ t,\ U, \ \frac{\partial U}{\partial x}, \ \frac{\partial^2 U}{\partial x^2} \right) = \frac{\partial U}{\partial t}
\end{equation*}
\begin{equation*}
f_a\left(U,\ t,\ \frac{\partial U}{\partial x}\right)=0 \hspace{1cm} x=a
\end{equation*}
\begin{equation*}
f_b\left(U, \ t, \ \frac{\partial U}{\partial x}\right)=0 \hspace{1cm} x=b
\end{equation*}


Besides, an initial condition must be established: $U(x,t = t_0) = U_0 (x)$.
The arguments of the subroutine are described in the following table.

\begin{table}[H]
	\begin{center}
		\begin{spacing}{1.2}
			\begin{tabular}{| l | p{3cm}| l | p{5.5cm} |}
				
				\hline
				
				\bf Argument & \bf Type & \bf Intent & \bf Description \\ \hline \hline
				
				Time\_Domain & vector of reals & in &  Time domain where the solution wants to be calculated.  \\ \hline
				
				x\_nodes & vector of reals & inout &  Contains the mesh nodes.  \\ \hline
				
				Order &  integer  & in & It indicates the order of the finitte differences.  \\ \hline
				
				Differential\_operator & \raggedright real function: $\mathscr{L}\left(x, t, U,  U_x,  U_{xx} \right)$ & in  & This function is the differential operator of the boundary value problem.   \\ \hline
				
				Boundary\_conditions & \raggedright real function: $f\left(x, t, U,  U_x \right)$  & in &  In this function, the boundary conditions are fixed. The user must include a conditional sentence which sets $f\left(a, \ t, \ U, \ U_x \right) = f_a$ and $f\left(b,\ t, \ U, \ U_x \right) = f_b$.  \\ \hline
				
				Scheme & temporal scheme  & in (optional) & Defines the scheme used to solve the problem. If it is not specified it uses a Runge Kutta of four steps by default.    \\ \hline
				
				Solution & two-dimensional array of reals  & out &  Contains the solution, $U = U(x,t)$, of the boundary value problem. \\ \hline
				
				
			\end{tabular}
		\end{spacing}
	\end{center}
	\caption{Description of \textbf{Initial\_Value\_Boundary\_ProblemS} arguments for 1D problems}
\end{table}



\subsection*{Initial\_Value\_Boundary\_ProblemS for 2D problems}

\lstset{backgroundcolor=\color{Goldenrod!10}}
\lstset{language=Fortran}
\begin{lstlisting}[frame=trBL]
call Initial_Value_Boundary_ProblemS( Time_Domain, x_nodes, y_nodes, Order,       & 
                                      Differential_operator, Boundary_conditions, & 
                                      Solution ) 
\end{lstlisting}   

The subroutine \textbf{Initial\_Value\_Boundary\_ProblemS} calculates the solution to a boundary initial value problem in a rectangular domain $[a,b] \times [c,d]$:

\begin{equation*}
\mathscr{L}\left(x,\ y,\ t,\  U, \ \frac{\partial U}{\partial x}, \ \frac{\partial U}{\partial y}, \ \frac{\partial^2 U}{\partial x^2}, \ \frac{\partial^2 U}{\partial y^2}, \ \frac{\partial^2 U}{\partial x \partial y} \right) = \frac{\partial U}{\partial t}
\end{equation*}
\begin{equation*}
f_{x=a}\left(U, \  t, \  \frac{\partial U}{\partial x}\right)=0  \quad ; \quad f_{x=b}\left(U, \ t, \ \frac{\partial U}{\partial x}\right)=0  
\end{equation*}
\begin{equation*}
f_{y=c}\left(U, \ t, \ \frac{\partial U}{\partial y}\right)=0  \quad ; \quad f_{y=d}\left(U, \ t, \ \frac{\partial U}{\partial y}\right)=0  
\end{equation*}


Besides, an initial condition must be established: $U(x, y ,t = t_0) = U_0 (x,y)$.
The arguments of the subroutine are described in the following table.

\begin{table}[H]
	\begin{center}
		\begin{spacing}{1.2}
			\begin{tabular}{| l | p{5cm}| l | p{4cm} |}
				
				\hline
				
				\bf Argument & \bf Type & \bf Intent & \bf Description \\ \hline \hline
				
				Time\_Domain & vector of reals & in &  Time domain where the solution wants to be calculated.  \\ \hline
				
				x\_nodes & vector of reals & inout &  Contains the mesh nodes in the first direction of the mesh.  \\ \hline
				
				y\_nodes & vector of reals & inout &  Contains the mesh nodes in the second direction of the mesh.  \\ \hline
				
				Order &  integer  & in & It indicates the order of the finitte differences.  \\ \hline
				
				Differential\_operator & \raggedright real function: $\mathscr{L}\left(x, y, t, U,  U_x,  U_y,  U_{xx},  U_{yy},  U_{xy} \right)$ & in  & This function is the differential operator of the boundary value problem.   \\ \hline
				
				Boundary\_conditions & \raggedright real function: $f\left(x, y, t, U,  U_x,  U_y \right)$  & in &  In this function, the boudary conditions are fixed. The user must use a conditional sentence to do it.  \\ \hline
				
				Scheme & temporal scheme  & in (optional) & Defines the scheme used to solve the problem. If it is not specified it uses a Runge Kutta of four steps by default.    \\ \hline
				
				Solution & three-dimensional array of reals  & out &  Contains the solution, $U = U(x, y, t)$, of the boundary value problem. \\ \hline
				
				
			\end{tabular}
		\end{spacing}
	\end{center}
	\caption{Description of \textbf{Initial\_Value\_Boundary\_ProblemS} arguments for 2D problems}
\end{table}



\newpage

\subsection{Temporal schemes}

The schemes that are available in the library for both, 1D and 2D problems, are listed below. $h$ denotes the time step.

\begin{table}[H]
	\begin{center}
		\begin{spacing}{1.5}
			\begin{tabular}{| l | l | l |}
				
				\hline
				
				\bf Scheme & \bf Name (in the code) & \bf Formula  \\ \hline \hline
				
				Euler & Euler & $  U_{n+1} = U_{n}+ h f(t_{n}, U_{n})$ \\ \hline
				
				Runge Kutta 2  &  Runge\_Kutta2   &  
				$ U_{n+1} = U_{n}+ h(\frac{k_{1}+k_{2}} {2} ) $ \\
				& & $\quad k_{1}= f(t_{n},U_{n}) $ \\
				& & $\quad k_{2}=f(t_{n}+h, U_{n}+hk_{1}) $ \\ \hline
				
				Runge Kutta 4 & Runge\_Kutta4 &  $ U_{n+1} = U_{n}+ \frac{1}{6}h (k_{1}+2k_{2}+2k_{3}+k_{4}) $ \\
				& & $\quad k_{1}= f(t_{n},U_{n}) $ \\
				& & $\quad k_{2}=f(t_{n}+\frac{1}{2}h, U_{n}+\frac{1}{2}hk_{1}) $ \\
				& & $\quad k_{3}=f(t_{n}+\frac{1}{2}h, U_{n}+\frac{1}{2}hk_{2}) $\\
				& & $\quad k_{4}= f(t_{n}+h,U_{n}+hk_{3}) $  \\ \hline
				
				Leap Frog & Leap\_Frog  & $ U_{n+2}=U_{n}+2f(x_{n+1},U_{n+1})$ \\ \hline
				
				Adams Bashforth 2 & Adams\_Bashforth  & $ U_{n+2}=U_{n+1}+ h \left( \frac{3}{2} f(t_{n+1},U_{n+1}) - \frac{1}{2} f(t_{n},U_{n}) \right) $  \\ \hline
				
				Adams Bashforth 3  & Adams\_Bashforth3  & $  U_{n+3} = U_{n+2}+ h \left( \frac{23}{12} f(t_{n+2},U_{n+2}) - \right.$ \\
				& & $\qquad\left.-\frac{4}{3} f(t_{n+1},U_{n+1}) + \frac{5}{12} f(t_{n},U_{n}) \right) $ \\ \hline
				
				Predictor Corrector & Predictor\_Corrector1  & $ \bar{U}_{n+1}=U_{n}+hf(t_{n},U_{n}) $ \\
				& & $ U_{n+1}=U_{n}+\frac{1}{2}h(f(t_{n+1},\bar{U}_{n+1})+f(t_{n},U_{n} )) $\\ \hline
				
				Euler Inverso & Inverse\_Euler & $ U_{n+1}=U_{n}+hf(t_{n+1},U_{n+1}) $\\ \hline
				
				Crank Nicolson & Crank\_Nicolson  & $  U_{n+1} = U_{n}+\frac{1}{2} h( f(t_{n+1}, U_{n+1})+f(t_{n},U_{n})) $ \\ \hline
			\end{tabular}
		\end{spacing}
	\end{center}
	\caption{Description of the available schemes }
\end{table}


