  %*************************************************************************
    \chapter{First examples: calculus and algebra  \label{primeros_programas} }
    %*************************************************************************
   
   
   %*************************************************************************
   \section{My first program: ``Hello world''  \label{hello_world} }
   %*************************************************************************
   
   
   batch bsh 
   
   scripts 
   
   
   gfortran 
   
   
   
   
   %*************************************************************************
   \section{sum of a numeric series  \label{numeric_series} }
   %*************************************************************************
   
   Dar el resultado de la suma de los 100 primeros términos de las siguientes series: 
   \begin{enumerate}    
    \item Serie de números naturales.
    \item Serie de números naturales impares.
    \item Serie numérica donde el término general de la serie es: $ a_n = 1/n^2 $ desde $ n= 1 $.  
    \item Serie numérica donde el término general de la serie es $ a_n = 1/ n! $ desde $ n=1$.  % e 
    \item Serie numérica donde el término general de la serie es
    $ a_n = (-1)^{n+1}/ (2n-1) $ desde $ n=1$.   
   \end{enumerate}         
   
   
   
   
   
   
   %*************************************************************************
   \section{Operaciones con matrices y vectores   \label{vectores_matrices} }
   %*************************************************************************
   
   
   
   
   Considerar los vectores 
   $V, W \in \mathbb{R}^N$ 
   de componentes: 
   $$
   \{ v_i =\frac{1}{i^2}, \ \ i = 1 \ldots  N \},
   $$
   
   $$
   \{ w_i = \frac{(-1)^{i+1}}{2i-1}, \ \ i = 1 \ldots  N \}.
   $$
   
   
   
   Considerar la matriz $A \in {\cal M}_{N \times N} (\mathbb{R})$ donde su término genérico vale 
   $ a_{ij} = (i/N)^j $. 
   Escribir un programa para calcular  las operaciones siguientes con $N=100$: 
   \begin{enumerate}
    \item Suma de  todas las componentes del vector $V$ y del vector $ W$. 
    \item Suma de  todas las componentes de la matriz $A$.   
    \item Suma de  las componentes del vector $W$ mayores que cero.
    \item Producto escalar de los vectores $V$ y $W$.   
    \item Producto escalar del vector $V$ y la columna $N$ de la matriz $A$. 
    \item Suma de las componentes de vector que resulta de  multiplicar la matriz $A$ por el vector $V$.
    \item Traza de la matriz $A$.
   \end{enumerate}
   
   
   
   
   
   
   %*************************************************************************
   \section{dynamic allocation of memory   \label{memoria_dinamica} }
   %*************************************************************************
   
   
   Dada la matriz  $A \in {\cal M}_{M \times M} (\mathbb{R})$  de término genérico 
   $$
   \{ a_{ij} = (i/M)^j, \ \ i=0, \ldots M-1, \ \  j=0, \ldots M-1 \}. 
   $$
   calcular las siguientes operaciones: 
   
   
   \begin{enumerate}    
    \item Calcular 
    $$  \sum_{M=1} ^{10} traza(A) $$ 
    \item Calcular 
    $$  \sum_{M=1} ^{5} traza(A^2) $$ 
    \item Calcular  con $ M=4 $
    $$  traza \left( \sum_{k=1} ^{5} A^k  \right) $$
   \end{enumerate}     
   
   
   
   
   %*************************************************************************
   \section{Piecewise functions  \label{funciones} }
   %*************************************************************************
   
   
   Sean los vectores $X,F  \in \mathbb{R}^{N + 1}$. Las componentes de  $X$ almacenan los valores
   discretos del dominio de definición y  $F$ las imágenes correspondientes
   de la función $F : \mathbb{R} \rightarrow \mathbb{R}$ continua a trozos siguiente:
   
   $$
   F (x) =
   \left\{
   \begin{array}{ll}
   \displaystyle                          
   1 , \hspace{1cm} \hspace{2cm} &  \displaystyle  a \leq x \leq - \frac{\pi}{2},      \\ \\
   \displaystyle                                                            
   \cos{(\pi x)} ,  & \displaystyle - \frac{\pi}{2} < x <  \frac{\pi}{2}, \\ \\
   \displaystyle                                                            
   0 ,              & \displaystyle  \frac{\pi}{2} \leq x \leq b.                                                                                                
   \end{array}                                                                   
   \right.                                                     
   $$ 
   Considerar una partición equiespaciada de la forma:
   $$ \{ x_i = a + i \Delta x, \ \ i = 0 \ldots  N \}, \qquad 
   \displaystyle \Delta x = \frac{b - a}{N}, \qquad a < - \frac{\pi}{2},  
   \qquad b > \frac{\pi}{2}. $$
   
   Se pide calcular la suma; 
   $$
   S_N = \sum _{i=0} ^{N} F_i \Delta x 
   $$
   \begin{enumerate} 
    \item con $ N = 10 $ 
    \item con $ N = 20 $ 
    \item con $ N = 100 $
   \end{enumerate}     
   
   
   
   
   %*************************************************************************
   \section{Series de funciones   \label{series_funciones} }
   %*************************************************************************
   
   
   Aproximar  mediante un desarrollo en serie de potencias de la forma
   \[  f(x) = \sum_{k=0} ^M a_k \  x^k, \qquad \qquad a_k = \frac{  f^{(k)} (0)  }{ k! },  \]  
   las funciones $F : \mathbb{R} \rightarrow \mathbb{R}$, siguientes:    
   \begin{enumerate}
    \item $f(x) = e^x$ y calcular el valor $ f(1) $ con $ M=5$.      
    \item $f(x) = \sin(x)$ y calcular el valor $ f(\pi/2)$  con $M=8$.      
    \item $f(x) = \cosh(x)$ y calcular el valor $ f(1) $  con $M=10$. 
    \item $f(x) = \displaystyle \frac{1}{1 - x}$ y calcular el valor $ f(0.9) $  con $M=20$.  
    \item $f(x) = e^x$ y calcular el valor más preciso de  $ f(1) $ con doble precisión.        
    \item $f(x) = \sin(x)$ y calcular el valor más preciso  $ f(\pi/2)$ con doble precisión.   
    \item $f(x) = \cosh(x)$ y calcular el valor más preciso de  $ f(1) $ con doble precisión.
    \item $f(x) = \displaystyle \frac{1}{1 - x}$ y calcular el valor  más preciso de $ f(0.9) $ con doble precisión.
    
   \end{enumerate}
   
   
   
   %*************************************************************************
   \section{Lectura y escritura de ficheros  \label{files_IO} }
   %*************************************************************************
   
   
   Crear los ficheros de datos ForTran con nombres  \verb|input_1.dat| e 
   \verb|input_2.dat| con la informaci\'on siguiente:
   
   \vspace{1cm}
   
   Contenido del fichero de entrada \verb|input_1.dat| :
   
   \begin{verbatim}
   1      Datos de entrada 1
   2
   3      1.2      3.4     6.2    -14.0    0.1
   4      -25.2    -8.6    5.1     9.9    17.0
   5      -1.0     -2.0    -5.4    -8.6    0.0
   6      3.14     -11.9   -7.0    -12.1   9.2
   7      6.66     5.32    0.001   0.2     0.001
   \end{verbatim}
   
   \vspace{1cm}
   
   Contenido del fichero de entrada \verb|input_2.dat| :
   
   \begin{verbatim}
   1      Datos de entrada 2
   2
   3      1.2      3.4     6.2    -14.0    0.1      4.89   7.54
   4      -25.2    -8.6    5.1    12.0     9.9      12.24  17.0
   5      0.0      34.5    -1.0   -2.0     -43.04   -8.6    0.0
   6      3.14     -11.9   71.0   7.0     17.0      -12.1   9.2
   7      6.66     5.32    0.001   0.2     0.001    0.008   -0.027
   8      54.0     77.1    -9.002  -13.2   0.017    65.53   -0.021
   9      23.04    -51.98  -34.2   9.99    5.34     8.87    3.22 
   \end{verbatim}
   
   \vspace{0.5cm}
   
   
   Escribir un programa que gestione los datos de los ficheros anteriores siguiendo los pasos siguientes:
   
   Declarar las matrices $A\in {\cal M}_{N \times N} (\mathbb{R})$,
   $B\in {\cal M}_{N \times 3} (\mathbb{R})$,
   $C\in {\cal M}_{N \times 2} (\mathbb{R})$
   y los vectores $U, V, W, T \in \mathbb{R}^N$.
   
   Leer el fichero de entrada (\verb| input_1.dat | o \verb| input_2.dat |) de la forma siguiente: 
   
   \begin{enumerate}
    \item Cargar el fichero completo en la matriz $A$.
    \item Cargar las cuatro primeras columnas del fichero en los vectores $U$, $V$ , $W$ y $T$. 
    \item Cargar la primera columna en el vector $T$ y las tres \'ultimas columnas en la
    matriz $B$.
    \item Cargar la segunda columna en el vector $U$ y las dos \'ultimas columnas en la
    matriz $C$.
    \item Cargar las columnas 1, 2 y 4 en la matriz $B$.
   \end{enumerate}
   Adem\'as, el programa debe crear el fichero de salida 
   (\verb| output_1.dat | o \verb| output_2.dat |), 
   donde se ir\'an escribiendo las matrices y vectores de los apartados anteriores. 
   El formato de escritura debe ser el de n\'umeros reales con cinco decimales.
   
   \vspace{0.5cm}
   
   Para el enunciado anterior, escribir los programas siguientes:
   \begin{enumerate}
    \item Programa 1 : Asignaci\'on est\'atica de memoria.   \\
    Ejecutar el programa por separado para los ficheros \verb|input_1.dat| e 
    \verb|input_2.dat|. Para ello modificar las dimensiones y en nombre de los
    ficheros en el programa fuente.  
    \item Programa 2 : Asignaci\'on din\'amica de memoria.   \\
    Ejecutar el programa una \'unica vez para gestionar los datos de
    los ficheros de entrada \verb|input_1.dat| e \verb|input_2.dat|.      
   \end{enumerate}         
   
   
   
   
   %*************************************************************************
   \section{Sistemas lineales de ecuaciones \label{sistemas_lineales} }
   %*************************************************************************
   
   Implementar un módulo para la resolución de sistemas lineales de ecuaciones algebraicas.
   Los métodos de resolución propuestos son el de eliminación Gaussiana, factorización LU,
   factorización LU de la biblioteca {\it Numerical Recipes} y Jacobi.
   
   Para cada método se pide:
   \begin{itemize}
    \item Validar los resultados con varios casos de prueba con dimensiones distintas.
    \item Evaluar tiempos de ejecución.
    \item Comparar resultados con los métodos restantes.          
   \end{itemize}
   
   Aplicación : Estudiar el condicionamiento de sistemas lineales de ecuaciones para matrices
   aleatorias y de Vandermonde. 
   
   
   
   
   
   %*************************************************************************
   \section{Sistemas no lineales de ecuaciones \label{sistemas_no_lineales} }
   %*************************************************************************
   
   
   
   Implementar un módulo para la resolución numérica de ecuaciones no lineales.
   Para funciones  $F:\mathbb{R} \rightarrow \mathbb{R}$, los métodos de resolución propuestos son el de la bisección y el de 
   Newton-Raphson.   
   Para funciones  $F:\mathbb{R}^N \rightarrow \mathbb{R}^N$, se proponen el método de Newton-Raphson con matriz Jacobiana analítica y 
   el método de Newton-Raphson con matriz Jacobiana numérica.
   Para la validación de los métodos propuestos, se pide implementar un módulo con al menos tres funciones $F:\mathbb{R} \rightarrow \mathbb{R}$
   y al menos tres funciones $F:\mathbb{R}^N \rightarrow \mathbb{R}^N$. Este módulo debe contener las derivadas y matrices Jacobianas correspondientes de las funciones propuestas.  
   
   En el informe correspondiente, presentar tablas de soluciones numéricas en cada paso de iteración para las funciones de prueba propuestas.
   
   
   
   
   
   
   %*************************************************************************
   \section{Autovalores y autovectores \label{autovalores_autovectores} }
   %*************************************************************************
   
   
   Implementar un módulo para el cálculo de autovalores y autovectores de una matriz.
   Los métodos de resolución propuestos son el método de la potencia y el método de la potencia inversa.
   Implementar el método de la potencia inversa a partir de la matriz inversa y resolviendo el sistema lineal
   correspondiente. 
   
   Para cada método se pide:
   \begin{itemize}
    \item Validar los resultados con varios casos de prueba con dimensiones distintas.
    \item Evaluar tiempos de ejecución. Comparar tiempos de ejecución del método de la potencia inversa
    mediante los dos algoritmos propuestos : matriz inversa y solución del sistema lineal.       
   \end{itemize}
   
   Aplicación : Estudiar el condicionamiento de sistemas lineales de ecuaciones para matrices
   aleatorias y de Vandermonde. Calcular la relación $\lambda_{max} / \lambda_{min}$ de los casos de prueba
   presentados en el hito 1 y relacionar y discutir los resultados. 
   
   
   
   \begin{verbatim} 
   subroutine   power_method 
   
   integer :: i, j, k  
   integer, parameter :: PI = 4 * atan(1d0) 
   integer, parameter :: N = 20
   real :: x(0:N), Vandermonde(0:N, 0:N), sigma   
   real :: a=-1, b=1 
   real V(0:N), V0(0:N) 
   
   
   x = [ ( a + (b-a)*i/N, i=0, N) ] 
   
   forall(i=0:N, j=0:N) Vandermonde(i,j) = x(i)**j
   
   V = 1 
   V0 = 0
   do while( abs(norm2(V)-norm2(V0)) > 1d-5 )  
   V0 = V
   V = matmul( Vandermonde, V ) / norm2(V) 
   write(*,*) maxval(V) 
   end do 
   sigma = dot_product( V, V ) 
   write(*,*) "sigma = ", sigma 
   
   
   end subroutine 
   
   \end{verbatim} 
   
   
   
   
   %*************************************************************************
   \section{Derivación numérica \label{derivacion_numerica} }
   %*************************************************************************
   
   \begin{enumerate}   
    
    \item Obtener las fórmulas de las derivadas numéricas primeras descentradas, con tres puntos
    equiespaciados a una distancia $\Delta x $.
    
    \item A partir de la función $f(x) = \mbox{e}^{x}$ en el punto $x=0$, 
    representar gráficamente el error total de las derivadas numéricas 
    frente al valor de $\Delta x$ en precisión simple y doble.
    En particular, representar gráficamente las derivadas primeras adelantada (definición de derivada), centrada y descentradas y la derivada segunda,
    con tres puntos equiespaciados a una distancia $\Delta x $. 
    Discutir los resultados obtenidos.
    
    
    
    \item Resolver los problemas de contorno en ecuaciones diferenciales ordinarias siguientes:
    
    \begin{itemize}
        
        \item {\bf Problema 1:}          
        $$
        u^{\prime\prime} + u = 0,\quad
        x\in[-1,1], \qquad \qquad u(-1)=1,\quad u(1)=0,
        $$
        
        \item {\bf Problema 2:}     
        $$
        u^{\prime\prime} + u^{\prime} - u=\sin(2 \pi x),\quad
        x\in[-1,1], \qquad \qquad u(-1)=0,\quad u^{\prime}(1)=0.
        $$
    \end{itemize}
    
    
    Para los problemas citados anteriormente se pide: 
    
    \begin{enumerate}
        \item A partir de las derivadas numéricas con tres puntos equiespaciados escribir el sistema de
        ecuaciones resultante.
        
        \item Obtener la solución numérica mediante la resolución de un sistema lineal de
        ecuaciones, con $N = 10$ y $N = 100$. 
        
        \item Representar gráficamente los resultados obtenidos.
    \end{enumerate}
    
   \end{enumerate}
   
   
   
   
   %*************************************************************************
   \section{Integración numérica \label{integracion_numerica} }
   %*************************************************************************
   
   
   
   Implementar un módulo para la resolución numérica de integrales definidas de funciones $F:\mathbb{R} \rightarrow \mathbb{R}$.
   Los métodos de resolución propuestos son las reglas del rectángulo, punto medio, trapecio y Simpson.
   Implementar un módulo de funciones $F:\mathbb{R} \rightarrow \mathbb{R}$ de prueba para validar los métodos numéricos
   propuestos. Este módulo debe contener al menos tres funciones con funciones primitivas conocidas y una función cuya 
   función primitiva sea desconocida.
   
   Evaluar el error de las soluciones numéricas para cada método propuesto y para distintos valores del incremento de la
   partición.   
   
   
   
   
   
    
    
    
    
    %*************************************************************************
    \section{to be included  \label{todo} }
    %*************************************************************************
    elemental 
    advance = no 
    dummy versus actual 
    assumed shape explicit shape
    
    global, local and scope in modules 
    
    1 versus 1. 
    
    tab instead of blanks 
    
    brackets 
    
    mask in intrinsic functions 
    
    ! and \& 
    
    ; 
    
    camel case versus underscore 
    
    
    overloading 
    
    forall parallel 
    
    enter matrices by row or columns 
    
    
    lower bound: upper bound 
    
    
    array operations C = A + B 
    
    public versus private 
    
    
    encapsulamiento y ocultaci%ón de datos 
    
    
    FORALL (I=1:N-1, J=1:N, J>I) A(I,J) = A(J,I)
    
    
    \begin{verbatim} 
    x   = 1.23456789123456789123456789Q00
    xd  = 1.23456789123456789123456789Q00
    xdd = 1.23456789123456789123456789Q00
    write(*, '(ES)' ) x  ! scientific notation 1.234 (first digit should be greater or equal than one ) 
    write(*, '(E)' ) x   ! exponential normalized  notation 0.1234 (first digit is zero ) 
    
    write(*,*) " Single, double and quadruple precision "
    write(*, '(E)' ) x 
    write(*, '(E)' ) xd
    write(*, '(E)' ) xdd
    
    \end{verbatim}
    
    
    
    \begin{verbatim}
    
    !*****************************************************************************
    !*
    !*****************************************************************************
    subroutine type_element 
    
    
    
    
    interface operator (+) 
    module procedure element
    end interface 
    
    character(len=20) :: name 
    real (kind=4) :: x 
    real (kind=8) :: xd 
    real (kind=16) :: xdd 
    
    real :: a, b, c; 
    !  real :: a1, a2, a3; 
    
    
    type (person) :: father = person( "juan"), mother = person("cris")  
    
    associate ( name1 => father%name, name2=>mother%name )
    name = trim(name1) // trim(name2)  
    end associate 
    
    a = 1; b = 1 ; c = 1; 
    associate ( a1 =>a, a2 =>b, a3=>c ) 
    a3 = a1 + a2 
    write(*,*) " a3 = ", a3 
    end associate 
    write(*,*) " c = ", c 
    !    write(*,*) " a3 = ", a3 
    
    !   write(*,*) " father =", father % name
    write(*,*) " element =", father + mother 
    write(*,*) " name =", name 
    
    x   = 1.23456789123456789123456789Q00
    xd  = 1.23456789123456789123456789Q00
    xdd = 1.23456789123456789123456789Q00
    write(*, '(ES)' ) x  ! scientific notation 1.234 (first digit should be greater or equal than one ) 
    write(*, '(E)' ) x   ! exponential normalized  notation 0.1234 (first digit is zero ) 
    
    write(*,*) " Single, double and quadruple precision "
    write(*, '(E)' ) x 
    write(*, '(E)' ) xd
    write(*, '(E)' ) xdd
    
    write(*, '(3(E, :, ","))' ) x, xd, xdd
    
    end subroutine 
    
    
    
    !*****************************************************************************
    !*
    !*****************************************************************************
    subroutine Hello_world 
    
    
    write(*,'(a)', advance='no') " Hello world..... " 
    write(*,'(a)', advance='no') " press enter"
    read(*,*) 
    
    end subroutine 
    
    
    
    !*****************************************************************************
    !*
    !***************************************************************************** 
    subroutine  commandline
    
    character(len=256) :: line,  enval
    integer :: i, iarg, stat, clen, len
    integer :: estat, cstat
    
    iarg = command_argument_count()
    write(*,*) " iarg = ", iarg 
    do i=1,iarg
    call get_command_argument(i,line,clen,stat)
    write (*,'(I0,A,A)') i,': ',line(1:clen)
    end do
    call get_command(line,clen,stat)
    write (*,'(A)') line(1:clen)
    
    call get_environment_variable('HOSTNAME',enval,len,stat)
    if (stat == 0) write (*,'(A,A)') 'Host=', enval(1:len)
    call get_environment_variable('USER',enval,len,stat)
    if (stat == 0) write (*,'(A,A)') 'User=', enval(1:len)
    
    
    ! call execute_command_line('ls -al', .TRUE., estat, cstat)
    call execute_command_line('dir', .TRUE., estat, cstat)
    if (estat==0) write (*,'(A)') "Command completed successfully"
    
    end subroutine 
    
    !*****************************************************************************
    !*
    !***************************************************************************** 
    subroutine  allocate_characteristics 
    
    real, allocatable :: V(:), A(:, :)
    character(:), allocatable :: S 
    integer :: i, j, N 
    
    real, pointer ::  B(:,:), Diagonal(:)  
    real, pointer :: memory(:) 
    
    real :: x, y, z 
    class(*), pointer :: p1(:)
    
    
    N = 10  
    V = [ ( i/real(N), i=1, N ) ]  ! automatic allocation allocate( V(N) ) 
    write(*,* ) " V = ", V
    
    N = 2  
    A = reshape( [ (  ( (i/real(N))**j ,i=1, N ), j=1, N ) ], [N, N] )  
    do i=1, N 
    write(*,* ) " A = ", A(i,:) 
    end do 
    
    N = 4  
    A = reshape( [ (  ( (i/real(N))**j ,i=1, N ), j=1, N ) ], [N, N] )  
    do i=1, N 
    write(*,'(A, 100f6.2)' ) " A = ", A(i,:) 
    end do  
    
    S = "Hello world"
    write(*,*) " S = ", S, len(S)   
    
    S = "Hello"
    write(*,*) " S = ", S, len(S)
    
    
    allocate( memory(1:N*N) ) 
    B(1:N,1:N) => memory  
    memory = 0
    forall(i=1:N) B(i,i) = 10. 
    do i=1, N 
    write(*,'(A, *(f6.2) )' ) " B = ", B(i,:) 
    end do 
    diagonal => memory(::N+1) 
    write(*,'(A, 100f6.2)' ) " diagonal = ", diagonal
    write(*,'(A, 100f6.2)' ) " trace = ", sum(diagonal)
    write(*,'(A, 100f6.2)' ) " trace = ", sum(memory(::N+1))
    
    x = 1; y = 2; z = 3; 
    
    write(*,'("i=",I0,", REALs=",*(G0,1X),"....")') i, x, y, z  ! C++ style 
    
    
    allocate( integer :: p1(5) ) ! p1 is an array of integers 
    
    select type (p1)
    type is (integer)
    p1 = [( i, i=1, size(p1) ) ] 
    write(*,'(" p1 = ", *(I0, 1x) )')  p1 
    
    class default
    stop 'Error in type selection'
    end select
    
    deallocate(p1) 
    allocate( real :: p1(3) ) ! now p1 is an array of reals 
    
    select type (p1)
    type is (real)
    p1 = [( i, i=1, size(p1) )  ] 
    write(*,'(" p1 = ", *(G0, 1x) )')  p1 
    
    class default
    stop 'Error in type selection'
    end select
    
    
    
    end subroutine 
    
    \end{verbatim}
    
    
    
      
