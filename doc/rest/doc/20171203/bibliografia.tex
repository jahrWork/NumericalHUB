
%**********************************************
%*  Bibliografia
%**********************************************
\addcontentsline{toc}{chapter}{Bibliograf�a}

%*****************************************************************************
\begin{thebibliography}{99}
%*****************************************************************************

%****
%\bibitem{Aho}
%{ \sc Aho, A.V., Hpcroft, J.E. y Ullman, J. D.}
%(1988)
%Estructuras de datos y algoritmos. 
%Addison--Wesley. 

%***
%\bibitem{Bird}
%{ \sc Bird, R. y Wadler, Ph.}
%(1988)
%Introduction fo Funcional Programming.
%Prentice--Hall.

%****
\bibitem{Bishop}
{\sc Bishop, P.} 
(1989) 
Conceptos de Inform�tica.
Anaya Multimedia.

%****
\bibitem{Borse}
{\sc   Borse, G. J. }
(1989)
Programaci�n en FORTRAN 77
con Aplicaciones de C�lculo Num�rico en Ciencias e Ingenier�a.
Anaya Multimedia.

%****
\bibitem{Cuevas}
{\sc Cuevas, G.} 
(1991)
Ingenier�a del software: pr�ctica de la programaci�n.  
RA--MA.

%****
\bibitem{Dowd}
{\sc Dowd, K.} 
(1993)
High performance computing. 
O'Reilly \& Associates, Inc.
 
%****
\bibitem{ESA}
{\sc  ESA.}(1991) 
ESA Software Engineering Standards.
PSS-05-0. Issue 2.

%****
\bibitem{Etter}
{\sc Etter, Delores M.} 
(1997)
Structured FORTRAN 77 for engineers and scientists.
Addison--Wesley.

%****
%\bibitem{Fokker}
%{ \sc Fokker, J.}
%(1995)
%Programaci�n funcional.
%Vakgroep Informatica. Univ. Utrecht.

%****
\bibitem{Merayo}
{\sc Garc�a Merayo, F..}
(1990)
Programaci�n en FORTRAN 77.
Paraninfo.

%****
\bibitem{Golden}
{\sc Golden, James T.}  
(1976)
FORTRAN IV. Programaci�n y c�lculo. 
 Urmo, S.A. de ediciones.

%****
%\bibitem{Hansen}
%{ \sc Hansen, K.}
%(1986)
%Data structured program design. Ken Orr \& Associates, Inc. 

%****
\bibitem{Joyanes}
{ \sc Joyanes, L. }
(1987)
Metodolog�a de la programaci�n; diagramas de flujo, 
algoritmos y programaci�n estructurada. 
Mc Graw--Hill.

%****
\bibitem{Kernighan}
{\sc Kernighan,  B. W. y Ritchie, D. M.}  
(1988)
The C  programming language.
Prentice--Hall. 

%****
\bibitem{Knuth}
{\sc Knuth, Donald E.} 
(1997)
The art of computing programming. 
Volume 1: Fundamental algorithms. 
Volume 2: Seminumerical algorithms. 
Addison--Wesley.

%****
\bibitem{Metcalf}
{\sc Metcalf, M. y Reid, J.}
(1999)
FORTRAN 90/95 explained.
OXFORD. University Press. 

%****
\bibitem{NASA}
{\sc  NASA. }
(1992) 
Recommended Approach to Software Development. Revision 3.
Software Engineering Laboratory Series. SEL--81--305.

%****
\bibitem{Piattini}
{\sc Piattini, Mario G. y Daryanani, Sunil N.}  
(1995)
Elementos y herramientas en el desarrollo de sistemas de informaci�n. 
Una visi�n actual de la tecnolog�a CASE.  
RA--MA.

%****
\bibitem{Pressman}
{\sc Pressman, Roger S.}  
(1997)
Ingenier�a del software. 
Un enfoque pr�ctico.  
Mc. Graw--Hill.

%****
\bibitem{Schofield}
{\sc Schofield, C. F.} 
(1989)
Optimising fortran programs. 
John Wiley \& Sons. 

%****
\bibitem{wirth}
{\sc   Wirth, N. }
(1980)
Algoritmos + Estructuras de Datos = Programas.
Ediciones del Castillo.

%****
\bibitem{Yourdon}
{\sc Yourdon, E.}  
(1989)
Managing the structrured techniques. 
Prentice--Hall.

\end{thebibliography}