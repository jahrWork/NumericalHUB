

\newcommand{\onefigure}[3]{%%%%%%%%%%%%%%%%%
    \begin{figure}[h]
        \centering
        \captionsetup{width=0.5\textwidth}
        \begin{minipage}{0.5\textwidth}
            \includegraphics[width=\textwidth]{#1/#2}   
            %   \subcaption{}
        \end{minipage}
        \caption[#3]{#3}
\end{figure} }

\newcommand{\twofigures}[4]{%%%%%%%%%%%%%%%%%
    \begin{figure}[h]
        \centering
        \captionsetup{width=0.85\textwidth}
        \begin{minipage}{0.4\textwidth}
            \includegraphics[width=\textwidth]{#1/#2}
            \vspace{-0.5cm}     
            \subcaption{}
        \end{minipage}
        \hspace{0.05\textwidth}
        \begin{minipage}{0.4\textwidth}
            \includegraphics[width=\textwidth]{#1/#3} 
            \vspace{-0.5cm} 
            \subcaption{}
        \end{minipage}
        \caption[#4]{#4}
\end{figure} }


\newcommand{\fourfigures}[6]{%%%%%%%%%%%%%%%%%
    \begin{figure}[h]
        \centering
        \captionsetup{width=0.85\textwidth}
        \begin{minipage}{0.4\textwidth}
            \includegraphics[width=\textwidth]{#1/#2}   
            \vspace{-0.5cm} 
            \subcaption{}
        \end{minipage}
        \hspace{0.05\textwidth}
        \begin{minipage}{0.4\textwidth}
            \includegraphics[width=\textwidth]{#1/#3}
            \vspace{-0.5cm}     
            \subcaption{}
        \end{minipage}
        \begin{minipage}{0.4\textwidth}
            \includegraphics[width=\textwidth]{#1/#4}
            \vspace{-0.5cm}     
            \subcaption{}
        \end{minipage}
        \hspace{0.05\textwidth}
        \begin{minipage}{0.4\textwidth}
            \includegraphics[width=\textwidth]{#1/#5}
            \vspace{-0.5cm} 
            \subcaption{}
        \end{minipage}  
        \caption[#6]{#6}
\end{figure} }




\newcommand{\wfigures}[2]
{
    \begin{figure}[h]
        \centering
        \captionsetup{width=0.85\textwidth}
        \begin{minipage}{0.4\textwidth}
            \includegraphics[width=\textwidth]{#1/#2}   
            \subcaption{}
        \end{minipage}
     \end{figure}
}


%\newcommand{\oneloggrapht}[5]{
%   \begin{tikzpicture}[scale = 0.9]
%   \begin{axis}[ymode=log, title style={at={(0.5,-0.4)},anchor=south}, title={#5}, {#1}]
%   \foreach \x in {#4}
%   \addplot table [skip first n=1, x index=0,y index=\x]{#2};
%   \legend{#3}
%   \end{axis}
%   \end{tikzpicture}
%}
\newcommand{\mixgraphs}[8]
{
	\begin{tikzpicture}[]
	\begin{axis}[ width = \textwidth, title style={at={(0.5,-0.5)},anchor=south}, title={#8}, xlabel={#1}, ylabel={#2} , view={0}{90}, legend style={font=\small}, cycle list name = black white, {#3}]
	\foreach \x/\y in {#6}
	\addplot table [skip first n=1, x index=\x,y index=\y]{#4};
	{#7}
	\legend{#5}
	\end{axis}
	\end{tikzpicture}
}

\newcommand{\onegraph}[7]{
    \begin{tikzpicture}[]
    \begin{axis}[ width = \textwidth, title style={at={(0.5,-0.45)},anchor=south}, title={#7}, xlabel={#1}, ylabel={#2}, {#3}, legend style={font=\small}, cycle list name = black white ]
    \foreach \x in {#6}
    \addplot table [skip first n=1, x index=0,y index=\x]{#4};
    \legend{#5}
    \end{axis}
    \end{tikzpicture}
}%

\newcommand{\curveplot}[6]
{
	\addplot3[ domain = #1, domain y = #2, #3]({ #4 },{ #5 },{ #6 });
}

\newcommand{\vectorplot}[8]
{
	\addplot3[ domain = #1, domain y = #2, #3,
	quiver={
		u={ #4 },
		v={ #5 } ,
		w={ #6 } ,
		scale arrows= #7
	},samples=#8, -stealth]{0};
}


\newcommand{\functiongraph}[7]
{
	\begin{tikzpicture}[]
	\begin{axis}[ title style={at={(0.5,-0.4)},anchor=south}, title={#7}, xlabel={#1}, ylabel={#2}, legend style={font=\small},  {#3}]
	\foreach \f in {#4}
	\addplot {\f};
	{#6}
	\legend{#5}
	\end{axis}
	\end{tikzpicture}
}%
\newcommand{\oneloggraph}[7]{
    \begin{tikzpicture}[]
    \begin{axis}[ width = \textwidth, ymode = log, title style={at={(0.5,-0.45)},anchor=south}, title={#7}, xlabel={#1}, ylabel={#2}, {#3}, legend style={font=\small},  cycle list name = black white]
    \foreach \x in {#6}
    \addplot table [skip first n=1, x index=0,y index=\x]{#4};
    \legend{#5}
    \end{axis}
    \end{tikzpicture}
}


\newcommand{\onecontour}[5]{
    \begin{tikzpicture}[]
    \begin{axis}[ 
                  height = \textwidth, width = \textwidth, 
                  title style={at={(0.5,-0.45)}, yshift=-0.05}, title={#5},
                  view={0}{90}, #1, 
                  colormap/blackwhite]
  %  \addplot3[smooth,surf,mesh/rows=#2, shader =interp,z buffer=sort,#4] table [skip first n=2] {#3}; 
  %  \addplot3[contour gnuplot = { levels={-8., -4., -1.6,  0.0,  1.6, 4., 8. }},  
  % \addplot3[contour gnuplot, domain=-3:3, data cs=cart 
   \addplot3[contour/labels=false, contour gnuplot={number=9}, thick,
              mesh/ordering = y varies,
              mesh/rows = #2, 
              mesh/cols = #2
             ] 
             table [skip first n=2] {#3};
    \end{axis}
    \end{tikzpicture}
}

%\addplot3[contour gnuplot,domain=-3:3,
%	  data cs=cart]
 
%                          title style={at={(0.5,-0.4)}, yshift=-0.05}, title={(a)},
%                          xlabel={ $x$ },
%                          ylabel={ $y$ },
%                          view={0}{90},
%                          colormap/blackwhite,
%                          ]
%               \addplot3[contour gnuplot = { levels={-8., -4., -1.6,  0.0,  1.6, 4., 8. }}, 
%                         mesh/ordering = y varies, 
%                         mesh/rows = 21, 
%                         mesh/cols = 21
%                         ] 
%                         table [skip first n=2] {./results/Uxx.plt};   
%            \end{axis}
%       \end{tikzpicture}

\newcommand{\isolines}[5]{
    \begin{tikzpicture}[]
    \begin{axis}[ height = 0.8\textwidth, width = 0.8\textwidth, title style={at={(0.5,-0.45)}, yshift=-0.1}, title={#5}, view={0}{90}, #1, colorbar]
    \addplot3[smooth,surf,mesh/rows=#2, shader =interp,z buffer=sort,#4] table [skip first n=2] {#3}; 
    \end{axis}
    \end{tikzpicture}
}


%\newcommand{\onegraphopt}[8]{
  %  \ifthenelse{\equal{#1}{}}
  %  {\begin{tikzpicture}[scale = 0.9]
%    \begin{axis}[ title style={at={(0.5,-0.4)},anchor=south}, title={#8}, xlabel={#2}, ylabel={#3}, {#4} ]
%    \foreach \x in {#7}
%    \addplot table [skip first n=1, x index=0,y index=\x]{#5};
%    \legend{#6}
%    \end{axis}
%    \end{tikzpicture}}
%   {\begin{tikzpicture}[scale = 0.9]
%     \begin{axis}[ title style={at={(0.5,-0.4)},anchor=south}, title={#8}, xlabel={#2}, ylabel={#3}, {#4} ]
%     \foreach \x in {#7}
%     \addplot table [skip first n=1, x index=#1,y index=\x]{#5};
%     \legend{#6}
%     \end{axis}
%     \end{tikzpicture}}
%}

%
%\newcommand{\onegrapht}[5]{
%   \begin{tikzpicture}[scale = 0.9]
%   \begin{axis}[title style={at={(0.5,-0.4)},anchor=south}, title={#5}, {#1}]
%   \foreach \x in {#4}
%   \addplot table [skip first n=1, x index=0,y index=\x]{#2};
%   \legend{#3}
%   \end{axis}
%   \end{tikzpicture}
%}



\newcommand{\onegraphw}[3]
{
	\begin{figure}[htpb]%{1.3\textwidth}
		%\hspace{-0.075\textwidth}
		%\hfill
		\centering
		\begin{minipage}{\textwidth}
			\centering
			{#1}
		\end{minipage}
		\vspace{-1mm}
		\caption{#2} \label{#3} 
	\end{figure}    
}



\newcommand{\twographs}[4]{

    \begin{figure}[htpb]
    %\hspace{-0.075\textwidth}
    %\hfill
        \begin{minipage}[t]{0.5\textwidth}
            {#1}
        \end{minipage}
        %\hspace{0.075\textwidth}
        \begin{minipage}[t]{0.5\textwidth}
            {#2}        
        \end{minipage} %\vspace{-1mm}
        \caption{#3} \label{#4} 
    \end{figure}    
    
}

\newcommand{\fourgraphs}[6]{
    \begin{figure}[htpb]
    %\hspace{-0.08\textwidth}
        \begin{minipage}[t]{0.5\textwidth}
            {#1}
        \end{minipage}
        %\hspace{0.075\textwidth}
        \begin{minipage}[t]{0.5\textwidth}
            {#2}        
        \end{minipage}
       
    %\hspace{-0.08\textwidth}
        \begin{minipage}[t]{0.5\textwidth}
            {#3}
        \end{minipage}
        %\hspace{0.075\textwidth}
        \begin{minipage}[t]{0.5\textwidth}
            {#4}        
        \end{minipage}
        \caption{#5} \label{#6} 
    \end{figure}    
    
}

