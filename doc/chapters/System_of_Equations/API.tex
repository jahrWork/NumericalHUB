\chapter{Systems of equations}
\section{Overview}

This is a library designed to solve systems of equations. The module \verb|Linear_systems| has functions and subroutines related to linear algebra. 

 \renewcommand{\home}{../libraries/Numerical_Methods/System_of_Equations}
  \listings{\home/sources/Linear_systems.f90}
       {module Linear_systems}
       {contains}{Linear_systems.f90}
\vspace{-0.3cm}       


\newpage


\section{Linear systems module}


\subsection*{LU factorization}


\begin{lstlisting}[frame=trBL]
call LU_factorization( A )
\end{lstlisting}
The subroutine \verb|LU_factorization| finds the LU factorization 
of the input matrix $ A$. The results is given in the same argument. 
The arguments of the subroutine are described in the following table.

\btable
A & two-dimensional array of reals & inout & Square matrix.\\ \hline
\etable{Description of \texttt{LU\_factorization} arguments}

%***********************************************************************************
\subsection*{Solve LU}
\begin{lstlisting}[frame=trBL]
x = Solve_LU( A , b )
\end{lstlisting}
The function \verb|Solve_LU| finds the solution to the linear system of equations $ 	A \ \vect{x}=\vect{b}, $  
where the matrix ${A}$ has been previously $ L \ U $ factorized and $\vect{b}$ is a  given vector. 
The arguments of the function are described in the following table.
\btable
A & two-dimensional array of reals & inout & Square matrix ${A}$ previously factorized by \verb|LU_factorization|.\\ \hline
b & vector of reals & in & Independent term  $\vect{b}$. \\ \hline
\etable{Description of \texttt{Solve\_LU} arguments}

%***********************************************************************************
\subsection*{Inverse}
\begin{lstlisting}[frame=trBL]
B = Inverse( A )
\end{lstlisting}
The function \verb|Inverse| finds the inverse of the square matrix $ A$ 
by means of a  $ L \ U $ factorization. 


%***********************************************************************************
\subsection*{Gauss}
\begin{lstlisting}[frame=trBL]
x = Gauss( A , b )
\end{lstlisting}
The function \verb|Gauss| finds the solution of the linear system of equations $ 	A \ \vect{x}=\vect{b} $  
by means of a classical Gaussian elimination.
The arguments of the function are described in the following table.
\btable
A & two-dimensional array of reals & inout & Square matrix ${A}$.\\ \hline
b & vector of reals & in & Independent term $\vect{b}$. \\ \hline
\etable{Description of \texttt{Gauss} arguments}

%***********************************************************************************
\subsection*{Condition number}
\begin{lstlisting}[frame=trBL]
kappa = Condition_number(A) 
\end{lstlisting}
The function \verb|Condition_number| determines the condition number   $ \kappa = ||A||_2 \ || A^{-1} ||_2  $  
where $ A $ is a square matrix. 

%***********************************************************************************
\subsection*{Tensor product}
\begin{lstlisting}[frame=trBL]
A = Tensor_product(u, v) 
\end{lstlisting}
The function \verb|Tensor_product| determines the matrix   $ A_{ij} = u_i \ v_j   $.  
The arguments of the function are described in the following table.
\btable
u & vector of reals & in & Vector $\vect{u}$.\\ \hline
v & vector of reals & in & Vector $\vect{v}$. \\ \hline
\etable{Description of \texttt{Tensor\_product} arguments}


\newpage 
%***********************************************************************************
\subsection*{Power method}
\begin{lstlisting}[frame=trBL]
call Power_method(A, lambda, U) 
\end{lstlisting}
The function \verb|Power_method| finds the largest eigenvalue of  $ 	A  $  
by the power method. 
The arguments of the function are described in the following table.
\btable
A &  array of reals & inout & Square matrix ${A}$.\\ \hline
lambda & real & out & Largest eigenvalue. \\ \hline
U & vector of reals & out & Associated eigenvector. \\ \hline
\etable{Description of \texttt{Power\_method} arguments}


%***********************************************************************************
\subsection*{Inverse Power method}
\begin{lstlisting}[frame=trBL]
call Inverse_Power_method(A, lambda, U) 
\end{lstlisting}
The function \verb|Power_method| finds the smallest eigenvalue of  $ 	A  $  
by the inverse power method. 
The arguments of the function are described in the following table.
\btable
A & array of reals & inout & Square matrix ${A}$.\\ \hline
lambda & real & out & Smallest eigenvalue. \\ \hline
U & vector of reals & out & Associated eigenvector. \\ \hline
\etable{Description of \texttt{Inverse\_power\_method} arguments}



\newpage 
%***********************************************************************************
\subsection*{Eigenvalues by means of the power method}
\begin{lstlisting}[frame=trBL]
call Eigenvalues_PM(A, lambda, U, lambda_min)  
\end{lstlisting}
The function \verb|Eigenvalues_PM| calculates 
the eigenvalues of a symmetric matrix $ A$. 
Since the power method obtains the greatest eigenvalue, 
eigenvalues are calculated sequentially until the minimum value 
\verb|lambda_min| is reached. 
The arguments of this subroutine are described in the following table.
\btable
A &  array of reals & in & symmetric matrix ${A}$.\\ \hline
lambda & vector of reals & out & eigenvalues of $ A $.  \\ \hline
U & array of reals & out & $U_{ik}$ is the associated  eigenvector of $ A $.   \\ \hline
lambda\_min &  real, optional &  in &  min value of calculated eigenvalues.  \\ \hline
\etable{Description of \texttt{Eigenvalues\_PM} arguments}





%***********************************************************************************
\subsection*{SVD}
\begin{lstlisting}[frame=trBL]
call SVD(A, sigma, U, V, sigma_min)  
\end{lstlisting}
The subroutine  \verb|SVD| finds the decomposition   $ A  = U \ S \ V^{T} $
of a non-square matrix $A$.  
The singular values \verb|sigma| 
are calculated sequentially until the minimum value 
\verb|sigma_min| is reached. 
If the rank of matrix $ A $ is $ r < \min(N,M) $,
eigenvalues are completed by means of Gram--Schmidt orthonormalization.
The arguments of the function are described in the following table.
\btable
A & two-dimensional array of reals & in & Square matrix ${A}$.\\ \hline
sigma & vector of reals & out & $\sigma^2_k$ eigenvalues of $ A^T  A $  \\ \hline
U & two-dimensional array of reals & out & $U_{ik}$ is the associated  eigenvector of $ A \ A^T $   \\ \hline
V & two-dimensional array of reals & out & $V_{ik}$ is the associated  eigenvector of $ A^T A $   \\ \hline
\etable{Description of \texttt{SVD} arguments}








\newpage 
\section{Non Linear Systems module}


 The module \verb|Non_Linear_Systems| is used to solve non linear system of equations. 


 \renewcommand{\home}{../libraries/Numerical_Methods/System_of_equations}
  \listings{\home/sources/Non_Linear_Systems.f90}
       {module Non_Linear_Systems}
       {contains}{Non_Linear_Systems.f90}
\vspace{-0.3cm}       





\subsection*{Newton}
\begin{lstlisting}[frame=trBL]
call Newton( F , x0 )
\end{lstlisting}

The subroutine \verb|Newton| returns the solution of a non-linear system of equations. The arguments of the subroutine are described in the following table.

\btable	
			F & vector function $ F : \mathbb{R}^{N} \rightarrow \mathbb{R}^{N}$ & in & System of equations to be solved. \\ \hline
			x0 & vector of reals & inout & Initial iteration point. When the iteration reaches convergence,  this vector contains the solution of the problem.  \\ \hline
			
\etable{Description of \texttt{Newton} arguments}






\newpage 
\subsection*{Newtonc}
\begin{lstlisting}[frame=trBL]
call Newtonc( F , x0 )
\end{lstlisting}

The subroutine \verb|Newtonc| returns the solution of implicit and explicit equations packed in the same function $ F(x)$.
Hence, the function $F(x)$  has internally the following form: 
\begin{eqnarray*}
     &  x_1  = & g_1( x_2, x_3, \ldots x_N), \\ 
     & x_2   =& g_2( x_1, x_3, \ldots x_N), \\ 
     & \vdots &  \vdots \\
     & x_m   =& g_m( x_1, x_2, \ldots x_N), \\ 
     &        & \\
     & F_1   =& 0, \\
     & F_2  =& 0, \\
     & \vdots & \vdots \\
     & F_m  =& 0, \\
     & & \\
     & F_{m+1}   =& g_{m+1}( x_1, x_2, \ldots x_N), \\
      & \vdots & \vdots \\
      & F_{N}   =& g_N( x_1, x_2, \ldots x_N). \\
\end{eqnarray*}

The arguments of the subroutine are described in the following table.

\btable
			F & vector function & in & System of implicit and explicit equations to be solved. \\ \hline
			x0 & vector of reals & inout & Initial iteration point. When the iteration reaches convergence,  this vector contains the solution of the problem.  \\ \hline
			
\etable{Description of \texttt{Newtonc} arguments}


