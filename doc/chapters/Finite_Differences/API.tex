\chapter{Collocation methods}
\vspace{-0.5cm}
\section{Overview}

This library is intended to calculate total or partial derivatives of functions at any specific   $ x  \in  \R{1},\R{2}, \R{3}  $.
Since the function is known through different data points $(x_i, f(x_i))$, it is necessary to build an interpolant. 
$$
  I(x) = f(x_0) \ \phi_0(x)+ f(x_1) \ \phi_1(x)+  \hdots + f(x_N) \ \phi_N(x). 
$$
%Once this interpolant is built, the derivatives of the Lagrange polynomials allows to determine the derivative of the function. 
%$$
%  \frac{dI(x)}{dx} = f(x_0) \ \frac{d \ell_0}{dx}(x)+ f(x_1) \ \frac{d \ell_1}{dx}(x)+  \hdots + f(x_N) \ \frac{d \ell_N}{dx}(x). 
%$$
Hence, given a set of nodals or interpolation points, the coefficients for different derivatives are calculated by means of the subroutine 
\verb|Grid_initialization|. Later, the subroutine \verb|Derivative| calculates the derivative by multiplying the function values by this calculated coefficients. 

\vspace{0.1cm}
 %\renewcommand{\home}{../libraries/Numerical_Methods/Finite_Differences}
%  \listings{\home/sources/Finite_differences.f90}
 \renewcommand{\home}{../libraries/Numerical_Methods/Collocation_methods}
   \listings{\home/Collocation_methods.f90} 
       {module Collocation_methods}
       {Derivative}{Collocation_methods.f90}

\newpage
\section{Collocation methods module}


\subsection*{Grid Initalization}
\begin{lstlisting}[frame=trBL]
call Grid_Initialization( grid_spacing, direction, q, grid_d )
\end{lstlisting}

 
\btable	
grid\_spacing & character & in &   
Collocation nodes
 \\ \hline

direction &  character  & in & Given name of the grid. \\ \hline

q & integer, optional & in &   
 Degree of the interpolant.

\\ \hline

grid\_d & vector of reals  & inout &  Calculated mesh:\verb|grid_d(0:)|  \\ \hline
\etable{Description of \texttt{Grid\_Initalization} arguments}

The argument \verb|grid_spacing| can take the values: 
\verb|'uniform'|(equally-spaced), \verb|'nonuniform'|, \verb|'Fourier'| or 
\verb|'Chebyshev'|.
If \verb|grid_spacing| is  \verb|'nonuniform'|, 
this subroutine calculates an optimum set of points within the space domain 
defined with the first point at $ x_0 $ 
and the last point $ x_N $. 
This set of points depends on the interpolation order \verb|q|.
Later, it builds the interpolant and its derivatives at the same data points $ x_i $ and it stores their values 
for future use by the subroutine \verb|Derivative|. 
 %The arguments of the subroutine are described in the following table:
The \verb|q| argument is optional and 
if it is only used when dealing with 
finite difference methods (\verb|'uniform'| or \verb|'nonuniform'|).  
It represents the degree of the interpolating polynomial 
 and \verb|q| nust be smaller or equal to  $N-1$ nodes. 
 
 If the selected collocation method is \verb|'Fourier'| or \verb|'Chebyshev'|, 
 derivatives are calculated by transforming the nodal points from the physical plane 
 to the spectral plane. Then, derivatives are calculated in the spectral plane and, 
 finally, an inverse
 transform form the spectral plane to the physical plane allows to obtain the 
 derivatives in the nodal grid or physical plane. 
%The number of nodes ($N$) should be  greater than the polynomials order (at 
%least $ N = \text{order} +1 $).  
%If \verb|grid_spacing| is \verb|'nonuniform'|, the nodes are calculated by obtaining the extrema of the polynomial error associated to the 
%polynomial of degree $N-1$ that the unknown nodes form.
%\listings{\home/Collocation_methods.f90} 
%       {subroutine Grid_Initialization}
%       {end subroutine}{Collocation_methods.f90}

\newpage
\subsection*{Derivatives for $x \in \R{k} $ }
Since the space domain $ \Omega \subset \R{k} $ with $ k=1,2,3$, derivatives (1D,2D or 3D) 
are calculated depending on the numerical problems. 
To avoid  dealing with different names associated to different space dimensions, the subroutine 
\verb|Derivative| is overloaded with the following subroutines: 
\vspace{0.5cm}
 \renewcommand{\home}{../libraries/Numerical_Methods/Collocation_methods}
    \listings{\home/Collocation_methods.f90} 
        {interface Derivative}
       {end interface}{Collocation_methods.f90}
   

\subsection*{Derivative for 1D grids}

\begin{lstlisting}[frame=trBL]
call Derivative( direction, derivative_order, W, Wxi, j )
\end{lstlisting}


\btable	
direction & character & in &  It selects the direction which composes the grid from the ones that have already been defined.  \\ \hline

derivative\_order &  integer & in & Order of derivation.\\ \hline

W &  vector of reals & in & Given nodal values \verb|W(0:N)|.\\ \hline

Wxi & vector of reals & out &  Value of the $k$-th derivate at the same nodal values
\verb|Wxi(0:N)|.\\ \hline
j & integer, optional & in &  Index in which the derivative is calculated.\\ \hline
\etable{Description of \texttt{Derivative} arguments for 1D grids}

If the \verb|j| argument is present then, only the derivative is calculated at $ x_j $. 
Otherwise, derivatives are calculated at every grid point form $ x_0 $ to $ x_N $. 

\newpage
To explain how the \verb|Collocation_methods| module works, 
the following snippet is shown:
\renewcommand{\home}{../libraries/Numerical_Methods/Collocation_methods}
    \listings{\home/Collocation_methods.f90} 
        {subroutine Derivative1D}
       {end subroutine}{Collocation_methods.f90}

Once \verb|Grid_Initialization| has selected the collocation methods, the subroutine 
\verb|Derivative| calculates derivatives of basis functions $ \phi_k(x)$. 
These functions are sine or cosine functions in the Fourier method or 
Lagrange polynomials in the finite difference method. 



\newpage
%******************************************************************************************
\subsection*{Derivative for 2D and 3D grids}
\begin{lstlisting}[frame=trBL]
call Derivative( direction , coordinate , derivative_order , W , Wxi )
\end{lstlisting}

\btable	
direction & vector of characters & in & 
It selects the directions  which compose the grid from the ones that have already 
been defined by \verb|Grid_Initilization|. \\ \hline

coordinate & integer & in & 
Coordinate at which the derivate is calculated. It can be 1,2 for 2D grids and 1,2 or 3 for 
3D grids.  \\ \hline

derivative\_order &  integer & in & Order of derivation.\\ \hline

W &  N-dimensional array of reals & in & \verb|W(:,:)| in 2D problems and 
 \verb|W(:,:,:)| in 3D problems.\\ \hline

Wxi & N-dimensional array of reals & out & 
Values of the derivative at grid points.
 \verb|Wxi(:,:)| in 2D and  \verb|Wxi(:,:,:)| in 3D.\\ \hline
\etable{Description of \texttt{Derivative} arguments for 2D and 3D grids}



If  \verb|direction = ["x", "y"]|,  \verb|coordinate = 2| and \verb|derivative_order = 1|,  then 
 \verb|Wxi| represent the first partial derivative of $ W(x,y) $ with respect to $ y $. 



