
    
    %*************************************************************************
    \chapter{Lagrange Interpolation } \label{dev:Interpolation}
    %************************************************************************* 
    \section{Overview}
    One of the most relevant matters of the theory of approximation is the interpolation of functions. The main idea of interpolation is to approximate a function $f(x)$ in an interval $[x_0,x_f]$ by means of a set of known functions $\{g_j(x)\}$ which are intended to be simpler than $f(x)$. The set $\{g_j(x)\}$ can be polynomials, monomials or trigonometric functions.
    
    
    
    \section{Lagrange interpolation}
    
    In this chapter, the polynomic interpolation will be presented. Particularly, the interpolation using Lagrange polynomials will be considered, as they are fundamental on the theory of interpolation.
    
    \subsection{Lagrange polynomials}
    
    The Lagrange polynomials $\ell_j(x)$ of grade $N$ for a set of points $\{x_j\}$ for $j=0,1,2\ldots,N$ are defined as:
    
    \begin{equation}
    	\ell_j(x) =  \prod_{\substack{i=0 \\ i\neq j}}^{N} \dfrac{x - x_i}{x_j - x_i},
    \end{equation}
    
    which satisfy: 
    
    \begin{equation}
    	\ell_j(x_i) = \delta_{ij}, \label{eq:Lagrange_kronecker}
    \end{equation}
    where $\delta_{ij}$ is the delta Kronecker function. This property is fundamental because as we will see it permits to obtain the Lagrange interpolant very easily once the Lagrange polynomials are determined. \\
    
    Another interesting property specially for recursively determining Lagrange polynomials appears when considering   $\ell_{jk}$ as the Lagrange polynomial of grade $k$ at $x_j$ for the set of nodes $\{x_0,x_1\ldots,x_k\}$ and $0 \leq j\leq k$, that is:
    
    \begin{equation}
     \ell_{jk}(x) = \prod_{\substack{i=0 \\ i\neq j}}^{k} \dfrac{x - x_i}{x_j - x_i} = \left(\dfrac{x - x_{k-1}}{x_j - x_{k-1}}\right) \prod_{\substack{i=0 \\ i\neq j}}^{k-1} \dfrac{x - x_i}{x_j - x_i},
    \end{equation}
     which results in the property:
     
     \begin{equation}
     \ell_{jk}(x) = \ell_{jk-1}(x) \, \left(\dfrac{x - x_{k-1}}{x_j - x_{k-1}}\right),
     \end{equation}
     where $1\leq k \leq N$. This property permits to obtain the Lagrange interpolant of grade $k$ for the set of points $\{x_0,x_1\ldots,x_k\}$ if its known the interpolant of grade $k-1$ for the set of points $\{x_0,x_1\ldots,x_{k-1}\}$. Besides, the value for $k=1$ satisfies:
     
     \begin{equation}
     	\ell_{j0}(x) = 1.
     \end{equation}
     
     Hence, it can be obtained recursively the interpolant at $x_j$ for each grade as:
     
     \begin{align*}
     	& \ell_{j1}(x) = \left(\dfrac{x - x_{0}}{x_j - x_{0}}\right), \\
     	& \ell_{j2}(x) = \left(\dfrac{x - x_{0}}{x_j - x_{0}}\right) \left(\dfrac{x - x_{1}}{x_j - x_{1}}\right), \\ 
     	& \qquad \vdots \\
     	& \ell_{jk}(x) = \underbrace{\left(\dfrac{x - x_{0}}{x_j - x_{0}}\right)  \ldots 
     	\left(\dfrac{x - x_{j-1}}{x_j - x_{j-1}}\right) \left(\dfrac{x - x_{j+1}}{x_j - x_{j+1}}\right) \ldots 
     	\left(\dfrac{x - x_{k-2}}{x_j - x_{k-2}}\right)}_{\ell_{jk-1}(x)} \left(\dfrac{x - x_{k-1}}{x_j - x_{k-1}}\right) . 
     \end{align*}
    \subsection{Single variable functions}
    
    Whenever is considered a single variable scalar function $f:\mathbb{R}\rightarrow\mathbb{R}$, whose value $f(x_j)$ at the nodes $x_j$ for $j=0,1,2\ldots,N$ is known,
     the Lagrange interpolant $I(x)$ that approximates the function in the interval $[x_0,x_N]$ takes the form:
     
     \begin{equation}
     	I(x) = \sum_{j=0}^{N}   b_j\ell_j (x).
     \end{equation}
    
    This interpolant is used to approximate the function $f(x)$ within the interval $[x_0,x_N]$. For this, the constants of the linear combination $b_j$ must be determined. The interpolant must satisfy to intersect the exact function $f(x)$ on the nodal points $x_i$ for $i=0,1,2\ldots,N$, that is:
    
    \begin{equation}
    	I(x_i) = f(x_i).
    \end{equation}
    
    Taking in account the property (\ref{eq:Lagrange_kronecker}) leads to:
    
    \begin{equation}
    I(x_i) = f(x_i) = \sum_{j=0}^{N}   b_j\ell_j (x_i) =\sum_{j=0}^{N}   b_j \delta_{ij}, \quad \Rightarrow \quad f(x_j) = b_j.
    \end{equation}
    
    Hence, the interpolant for $f(x)$ on the nodal points $x_j$ for $j=0,1,2\ldots,N$ is written:
    
    \begin{equation}
    I(x) = \sum_{j=0}^{N}  f(x_j) \ell_j (x).
    \end{equation}
    
    Notice that the interpolant can be interpreted at each point $x$ as the scalar product:
    
    \begin{equation}
    I(x) = 
    \begin{bmatrix}
    f(x_0), & f(x_1) & \ldots, f(x_N)
    \end{bmatrix} \cdot
    {\begin{bmatrix}
    \ell_0 (x) & \ell_1 (x) & \ldots  \ell_N (x)
    \end{bmatrix}}^T
    \end{equation}.
    
    
    \subsection{Two variables functions}
    
    Whenever the approximated function for the set of nodes $\{(x_i,y_j)\}$ for $i=0,1\ldots , N_x$ and $j=0,1\ldots , N_y$ is $f:\mathbb{R}^2\rightarrow\mathbb{R}$, the interpolant $I(x,y)$ can be calculated as a two dimensional extension of the interpolant for the single variable function. In such case, a way in which the interpolant $I(x,y)$ can be expressed is:
    
    \begin{equation}
     {I}(x,y) = \sum_{i=0}^{N_x} \sum_{j=0}^{N_y} b_{ij} \ell_i(x) \ell_j(y) .
    \end{equation}
    
    Again, using the property of Lagrange polinomyals (\ref{eq:Lagrange_kronecker}), the coefficients $b_{ij}$ are determined as:
    
    \begin{equation}
    	b_{ij}  = f(x_i,y_j),
    \end{equation}
     
     leading to the final expression for the interpolant:
     
     \begin{equation}
     {I}(x,y) = \sum_{i=0}^{N_x} \sum_{j=0}^{N_y} f(x_i,y_j) \ell_i(x) \ell_j(y) .
     \end{equation}
     
     Notice that when the interpolant is evaluated at a particular coordinate line $x=x_m$ or alternatively at $y=y_n$, it is obtained:
     
     \begin{equation}
     	{I}(x_m,y) = \sum_{j=0}^{N_y} f(x_m,y_j)  \ell_j(y) , 
     	\qquad {I}(x,y_n) = \sum_{i=0}^{N_x}  f(x_i,y_n) \ell_i(x)  ,
     	\label{eq:Interpolant_2D}
     \end{equation}
     which permits writing the interpolant as
     
     \begin{align}
     	I(x,y) & = \sum_{i=0}^{N_x} {I}(x_i,y) \ell_i(x) \nonumber \\
     	       & = \sum_{j=0}^{N_y} {I}(x,y_j) \ell_j(y). \label{eq:Interpolant_2D_recursive}
     \end{align}
 
     The form in which the interpolant is written in (\ref{eq:Interpolant_2D_recursive}) suggests a procedure to obtain the interpolant recursively.\\
     
     Another manner to interpret the equation (\ref{eq:Interpolant_2D}) is  as a bilinear form. If the vectors $\vect{\ell}_x = \ell_i(x) \vect{e}_i$, $\vect{\ell}_y = \ell_j(y) \vect{e}_j$ and the second order tensor $\mathcal{F} = f(x_i,y_j) \vect{e}_i  \otimes \vect{e}_j$ are defined, where the index $(i,j)$ go through $[0,N_x]\times[0,N_y]$. This manner, the equation (\ref{eq:Interpolant_2D}) can be written:
     
     \begin{equation}
     	I(x,y) =  \vect{\ell}_x \, \mathcal{F} \, \vect{\ell}_y^T .
     \end{equation}
     
     
     Another perspective to interpret the interpolation is obtained by considering the process geometrically. In first place, it is calculated a single variable Lagrange interpolant
     for the function restricted at $y=s$:
     
     \begin{equation}
     \eval{f(x,y)}_{y=s} = \tilde{f}(x;s)  \simeq \tilde{I}(x;s) = \sum_{i=0}^{N_x} b_i(s) \ell_i(x),
     \end{equation}
     where $\tilde{f}(x;s)$ is the restricted function, $\tilde{I}(x;s)$ its interpolant, $b_i(s) = \tilde{f}(x_i;s)$ are the coefficients of the interpolation and $\ell_i(x)$ are Lagrange polynomials. \\
     
     The coefficients $b_i(s)$ can also be interpolated as:
     
     \begin{equation}
     b_i(s) = \sum_{j=0}^{N_y} b_{ij} \ell_j(s) .
     \end{equation}
     
     Hence, the restricted interpolant can be written:
     
     \begin{equation}
     \tilde{I}(x;s) = \eval{I(x,y)}_{y=s} =\sum_{i=0}^{N_x} \sum_{j=0}^{N_y} b_{ij} \ell_i(x) \ell_j(s) ,
     \end{equation}
     and therefore the interpolant $I(x,y)$ can be expressed:
     \begin{equation}
     {I}(x,y) = \sum_{i=0}^{N_x} \sum_{j=0}^{N_y} b_{ij} \ell_i(x) \ell_j(y) .
     \end{equation}
     
     In the same manner, the interpolated value $I(x,y)$ can be achieved restricting the value in $x=s$:
     
     \begin{equation}
     \eval{f(x,y)}_{x=s} = \tilde{f}(y;s)  \simeq \tilde{I}(y;s) = \sum_{j=0}^{N_y} b_j(s) \ell_j(y),
     \end{equation}
     
     in which the coefficients $b_j(s)$ can be interpolated aswell:
     
     \begin{equation}
     b_j(s) = \sum_{i=0}^{N_x} b_{ij} \ell_i(s) .
     \end{equation}
     
     This time, the restricted interpolant is expressed as:
     
     \begin{equation}
     \tilde{I}(y;s) = \eval{I(x,y)}_{x=s} =\sum_{i=0}^{N_x} \sum_{j=0}^{N_y} b_{ij} \ell_i(s) \ell_j(y) ,
     \end{equation}
     
     which leads to the interpolated value:
     
     \begin{equation}
     {I}(x,y) = \sum_{i=0}^{N_x} \sum_{j=0}^{N_y} b_{ij} \ell_i(x) \ell_j(y) .
     \end{equation}
     
     The interpolation procedure and its geometric interpretation can be observed on the figure \ref{fig:Geometric_interpolation}. In blue it is represented the values $b_{ij}=f(x_i,y_j)$, in red the desired value $f(x,y)\simeq I(x,y)$ and in black the restricted interpolants.
     
     \twographs{\InterpolX{(a)}}{\InterpolY{(b)}}{Geometric interpretation of the interpolation of a 2D function. (a) Geometric interpretation when restricted along $y$.  (b) Geometric interpretation when restricted along $x$}{fig:Geometric_interpolation}
     
     \newpage
     \subsection{Algorithm implementation}
     In order to compute the interpolation of different functions a Fortran source code is implemented. The library for Lagrange interpolation is composed of four functions:
     \texttt{Lagrange\_polynomials}, \texttt{Interpolated\_value}, \texttt{Integral} and \texttt{Interpolant}.
     \newpage
     The function \texttt{Lagrange\_polynomials} calculates the Lagrange polynomials and its derivatives at a point, given a spatial grid. 
     %\lstinputlisting[language=Fortran, firstline=33, lastline=64]{\home/sources/Lagrange_interpolation.f90}\newpage
     %\lstinputlisting[language=Fortran, firstline=65, lastline=79, caption=\mycap{Lagrange_interpolation.f90}]{\home/sources/Lagrange_interpolation.f90}\vspace{0.5cm}
     
     \vspace{0.5cm} 
     \listings{\home/sources/Lagrange_interpolation.f90}
     {function Lagrange_polynomials}
     {end function}{Lagrange_interpolation.f90}
     
     \newpage
     The function \texttt{Interpolated\_value} calculates the interpolated value of a certain degree, given a spatial grid, the image of the function evaluated at those points and the grade of the interpolant. 
     
     \vspace{0.5cm} 
     \listings{\home/sources/Interpolation.f90}
     {function Interpolated_value}
     {end function}{Interpolation.f90}
     
     %\lstinputlisting[language=Fortran, firstline=21, lastline=37]{\home/sources/Interpolation.f90}
     %\newpage
     %\lstinputlisting[language=Fortran, firstline=38, lastline=63, caption=\mycap{Lagrange_interpolation.f90}]{\home/sources/Interpolation.f90}\vspace{0.5cm}
     \newpage 
     The function \texttt{Integral} integrates a function over a given spatial grid, the function at the interval and the approximation order.   
     
     \vspace{0.5cm} 
     \listings{\home/sources/Interpolation.f90}
     {function Integral}
     {end function}{Interpolation.f90}
    
     %\lstinputlisting[language=Fortran, firstline=70, lastline=77]{\home/sources/Interpolation.f90}
     %\newpage
    % \lstinputlisting[language=Fortran, firstline=78, lastline=112, caption=\mycap{Lagrange_interpolation.f90}]{\home/sources/Interpolation.f90}\vspace{0.5cm}
     
    \newpage
     The function \texttt{Interpolant} gives the Lagrange interpolant, its integral and the derivatives given a grid, the values of the function over it and the order of the interpolation.\\
     
     \vspace{0.5cm} 
     \listings{\home/sources/Interpolation.f90}
     {function Interpolant}
     {end function}{Interpolation.f90}
     %\lstinputlisting[language=Fortran, firstline=136, lastline=169]{\home/sources/Interpolation.f90}
     %
     %\lstinputlisting[language=Fortran, firstline=170, lastline=180, caption=\mycap{Lagrange_interpolation.f90}]{\home/sources/Interpolation.f90}
     \vspace{0.5cm}
     
     Finally, an additional function which will be relevant in the following chapter is defined. This function contains the information of the stencil or information that a $q$ order interpolation requires.
     
     \vspace{0.5cm} 
     \listings{\home/sources/Lagrange_interpolation.f90}
     {function Stencilv}
     {end function}{Lagrange_interpolation.f90}
     
     