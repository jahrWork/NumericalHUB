\chapter{Interpolation}

\section{Overview}

This library is intended to solve an interpolation problem. 
It comprises:  Lagrangian interpolation, Chebyshev interpolation and Fourier interpolation. 
To accomplish this purpose, \verb|Interpolation| module uses three modules as it is shown in the following code: 
  
\vspace{0.1cm}
 \renewcommand{\home}{./sources/Numerical_Methods/Interpolation}
  \listings{\home/sources/Interpolation.f90}
       {module Interpolation}
       {contains}{Interpolation.f90}
\vspace{-0.3cm}       
%The  function \verb|Interpolated_value| interpolates the value of a function at a certain point taking into account values of that function 
%at other points. The function \verb|Integral| computes the integral of a function in a certain interval
%and, finally, the function \verb|Interpolant|  calculates the interpolated values at different points.



\newpage
%****************************************************
\section{Interpolation module}



%****************************************************
\subsection*{Interpolated value}

The function \verb|interpolated_value| is devoted to conduct a piecewise polynomial interpolation 
of the value of a certain function $f(x)$ in $x=x_p$. 
The data provided to carry out the interpolation is the value of that function $f(x)$ in a set of nodes.

\vspace{0.5cm}
\begin{lstlisting}[frame=trBL]
yp = Interpolated_value( x, f, xp, degree )
\end{lstlisting}




\btable
				x & vector of reals & in & Points in which the value of the function $f(x)$ is provided.\\ \hline
				
				f & vector of reals & in & Values of the function $f(x)$ in the group of points denoted by $x$. \\ \hline
				
				xp & real & in & Point in which the value of the function $f$ will be interpolated. \\ \hline
				
				degree & integer & optional, in & Degree of the polynomial used in the interpolation. If it is not presented, it takes the value 2. \\ \hline
\etable{Description of \texttt{interpolated\_value} arguments}






\newpage

%___________________________________________________________________
\subsection*{Integral}
\begin{lstlisting}[frame=trBL]
I = Integral( x, f, degree )
\end{lstlisting}


The function \verb|Integral| is devoted to conduct a piecewise polynomial integration of a certain function $f(x)$. The data provided to 
carry out the interpolation is the value of that function $f(x)$ in a group of nodes. The limits of the integral correspond to the minimum 
and maximum values of the nodes.





The arguments of the function are described in the following table.

\btable	
				x & vector of reals & in & Points in which the value of the function $f(x)$ is provided.\\ \hline
				
				f & vector of reals & in & Values of the function $f(x)$ in the group of points denoted by $x$. \\ \hline
				
				degree & integer & in (optional) & Degree of the polynomial used in the interpolation. If it is not presented, it takes the value 2. \\ \hline
				
\etable{Description of \texttt{Integral} arguments}


\newpage
\subsection*{Interpolant for $\bold{x} \in \R{k} $ }
Since the space domain $ \Omega \subset \R{k} $ with $ k=1,2,3$, 1D, 2D or 3D interpolants
are calculated depending on the numerical problems. 
To avoid  dealing with different names associated to different space dimensions, the subroutine 
\verb|Interpolant| is overloaded with the following subroutines: 
\vspace{0.5cm}
 \renewcommand{\home}{./sources/Numerical_Methods/Interpolation}
    \listings{\home/sources/Interpolation.f90} 
        {interface Interpolant}
       {end interface}{Interpolation.f90}
   



%****************************************************
\subsection*{Interpolant1D}

The function \verb|Interpolant| builds a polynomial interpolation and its derivatives 
by means of a given set of points  $(x_i, f_i) $.  
The interpolant and its derivatives are evaluated at the set of points $ x_p $. 

\vspace{0.5cm}
\begin{lstlisting}[frame=trBL]
Ip = Interpolant( x, f, xp, degree )
\end{lstlisting}

%The interface of the function is: 
%\vspace{0.5cm}
% \renewcommand{\home}{./sources/Numerical_Methods/Interpolation}
%  \listings{\home/sources/Interpolation.f90}
%       {function Interpolant}
%       {-1}{Interpolation.f90}
The result \verb|Ip| is a matrix containing the interpolant and its derivatives evaluated 
at the set of points $ x_p $. First index represents the derivative order 
and the second index represents the point in which the interpolant is evaluated. 
\btable
				x & vector of reals & in & Points in which the value of the function $f(x)$ is provided.\\ \hline
				
				f & vector of reals & in & Values of the function $f(x)$ in the set of points denoted by $x$. \\ \hline
				
				xp & vector of reals & in & Points in which the interpolant and its derivatives are evaluated.   \\ \hline
				
				degree & integer & in & Degree of the polynomial used in the interpolation. \\ \hline
\etable{Description of \texttt{Interpolant} arguments}


\newpage
%****************************************************
\subsection*{Interpolant 2D}

The function \verb|Interpolant| builds a polynomial interpolation 
by means of a given set of points  $(x_i, y_j, f_{ij}) $.  
The interpolant   are evaluated at the set of points $ (x_p, y_p) $. 

\vspace{0.5cm}
\begin{lstlisting}[frame=trBL]
Ip = Interpolant( x, y, f, xp, yp, degree )
\end{lstlisting}

%The interface of the function is: 
%\vspace{0.5cm}
% \renewcommand{\home}{./sources/Numerical_Methods/Interpolation}
%  \listings{\home/sources/Interpolation.f90}
%       {function Interpolant}
%       {-1}{Interpolation.f90}
The result is a tensor of rank 2 containing the interpolant  evaluated 
at the set of points $ (x_p, y_p) $. First and second  indexes 
represent the point $(x_p, y_p)$ in which the interpolant is evaluated. 

\btable
				x & vector of reals & in & Points in which the value of the function $f(x, y)$ is provided.\\ \hline
				
				y & vector of reals & in & Points in which the value of the function $f(x, y)$ is provided.\\ \hline
				f & vector of reals & in & Values of the function $f(x,y)$ in the set of points $(x_i, y_j)$. \\ \hline
				
				xp & vector of reals & in & Points in which the interpolant is evaluated.   \\ \hline
				yp & vector of reals & in & Points in which the interpolant is evaluated.   \\ \hline
				
				degree & integer & in & Degree of the polynomial used in the interpolation. \\ \hline
\etable{Description of \texttt{Interpolant} arguments}






\newpage
%****************************************************
\subsection*{Interpolant 3D}

The function \verb|Interpolant| builds a polynomial interpolation 
by means of a given set of points  $(x_i, y_j, z_k, f_{ijk}) $.  
The interpolant are evaluated at the set of points $ (x_p, y_p, z_p) $. 

\vspace{0.5cm}
\begin{lstlisting}[frame=trBL]
yp = Interpolant( x, y, z, f, xp, yp, zp, degree )
\end{lstlisting}

%The interface of the function is: 
%\vspace{0.5cm}
% \renewcommand{\home}{./sources/Numerical_Methods/Interpolation}
%  \listings{\home/sources/Interpolation.f90}
%       {function Interpolant}
%       {-1}{Interpolation.f90}
The result is a matrix containing the interpolant evaluated 
at the set of points $ (x_p, y_p, z_p) $. First index and second indexes 
represent the point in which the interpolant is evaluated. 


\btable
	x & vector of reals & in & Interpolation  x--coordinate\\ \hline
	y & vector of reals & in &Interpolation  y--coordinate\\ \hline
	z & vector of reals & in &Interpolation  z--coordinate\\ \hline				
	f & vector of reals & in & Values  $f(x_i,y_j,z_k)$. \\ \hline
	xp & vector of reals & in & Points in which the interpolant is evaluated.   \\ \hline
	yp & vector of reals & in & Points in which the interpolant is evaluated.   \\ \hline
	zp & vector of reals & in & Points in which the interpolant is evaluated.   \\ \hline
	degree & integer & in & Degree of the polynomial used in the interpolation. \\ \hline
\etable{Description of \texttt{Interpolant} arguments}






\newpage
%****************************************************
\section{Lagrange interpolation module}

The Lagrange  interpolation module is devoted to determine Lagrange interpolants as well as errors associated to the interpolation. 
To accomplish this purpose, \verb|Lagrange_interpolation| module comprises the two following functions: 
  
\vspace{0.5cm}
 \renewcommand{\home}{./sources/Numerical_Methods/Interpolation}
  \listings{\home/sources/Lagrange_interpolation.f90}
       {module Lagrange_interpolation}
       {contains}{Lagrange_interpolation.f90}
      


%****************************************************
\subsection*{Lagrange polynomials}
The  function \verb|Lagrange_interpolation| determines the value of the different Lagrange polynomials at some point \verb|xp|.
Given a set of nodal or interpolation points \verb|x|, the following sentence determines the Lagrange polynomials: 
\vspace{0.5cm}
\begin{lstlisting}[frame=trBL]
yp =  Lagrange_polynomials( x, xp ) 
\end{lstlisting}
The interface of the function is: 
\vspace{0.5cm}
 \renewcommand{\home}{./sources/Numerical_Methods/Interpolation}
  \listings{\home/sources/Lagrange_interpolation.f90}
       {pure function Lagrange_polynomials}
       {-1}{Lagrange_interpolation.f90}
The result is a matrix containing all Lagrange polynomials
$$
    \ell_0(x),  \ \ell_1(x), \hdots \ell_N(x)  
$$
and their derivatives $\ell^{(i)}_j(x)$ (first index of the array) calculated at the scalar point \verb|xp|.       
The integral of the Lagrange polynomials is taken into account by the first index of the array with value equal to -1. The index 0 means the value of the 
Lagrange polynomials and an index $k$  greater than 0 represents the  "$k$-th" derivative of the Lagrange polynomial.   


%****************************************************
\subsection*{Lebesgue functions}
The function \verb|Lebesgue_functions| computes the Lebesgue function and its derivatives at different points  \verb|xp|. 
Given a set of nodal or interpolation points \verb|x|, 
the following sentence determines the Lebesgue function: 
\vspace{0.5cm}
\begin{lstlisting}[frame=trBL]
yp =  Lebesgue_functions( x, xp ) 
\end{lstlisting}
The interface of the function is: 
\vspace{0.5cm}
 \renewcommand{\home}{./sources/Numerical_Methods/Interpolation}
  \listings{\home/sources/Lagrange_interpolation.f90}
       {pure function Lebesgue_functions}
       {-1}{Lagrange_interpolation.f90}
The result is a matrix containing the Lebesgue function: 
$$
    \lambda(x) = |\ell_0(x)|  + |\ell_1(x)| + \hdots+ |\ell_N(x)|  
$$
and their derivatives $\lambda^{(i)}(xp_j)$ (first index of the array) calculated at different points $xp_j$.       
The integral of the Lebesgue function is represented  by the first index with value equal to -1. The index 0 means the value of the 
Lebesgue function and an index $k$  greater than 0 represents the  "$k$-th" derivative of the Lebesgue function. The second index of the array 
takes into account different components of  \verb|xp|. 



\subsection{Error polynomial} 
Given a set of of nodal or interpolation points \verb|x|, 
the function \verb|PI_Error_polynomial| computes the truncation error polynomial 
$$
    \pi_{N+1}(x) = |x-x_0| \ |x-x_1| \ \hdots \ |x-x_N|  
$$
and its derivatives at different points  \verb|xp|. 
The following sentence determines the error polynomial and its derivatives: 
\vspace{0.5cm}
\begin{lstlisting}[frame=trBL]
yp =  PI_error_polynomial( x, xp )  
\end{lstlisting}
\newpage
The interface of the function is: 
\vspace{0.5cm}
 \renewcommand{\home}{./sources/Numerical_Methods/Interpolation}
  \listings{\home/sources/Lagrange_interpolation.f90}
       {function PI_error_polynomial}
       {-1}{Lagrange_interpolation.f90}
The result is a matrix containing the error polynomial
and its derivatives of $\pi_{N+1}$ 
calculated at different points \verb|xp|.  
The first index of the array is used to hold the value of the derivatives if the error polynomial 
(0 for the value and values greatr than zero for high order derivatives).
The second index takes into account different components of  \verb|xp|.  

 










