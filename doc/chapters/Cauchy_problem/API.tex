\chapter{Cauchy Problem}

\vspace{-0.9cm}
\section{Overview}

The module \verb|Cauchy_Problem| is designed to solve the following problem:
\begin{equation*}
\frac{\text{d}\vect{U}}{\text{d}t}=\vect{F}\ (\vect{U},\ t), \quad  \vect{U}(0) 
=\vect{U}^0,  \ 
\quad \vect{F}:\mathbb{R}^{Nv} \times\mathbb{R} \rightarrow \mathbb{R}^{Nv}
\end{equation*}
 \renewcommand{\home}{./sources/Numerical_Methods/Cauchy_Problem}
  \listings{\home/sources/Cauchy_Problem.f90}
       {module Cauchy_Problem}
       {contains}{Cauchy_Problem.f90}
\vspace{-0.3cm}       
The subroutine \verb|Cauchy_ProblemS| is called to calculate the solution $ \vect{U}(t) $. If no numerical method is defined, the system is integrated by means of a fourth order Runge Kutta method. To define the error tolerance, the subroutine \verb|set_tolerance| is used. To specify the discrete temporal method,  the subroutine \verb|set_solver| is called.  


%__________________________________________________
\section{Cauchy problem module}
%__________________________________________________

\subsection*{Cauchy ProblemS}
\begin{lstlisting}[frame=trBL]
call Cauchy_ProblemS(Time_Domain, Differential_operator, Solution, Scheme) 
\end{lstlisting}   

The subroutine \verb|Cauchy_ProblemS| calculates the solution to a Cauchy problem. Previously to using it, the initial conditions must be imposed. The arguments of the subroutine are described in the following table.

\btable         
                Time\_Domain(0:N) & vector of reals & in &  Time domain partition where the solution is calculated.  \\ \hline
                
                Differential\_operator &  vector function: $\vect{F}: 
                \mathbb{R}^{N} \times\mathbb{R} \rightarrow \mathbb{R}^{N}$  & 
                in & It is the function $\vect{F}\ (\vect{U},\ t) $ described 
                in the overview.  \\ \hline
                
                Solution(0:N, 1:Nv) & matrix of reals.    & out &  
                                The first index represents the time and the second index contains the components of the solution.  \\ \hline
                                
                Scheme & temporal scheme  &  optional in& Defines the scheme used to solve the problem. 
                If it is not present, the subroutine \verb|set_solver| allows to define the family and the member of the family. If the family is not defined,  
                it uses a Runge Kutta of four stages.    \\ \hline
                
\etable{Description of \texttt{Cauchy\_ProblemS} arguments}



\newpage
\subsection*{Set solver}
\begin{lstlisting}[frame=trBL]
call set_solver( family_name, scheme_name) 
\end{lstlisting}   

The subroutine \verb|set_solver| allows to select the family of the numerical method and some specific member of the family to integrate the Cauchy problem. 
\btable  
                familiy\_name & character array  & in & family name of the numerical scheme to integrate the evolution problem. \\ \hline
                scheme\_name & character array & in &  name of a specific member of the family.   \\ \hline
\etable{Description of \texttt{set\_tolerance} arguments}

%\newpage
The following list describes new software implementations of the different families and members:  
 \begin{enumerate} 
    \item   Embbeded Runge Kutta familty (\verb|"eRK"|). Specific scheme names:  
         \begin{enumerate} 
         \setlength\itemsep{0cm}
          \item \verb| "HeunEuler21".  |
          \item \verb| "RK21".  |
          \item \verb| "BogackiShampine".  |
          \item \verb| "DOPRI54".      |
          \item \verb|  "Fehlberg54".  |
          \item \verb|  "Cash_Karp".   |
          \item \verb|  "Fehlberg87".  |
          \item \verb|  "Verner65".    |
          \item \verb|  "RK65".        |
          \item \verb|  "RK87".        |
        \end{enumerate}  
    \item  Gragg,  Burlish and Stoer method (\verb|"GBS"|). Specific scheme names: 
               \begin{enumerate} 
                  \item  \verb|  "GBS". |  
               \end{enumerate}     
    \item  Adams, Bashforth, Moulton methods (\verb|"ABM"|) implemented as multivalue methods. Specific scheme names: 
                \begin{enumerate} 
                   \item  \verb|  "PC_ABM". |
                 \end{enumerate} 
  \end{enumerate} 


\newpage
The following list describes wrappers for classical codes for the different families:  
 \begin{enumerate} 
     \setlength\itemsep{0cm}
    \item  Wrappers of classical embbeded Runge Kutta (\verb|"weRK"|). Specific scheme names:  
          \begin{enumerate} 
              \item  \verb|  "WDOP853".| 
              \item  \verb|  "WDOPRI5".|
            \end{enumerate}  
    \item  Wrappers of classical Gragg Burlish and (\verb|"wGBS"|). Specific scheme names: 
               \begin{enumerate} 
                  \item  \verb|  "WODEX". |   
               \end{enumerate}        
    \item  Wrappers of classical Adams, Bashforth Methods (\verb|"wABM"|). Specific scheme names: 
           \begin{enumerate} 
              \item  \verb|  "WODE113". |
            \end{enumerate}  
  \end{enumerate} 

\subsection*{Set tolerance}
\begin{lstlisting}[frame=trBL]
call set_tolerance(Tolerance) 
\end{lstlisting}   

The subroutine \verb|set_tolerance| allows to fix the relative and absolute error tolerance of the solution. Embedded Runge-Kutta methods, Adams Bashforth or GBS methods are able to modify locally their time step to attain the required error tolerance.  

%\btable      
%                tolerance & real & in &  Relative or absolute error tolerance of the solution.  \\ \hline
%\etable{Description of \texttt{set\_tolerance} arguments \vspace{-0.5cm} }


\subsection*{Set levels of the GBS scheme}
\begin{lstlisting}[frame=trBL]
call set_GBS_levels(NL) 
\end{lstlisting}   

The subroutine \verb|set_GBS_levels| allows to fix the number \verb|NL| of refinement 
levels of the GBS methods. If this number of levels is not fixed, the GBS 
determines automatically the number of levels required to achieve a specified 
error tolerance. 

%\btable      
%                NL & integer & in &  Number of levels of the GBS method.  \\ 
%                \hline
%\etable{Description of \texttt{set\_GBS\_levels} \vspace{-0.5cm} }



\subsection*{Get effort}
\begin{lstlisting}[frame=trBL]
 get_effort() 
\end{lstlisting}   

The function \verb|get_effort| determines the number of evaluations of the vector function associated to the Cauchy problem that are done by 
the numerical scheme to accomplish the required tolerance. 

\newpage
%_______________________________________
\section{Temporal schemes}
%_______________________________________


The module \verb|Temporal_schemes| comprises easy examples of temporal schemes and allows checking new methods developed by the user.  
 \renewcommand{\home}{./sources/Numerical_Methods/Cauchy_Problem}
  \listings{\home/sources/Temporal_schemes.f90}
       {module Temporal_Schemes}
       {Predictor_Corrector1}{Temporal_schemes.f90}

The \verb|Cauchy_problem| module uses schemes with the following interface: 
 \listings{\home/sources/Temporal_scheme_interface.f90}
       {module Temporal_scheme_interface}
       {end module}{Temporal_scheme_interface.f90}


%______________________________________________
\section{Stability}
%______________________________________________
The module \verb|Stability_regions| allows to calculate the region of absolute stability of any numerical method. This region is defined by the following expression: 
\begin{equation*}
{\cal R } = \{ \omega \in\mathbb{C}, \pi(\rho, \omega) =0, |\rho | < 1  \}
\end{equation*}
where $ \pi(\rho, \omega) $ is the characteristic polynomial of stability 
of the numerical scheme. 
 \renewcommand{\home}{./sources/Numerical_Methods/Cauchy_Problem}
  \listings{\home/sources/tools/Stability_regions.f90}
       {module Stability_regions}
       {contains}{Stability_regions.f90}

%\subsection*{Absolute Stability Region}

\begin{lstlisting}[frame=trBL]
call Absolute_Stability_Region(Scheme, x, y, Region)
\end{lstlisting}


\btable     
                Scheme & temporal scheme  & in  & Selects the scheme whose stability region is computed.    \\ \hline
                
                x & vector of reals & in &  Real domain $\Re z$ of the complex plane.  \\ \hline
                
                y & vector of reals & in &  Imaginary domain $\Im z$ of the complex plane.  \\ \hline
                
                Region & matrix of reals & in &  Maximum value of the roots of the characteristic polynomial for each point of the complex domain.  \\ \hline
\etable{Description of \texttt{Cauchy\_ProblemS} arguments}

\newpage 
The module \verb|Stability| allows to calculate the eigenvalues $ \|lambda_k $  of the Jacobian 
of $ F(u,t) $ particularized at $ u_0, t_0$ to determine 
the time step to have a stable solution by imposing 
$ \lambda_k \Delta t \in {\cal R } $.
\vspace{0.5cm} 
\renewcommand{\home}{./sources/Numerical_Methods/Cauchy_Problem}
  \listings{\home/sources/tools/Stability.f90}
       {module Stability}
       {contains}{Stability.f90}
       
To calculate the Jacobian of $ F(U,t) $
$$  
     J_{ij} = \frac{ \partial F_i }{ \partial U_j } (U_0, t_0 )
$$
evaluated at $ u_0, t_0 $ the following sentence is used:  
\vspace{0.5cm}  
\begin{lstlisting}[frame=trBL]
  A = System_matrix( F, U0, t0) 
\end{lstlisting}  
To calculate the eigenvalues of the above Jacobian,        
the following sentence is used:  
\vspace{0.5cm} 
\begin{lstlisting}[frame=trBL]
  lambda = Eigenvalues_Jacobian( F, U0, t0)
\end{lstlisting}         
where \verb|lambda| is a vector of complex numbers that holds the eigenvalues of the 
Jacobian of $F(U,t)$. 
       
%
%%\subsection*{Absolute Stability Region}
%
%\begin{lstlisting}[frame=trBL]
%call Absolute_Stability_Region(Scheme, x, y, Region)
%\end{lstlisting}
%
%
%\btable     
%                Scheme & temporal scheme  & in  & Selects the scheme whose stability region is computed.    \\ \hline
%                
%                x & vector of reals & in &  Real domain $\Re z$ of the complex plane.  \\ \hline
%                
%                y & vector of reals & in &  Imaginary domain $\Im z$ of the complex plane.  \\ \hline
%                
%                Region & matrix of reals & in &  Maximum value of the roots of the characteristic polynomial for each point of the complex 
%domain.  \\ \hline
%\etable{Description of \texttt{Cauchy\_ProblemS} arguments}
%




\newpage
%______________________________________________
\section{Temporal error}
%______________________________________________


The module \verb|Temporal_error| allows to determine, based on Richardson extrapolation, the error of a numerical solution.   
 \renewcommand{\home}{./sources/Numerical_Methods/Cauchy_Problem}
  \listings{\home/sources/tools/Temporal_error.f90}
       {module Temporal_error}
       {contains}{Temporal_error.f90}

The module uses the  \verb|Cauchy_Problem| module and comprises three subroutines to analyze the error of the temporal schemes.
It is an application layer based on the \verb|Cauchy_Problem| layer. The error is calculated by integrating the same solution in successive time grids. By using the Richardson extrapolation method, the error is determined. 
\newpage 



\subsection*{Error of the solution}
\begin{lstlisting}[frame=trBL]
call Error_Cauchy_Problem( Time_Domain, Differential_operator, Scheme, & 
                           order, Solution, Error ) 
\end{lstlisting}


\btable 
                Time\_Domain(0:N) & vector of reals & in &  Time domain partition where the solution is calculated.  \\ \hline
                                
                Differential\_operator &  vector function  $ \ \vect{F}: 
                \mathbb{R}^{N} \times\mathbb{R} \rightarrow \mathbb{R}^{N}$  & 
                in & It is the function $\vect{F}\ (\vect{U},\ t) $ described 
                in the overview.  \\ \hline
                                
                Scheme & temporal scheme  & optional in & Defines the scheme used to solve the problem.   \\ \hline
                
                order & integer & in & order of the numerical scheme. \\ \hline  
                                
                 Solution(0:N, 1:Nv) & matrix of reals.    & out &  
                 The first index represents the time and the second index contains the components of the solution.  \\ \hline
                 
                 Error(0:N, 1:Nv) & matrix of reals.    & out &  
                 The first index represents the time and the second index contains the components of the solution.  \\ \hline
\etable{Description of \texttt{Error\_Cauchy\_Problem} arguments}

\newpage 
\subsection*{Convergence rate with time steps}
\begin{lstlisting}[frame=trBL]
call Temporal_convergence_rate( Time_Domain, Differential_operator,   & 
                                U0, Scheme, order, log_E, log_N) 
\end{lstlisting}


\btable      
                Time\_Domain(0:N) & vector of reals & in &  Time domain partition where the solution is calculated.  \\ \hline
                                
                Differential\_operator &  vector function $ \ \vect{F} : 
                \mathbb{R}^{N} \times\mathbb{R} \rightarrow \mathbb{R}^{N}$  & 
                in & 
                It is the function $\vect{F}\ (\vect{U},\ t) $ described in the 
                overview.  \\ \hline
                
                U0      & vector of reals    & in &   Components of the initial conditions.  \\ \hline     
                
                          
                Scheme & temporal scheme   & in (optional) & Defines the scheme used to solve the problem.   \\ \hline
                
                order  & integer           & in            & order of the numerical scheme. \\ \hline    
                              
                log\_E & vector of reals   & out           &  
                Log of the norm2 of the error solution. Each component represents a different time grid.  \\ \hline
                
                log\_N & vector of reals   & out           &  
                Log of number of time steps to integrate the solution. Sequence of time grids N, 2N, 4N...  
                Each component represents a different time grid.  \\ \hline
\etable{Description of \texttt{Temporal\_convergence\_rate} } 


\newpage 
\subsection*{Error behavior with tolerance}
\begin{lstlisting}[frame=trBL]
call Temporal_steps_with_tolerance(Time_Domain, Differential_operator, & 
                                   U0, log_mu, log_steps)
\end{lstlisting}


\btable      
                Time\_Domain(0:N) & vector of reals & in &  Time domain partition where the solution is calculated.  \\ \hline
                                
                Differential\_operator &  vector function $\ \vect{F} : 
                \mathbb{R}^{N} \times\mathbb{R} \rightarrow \mathbb{R}^{N}$  & 
                in & 
                It is the function $\vect{F}\ (\vect{U},\ t) $ described in the 
                overview.  \\ \hline
                                    
                U0      & vector of reals    & in &   Initial conditions.  \\ \hline 
                    
                log\_mu & vector of reals   & in           &  
                                Log of the 1/tolerances. This vector is given and it allows to integrate internally different simulations with different error tolerances.    \\ \hline           
                                               
                log\_steps & vector of reals   & out           &  
                Log of the number of time steps to accomplish a simulation with a given error tolerance. 
                The numerical scheme has to be selected previously with \verb|set_solver|. \\ \hline
\etable{Description of \texttt{Temporal\_error\_with\_tolerance} } 







