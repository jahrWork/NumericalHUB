\chapter{Boundary Value Problems}
\vspace{-0.5cm}
\section{Overview}

This library is intended  to solve linear and nonlinear boundary value problems. 
An equation involving partial derivatives together with some constraints applied to the frontier of its spatial domain constitute a boundary 
value problem. 
 
\vspace{0.1cm}
 \renewcommand{\home}{./sources/Numerical_Methods/Boundary_value_problem}
  \listings{\home/sources/Boundary_value_problems.f90}
       {module Boundary_value_problems}
       {end module}{Boundary_value_problems.f90}
\vspace{-0.1cm}       
Since the space domain $ \Omega \subset \R{k} $ with $ k=1,2,3$, boundary value problems are stated in 1D, 2D and 3D grids. 
To have the same name interface when dealing with different space dimensions, the subroutine 
\verb|Boundary_value_problem| has been overloaded. 



\newpage
\section{Boundary value problems  module}
%
%\subsection*{1D Boundary Value Problems}
%\begin{lstlisting}[frame=trBL]
%call Boundary_Value_Problem( x, Differential_operator,  & 
%                             Boundary_conditions , Solution )  
%\end{lstlisting}   
%
%The subroutine  calculates the solution of the following boundary value problem:
%
%\begin{equation*}
%\mathcal{L}\left(x,\ u, \ \frac{\partial u}{\partial x}, \ \frac{\partial^2 u}{\partial x^2} \right) = 0, 
%\qquad  h \left(x,\ u, \ \frac{\partial u}{\partial x}  \right) = 0
%\end{equation*}
%
%
%
%The arguments of the subroutine are described in the following table.
%
%\btable 		
%				x & vector of reals & in &  Mesh nodes.  \\ \hline
%				
%				
%				Differential\_operator & \raggedright scalar function: $\mathcal{L}$ & in  &
% $\mathcal{L}:\R{} \times \R{} \times \R{}  \times \R{} \rightarrow \R{} $    
% \\ \hline
%				
%				Boundary\_conditions & \raggedright scalar function: $h$  & in &  
% $h:\R{} \times \R{} \times \R{}  \rightarrow \R{} $  				
%  \\ \hline
%				
%				Solution & vector of reals  & out &  \verb|Solution(:)|  
%			 \\ \hline
%\etable{Description of \texttt{Boundary\_Value\_Problem}
% arguments for 1D problems}


%\newpage
\subsection*{1D Boundary Value Problems for system of equations} 


\begin{lstlisting}[frame=trBL]
call Boundary_Value_Problem(x,  Differential_operator,  &
                            Boundary_conditions, Solution) 
\end{lstlisting}
This subroutine calculates the solution of a  boundary value problem  
of N variables in a one dimensional domain $[a,b]$:

\begin{equation*}
\vect{\mathcal{L}}	\left({x},\  \vect{u}, 
\ \frac{\partial \vect{u}}{\partial x}, 
\ \frac{\partial^2 \vect{u}}{\partial x^2} 
 \right) = \vect{0}, 
 \qquad 
\vect{h}	\left({x},\  \vect{u}, 
\ \frac{\partial \vect{u}}{\partial x}
 \right) = \vect{0}. 
\end{equation*}

%\newpage
%The arguments of the subroutine are described in the following table.

\btable	
	x & vector of reals & in &  Mesh nodes   \\ \hline
			
				
					
	Differential\_operator & \raggedright  
              %  function: $\vect{\mathcal{L}}\left({x}, {y}, \vect{u}, \vect{u}_x,  \vect{u}_y,  %\vect{u}_{xx},  \vect{u}_{yy},  
              %%\vect{u}_{xy} \right)$ 
                vector function:$\vect{\mathcal{L}}$  
                & in  & 
   $\vect{\mathcal{L}}:\R{} \times \R{N} \times \R{N} \times  \R{N}  \rightarrow \R{N} $   \\ \hline
					Boundary\_conditions & \raggedright vector function:$\vect{h}$  & in & 
   $\vect{h}:\R{} \times \R{N} \times \R{N}  \rightarrow \R{N} $  
   \\ \hline
				
				Solution & two-dimensional array  of reals  & out &  
				\verb|Solution(:,:)|. First index stands for x 
				and the second index stands for different variables. 
				                                                                                     \\ \hline
\etable{Description of \texttt{Boundary\_Value\_Problem} arguments for 1D vector problems}




\newpage
%****************************************************************************
%\subsection*{2D Boundary Value Problems}
%\begin{lstlisting}[frame=trBL]
%call Boundary_Value_Problem( x_nodes, y_nodes,                  & 
%                               Differential_operator,           &  
%                               Boundary_conditions, Solution )
%\end{lstlisting}   
%
%%\vspace{-1cm} 
%This subroutine calculates the solution to a linear boundary value problem in a 
%rectangular domain $[a,b] \times [c,d]$.
%The differential operator $\mathcal{L}$  and the boundary conditions 
% $h$ are functions expressed as:
%
%%\vspace{-3cm }
%\begin{align*}
%\mathcal{L}\left(x,\ y,\ u, 
%\ \frac{\partial u}{\partial x}, 
%\ \frac{\partial u}{\partial y}, 
%\ \frac{\partial^2 u}{\partial x^2}, 
%\ \frac{\partial^2 u}{\partial y^2}, 
%\ \frac{\partial^2 u}{\partial x \partial y} \right) = 0,
%\end{align*} 
%
%\begin{align*}
%h \left(x,\ y,\ u, 
%\ \frac{\partial u}{\partial x}, 
%\ \frac{\partial u}{\partial y} 
% \right) = 0.
%\end{align*} 
%
%
%The arguments of the subroutine are described in the following table.
%
%\btable	
%				x\_nodes & vector of reals & in &  Mesh nodes in $x$ direction.  \\ \hline
%				
%				y\_nodes & vector of reals & in &  Mesh nodes in $y$ direction.  \\ \hline
%				
%						
%				Differential\_operator &  real function: $\mathcal{L} $ & 
%                in  & 
%                  $\mathcal{L}:\R{} \times \R{} \ldots \times  \R{} \rightarrow \R{} $ 
%                 \\ \hline
%				
%				Boundary\_conditions & \raggedright real function: $h$  & in &   
%	  $h:\R{} \times \R{} \times \R{} \times  \R{}  \times  \R{} \rightarrow \R{} $
%				 \\ \hline
%				
%				Solution & two-dimensional array of reals  & out &  
%				\verb|Solution(:,:)|. First  index stands for 
%				x and second index stands for y. \\ \hline
%\etable{Description of \texttt{Boundary\_Value\_Problem} arguments for 2D problems}
%
%
%\newpage
\subsection*{2D Boundary Value Problems for system of equations} 

\begin{lstlisting}[frame=trBL]
call Boundary_Value_Problem(x, y,                       &
                            Differential_operator,         &
                            Boundary_conditions, Solution) 
\end{lstlisting} 
This subroutine calculates the solution of a 2D boundary value problem  of N variables in a 
rectangular domain $[a,b] \times [c,d]$. 
The differential operator $\vect{\mathcal{L}}$  and the boundary conditions 
 $\vect{\mathcal{L}}$ are expressed as: 


%\vspace{-0.5cm}

\begin{equation*}
\vect{\mathcal{L}}	\left({x},\ {y},\ \vect{u}, \ \frac{\partial \vect{u}}{\partial x}, \ \frac{\partial \vect{u}}{\partial y}, \ 
\frac{\partial^2 \vect{u}}{\partial x^2}, \ \frac{\partial^2 \vect{u}}{\partial y^2}, \ \frac{\partial^2 \vect{u}}{\partial x \partial y} 
\right) = \vect{0},
\end{equation*}

\begin{equation*}
\vect{h}	\left({x},\ {y},\ \vect{u}, 
\ \frac{\partial \vect{u}}{\partial x}, 
\ \frac{\partial \vect{u}}{\partial y}
 \right) = \vect{0}. 
\end{equation*}


The solution of this problem is calculated using the libraries by a simple call to the subroutine:

%\newpage
%The arguments of the subroutine are described in the following table.

\btable	
				x & vector of reals & in &  Mesh nodes in  $x$  direction.  \\ \hline
				
				y & vector of reals & in &  Mesh nodes in $y$ direction.  \\ \hline
				
					
				
				Differential\_operator & \raggedright  
              %  function: $\vect{\mathcal{L}}\left({x}, {y}, \vect{u}, \vect{u}_x,  \vect{u}_y,  %\vect{u}_{xx},  \vect{u}_{yy},  \vect{u}_{xy} \right)$ 
                function: $\vect{\mathcal{L}}$  
                & in  & Vector function to implement the differential operator.   \\ \hline
				
				Boundary\_conditions & \raggedright function: $\vect{h}$  & in &  
				Vector function to implement different boundary conditions.  \\ \hline
				
				Solution & three-dimensional array  of reals  & out &  
		\verb|Solution(:,:,:).| Third index stands for different variables. 
				                                                                                     \\ \hline
\etable{Description of \texttt{Boundary\_Value\_Problem} arguments for vector 2D problems}



%\newpage
%\subsection*{3D Boundary Value Problems for systems of equations}
%
%%\vspace{-3cm}
%
%\begin{lstlisting}[frame=trBL]
%call Boundary_Value_Problem(x_nodes, y_nodes, z_no        & 
%                            Differential_operator,        & 
%                            Boundary_conditions, Solution )
%\end{lstlisting}  
%This subroutine calculates the solution of a boundary value problem system of N variables in a rectangular domain $[a,b] \times [c,d]\times 
%[e,f]$:
%
%
%
%\begin{equation*}
%\vect{\mathcal{L}}	\left({x},\ {y},\ {z},\ \vect{u}, \ \frac{\partial \vect{u}}{\partial x}, \ \frac{\partial \vect{u}}{\partial y}, \ 
%\frac{\partial \vect{u}}{\partial z}, \ \frac{\partial^2 \vect{u}}{\partial x^2}, \ \frac{\partial^2 \vect{u}}{\partial y^2}, \ 
%\frac{\partial^2 \vect{u}}{\partial z^2}, \ \frac{\partial^2 \vect{u}}{\partial x \partial y}, \ \frac{\partial^2 \vect{u}}{\partial x 
%\partial z}, \ \frac{\partial^2 \vect{u}}{\partial y \partial z} \right) = \vect{0}
%\end{equation*}
%
%
%\btable		
%				x\_nodes & vector of reals & in &  Nodes in  $x$ direction.  \\ \hline
%				
%				y\_nodes & vector of reals & in &  Nodes in $y$ direction.  \\ \hline
%				
%				z\_nodes & vector of reals & in &  Nodes in $z$ direction.  \\ \hline
%				
%			
%				
%				Differential\_operator & 
%                 function: $ \vect{\mathcal{L}} $
%				 & in  & Differential operator.   \\ \hline
%				
%				Boundary\_conditions & \raggedright function: $\vect{h}$  & in &  Boundary conditions.  \\ \hline
%				
%				Solution & 4-dimensional array of reals  & out &  
%				\verb|Solution(:,:,:,:).| Fourth index stands for different variables. \\ \hline
%\etable{Description of \texttt{Boundary\_Value\_Problem} arguments for 3D vector problems}
