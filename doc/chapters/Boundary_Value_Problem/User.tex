     
%*************************************************************************
\chapter{Boundary Value Problems   }\label{BVP}
%*************************************************************************  
\vspace{-1cm}
\section{Overview}
Let  $\Omega \subset \mathbb{ R}^p$ be an open and connected set and $\partial \Omega$ its boundary set. The spatial domain $D$ is defined as 
its closure, $D \equiv \{\Omega \cup \partial \Omega\}$. Each point of the spatial domain is written  $\vect{x} \in D $.
A Boundary Value Problem for a vectorial function $\vect{u}: D \rightarrow \mathbb{R}^{N}$ of $N$ variables is defined as:
       \begin{align*}
           &\vect{\mathcal{L}} (\vect{x},\vect{u}(\vect{x})) = 0, & \forall \ \ \vect{x} \in  \Omega, \\
           &\vect{h} (\vect{x},\vect{u}(\vect{x}))\big\rvert_{\partial \Omega}=0 ,  & \forall \ \ \vect{x} \in \partial \Omega,
       \end{align*}
where $\vect{\mathcal{L}}$ is the spatial differential operator and $\vect{h}$ is the boundary conditions operator that must satisfy the 
solution at the boundary $\partial \Omega$.
In the subroutine \verb|BVP_examples|, examples of boundary value problems are presented.
The first and second examples are 1D boundary value problems. 
The first one (\verb|Legendre_1D|) associated to one unknown function 
and the second one (\verb|Elastic_beam_1D|) associated to a system of differential equations. 
Then, 2D BVP are considered: scalar, vector, linear and nonlinear problems. 
%  second solves a Poisson problem in a 2D. The third studies the 
% deflection of an elastic plate subjected to external loads in a 2D space. 
%Finally, the fourth example analyzes the deflection on an 
% nonlinear plate subjected to external loads.
       \vspace{0.2cm} 
       \listings{\home/examples/API_example_Boundary_Value_Problem.f90}
       {subroutine BVP_examples}
       {end subroutine}{API_example_Boundary_Value_Problem.f90}
\vspace{-0.5cm}       
To  use all functions of this module, the statement: 
 \verb|use Boundary_value_problems|
 should be included at the beginning of the program.          
  
       
\newpage
%_______________________________________________________________________________________
\section{Legendre equation}
       Legendre polynomials are a system of complete and orthogonal polynomials with numerous applications.
              Legendre polynomials can be defined as the solutions of the Legendre's differential equation on a domain $\Omega \subset 
              \mathbb{R} : $ $\{x\in  [-1,1]\}$:
              \begin{equation*}      	
              (1 - x^2) \frac{\text{d}^2 y}{\text{d} x^2} - 2x \frac{\text{d} y}{\text{d} x} + n (n + 1) y = 0,
              \end{equation*}
              where $ n $ stands for the degree of the Legendre polynomial. 
              For $n = 6$,  the boundary conditions are: $y(-1) = - 1, \  y(1) = 1$ and the exact solution  is:
              \begin{equation*}
              	y(x) = \frac{1}{16}( 231 x^6 - 315 x^4 + 105 x^2 - 5).
              \end{equation*}
              This problem is solved by means of piecewise polynomial interpolation of degree q or finite differences of order q.  The 
              implementation of the problem requires the definition of the differential operator $\vect{\mathcal{L}} 
              (\vect{x},\vect{u}(\vect{x}))$:
              
              \vspace{0.5cm} 
              \listings{\home/examples/API_example_Boundary_Value_Problem.f90}
              {real function Legendre}
              {end function}
              {API_example_Boundary_Value_Problem.f90}
                    
              And the boundary conditions $\vect{h} (\vect{x},\vect{u}(\vect{x})) $  are  implemented as a function:
              \vspace{0.5cm} 
              \listings{\home/examples/API_example_Boundary_Value_Problem.f90}
              {function Legendre_BCs}
              {end function}
              {API_example_Boundary_Value_Problem.f90}
              These two functions are input arguments of the subroutine \verb|Boundary_Value_Problem|:  
              \vspace{0.5cm} 
              \listings{\home/examples/API_example_Boundary_Value_Problem.f90}
              {Legendre solution}
              {Error}
              {API_example_Boundary_Value_Problem.f90}
              In this example, the piecewise polynomial interpolation is of degree $ q=6 $ and the problem is discretized with $ N = 40 $ 
              grid nodes. 
               \vspace{0.5cm} 
                     \listings{\home/examples/API_example_Boundary_Value_Problem.f90}
                     {N = 40}
                     {N = 40}
                     {API_example_Boundary_Value_Problem.f90}
              Since the degree of the piecewise polynomial interpolation coincides with the degree of the solution or the Legendre 
              polynomial, no error is expected to obtain.  
              It can be observed in figure  \ref{fig:Legendre}b that error is of the order of the round--off value.  The solution or the 
              Legendre polynomial is shown in  \ref{fig:Legendre}a.
              \twographs
              {  \begin{tikzpicture}[]
     \begin{axis}[ width = \textwidth,
        title style={at={(0.5,-0.45)},anchor=south}, title={(a)},
        xlabel={ $x$ },
        ylabel={ $y$ },
        {}, legend style={font=\small},  {mark repeat =  2, cycle list name = black white, legend columns = 1 }, legend pos = north east  ]
        \foreach \x/\y in { 0/1}
        \addplot table [skip first n=2, x index=\x, y index=\y]{./doc/chapters/Boundary_Value_Problem/figures/Legendrea.plt   };
        \legend{ P6 }
     \end{axis}
  \end{tikzpicture}
}
              {  \begin{tikzpicture}[]
     \begin{axis}[ width = \textwidth,
        title style={at={(0.5,-0.45)},anchor=south}, title={(b)},
        xlabel={ $x$ },
        ylabel={ $E$ },
        {}, legend style={font=\small},  {mark repeat =  2, cycle list name = black white, legend columns = 1 }, legend pos = north east  ]
        \foreach \x/\y in { 0/1}
        \addplot table [skip first n=2, x index=\x, y index=\y]{./doc/chapters/Boundary_Value_Problem/figures/Legendreb.plt   };
        \legend{ E }
     \end{axis}
  \end{tikzpicture}
}
              {Solution of the Legendre equation with $N=40$ grid points. (a) Legrendre polynomial of degree   $n=6$. (b) Error of the 
              solution.}{fig:Legendre}
       
\newpage 
%____________________________________________________________________________      
\section{Beam deflection}\label{beam} 

If beams are straight and slender and only if small 
deflections are considered, the equations governing the beam deflection $ w(x) $ can be approximated as: 
 \begin{equation*}      	
      \frac{ d^2 w}{d  x^2}  = M(x), \qquad  \frac{d^2 M}{d x^2 } = q(x),
 \end{equation*}
where $ M(x) $ is the internal bending moment in the beam and $ q(x) $ is a distributed load.
This equations are completed with four  boundary conditions to yield the deflection $ w(x) $
and the moment $ M(x)$. In this section fixed supports at both ends are considered: 
\begin{equation*}      	
    w(-1) = 0, \quad\frac{dw}{d x }(-1)  = 0, \quad w(+1) = 0, \quad \frac{dw}{d x }(+1)  = 0.
\end{equation*} 
The  problem is implemented with the definition of the differential operator 
$\vect{\mathcal{L}} (\vect{x},\vect{u})$ and the boundary conditions function
 $\vect{h} (\vect{x},\vect{u}) $.
 \vspace{0.5cm} 
\listings{\home/examples/API_example_Boundary_Value_Problem.f90}
{function Beam_equations}
{end function}
{API_example_Boundary_Value_Problem.f90}
\listings{\home/examples/API_example_Boundary_Value_Problem.f90}
{function Beam_BCs}
{end function}
{API_example_Boundary_Value_Problem.f90}

%These two functions are input arguments of the subroutine \verb|Boundary_Value_Problem|:  
%\vspace{0.5cm} 
%\listings{\home/examples/API_example_Boundary_Value_Problem.f90}
%{Legendre solution}
%{Error}
%{API_example_Boundary_Value_Problem.f90}
%
%In this example, the piecewise polynomial interpolation is of degree $ q=6 $ and the problem is discretized with $ N = 40 $ 
%grid nodes. 
%     \vspace{0.5cm}
%     \listings{\home/examples/API_example_Boundary_Value_Problem.f90}
%      {N = 40}
%      {N = 40}
%      {API_example_Boundary_Value_Problem.f90}
%Sinnce the degree of the piecewise polynomial interpolation coincides with the degree of the solution or the Legendre 
%polynomial, no error is expected to obtain.  
%It can be observed in figure  \ref{fig:Legendre}b that error is of the order of the round--off value.  The solution or the 
%Legendre polynomial is shown in  \ref{fig:Legendre}a.
%\twographs
%{  \begin{tikzpicture}[]
     \begin{axis}[ width = \textwidth,
        title style={at={(0.5,-0.45)},anchor=south}, title={(a)},
        xlabel={ $x$ },
        ylabel={ $y$ },
        {}, legend style={font=\small},  {mark repeat =  2, cycle list name = black white, legend columns = 1 }, legend pos = north east  ]
        \foreach \x/\y in { 0/1}
        \addplot table [skip first n=2, x index=\x, y index=\y]{./doc/chapters/Boundary_Value_Problem/figures/Legendrea.plt   };
        \legend{ P6 }
     \end{axis}
  \end{tikzpicture}
}
%{  \begin{tikzpicture}[]
     \begin{axis}[ width = \textwidth,
        title style={at={(0.5,-0.45)},anchor=south}, title={(b)},
        xlabel={ $x$ },
        ylabel={ $E$ },
        {}, legend style={font=\small},  {mark repeat =  2, cycle list name = black white, legend columns = 1 }, legend pos = north east  ]
        \foreach \x/\y in { 0/1}
        \addplot table [skip first n=2, x index=\x, y index=\y]{./doc/chapters/Boundary_Value_Problem/figures/Legendreb.plt   };
        \legend{ E }
     \end{axis}
  \end{tikzpicture}
}
%{Solution of the Legendre equation with $N=40$ grid points. (a) Legrendre polynomial of degree   $n=6$. (b) Error of the 
%solution.}{fig:Legendre}
        
       
%___________________________________________________________________________
\newpage       
\section{Poisson equation}\label{Poisson}  
Poisson's equation is a partial differential equation of elliptic type with broad utility in mechanical engineering and theoretical physics. 
This equation arises to describe the potential field caused by a given charge or mass density distribution. 
In the case of fluid mechanics, it is used  to determine potential flows, streamlines and  pressure distributions for incompressible flows.   
It is an in-homogeneous differential equation with a source term representing the volume charge density, the mass density or the vorticity 
function in the case of a fluid. It is written in the following form:  
\begin{equation*}
             	\nabla^2 u = s(x,y), 
\end{equation*} 
where $\nabla^2 u = \partial^2 u / \partial x^2 + \partial^2 u / \partial y^2 $ and $ s(x,y) $ is the source term.
This Poisson equation is implemented by the following code: 
\vspace{0.8cm} 
\listings{\home/examples/API_example_Boundary_Value_Problem.f90}
         {real function Poisson}
         {end function}
         {API_example_Boundary_Value_Problem.f90}
         
 
It is considered two punctual sources given by the expression:
$$
      S(x, y) = a e ^{ -a r^2_1} + a e^{ -a r^2 _2 },  \quad r^2_i = (x-x_i)^ 2 + (y-y_i)^2, 
$$ 
where $ a $ is an attenuation parameter and $ (x_1, y_1) $ and  $ (x_2, y_2) $ are the positions of the sources.   
The source term is implemented in the following code:       
 \vspace{0.8cm} 
  \listings{\home/examples/API_example_Boundary_Value_Problem.f90}
           {real function source}
           {end function}
           {API_example_Boundary_Value_Problem.f90}      
In this example, homogeneous boundary conditions are considered and they implemented by the  function \verb|PBCs|:  
 \vspace{0.5cm} 
 \listings{\home/examples/API_example_Boundary_Value_Problem.f90}
          {function PBCs}
          {end function}
          {API_example_Boundary_Value_Problem.f90}
The differential operator \verb|Poisson|  with its boundary conditions \verb|PBCs| are used as input arguments for the subroutine 
\verb|Boundary_value_problem|               
\vspace{0.3cm} 
 \listings{\home/examples/API_example_Boundary_Value_Problem.f90}
          {Poisson equation}
          {Solution}
          {API_example_Boundary_Value_Problem.f90}
In figure  \ref{fig:Poisson} the solution of this Poisson equation is shown. 
\vspace{-0.1cm}
   \twographs
       {  \begin{tikzpicture}[]
     \begin{axis}[ height = \textwidth, width = \textwidth,
        title style={at={(0.5,-0.4)}, yshift=-0.05}, title={(a)},
        xlabel={ $x$ },    xmin=    0.00, xmax =    1.00,
        xlabel={ $y$ },    ymin=    0.00, ymax =    1.00,
         view={0}{90}, colormap/blackwhite ]
  \addplot3[contour/labels=false, contour gnuplot = { levels={    0.00,   5.26,  10.51,  15.77,  21.03,  26.29,  31.54,  36.80,  42.06,  47.32,  52.57,  57.83,  63.09,  68.35,  73.60,  78.86,  84.12,  89.37,  94.63,  99.89 }},
            mesh/ordering = y varies,
            mesh/rows =   31,
            mesh/cols =   31
           ]
           table [skip first n=2] {./doc/chapters/Boundary_Value_Problem/figures/Poissona.plt};
     \end{axis}
 \end{tikzpicture}
}
       {  \begin{tikzpicture}[]
     \begin{axis}[ height = \textwidth, width = \textwidth,
        title style={at={(0.5,-0.4)}, yshift=-0.05}, title={(b)},
        xlabel={ $x$ },    xmin=    0.00, xmax =    1.00,
        xlabel={ $y$ },    ymin=    0.00, ymax =    1.00,
         view={0}{90}, colormap/blackwhite ]
  \addplot3[contour/labels=false, contour gnuplot = { levels={   -0.87,  -0.82,  -0.78,  -0.73,  -0.69,  -0.64,  -0.59,  -0.55,  -0.50,  -0.46,  -0.41,  -0.37,  -0.32,  -0.27,  -0.23,  -0.18,  -0.14,  -0.09,  -0.05,   0.00 }},
            mesh/ordering = y varies,
            mesh/rows =   31,
            mesh/cols =   31
           ]
           table [skip first n=2] {./doc/chapters/Boundary_Value_Problem/figures/Poissonb.plt};
     \end{axis}
 \end{tikzpicture}
}
       {Solution of the Poisson equation with $Nx = 30$, $Ny = 30$ and piecewise interpolation of degree $ q = 11$.
       (a) Source term s(x,y).(b) Solution u(x,y). }
       {fig:Poisson}    
       
       
 \FloatBarrier      
 %_______________________________________________________________________      
 \newpage
 \section{Deflection of an elastic linear plate}\label{LinearPlate}  
 In this section,  an elastic plate submitted to a distributed load is implemented.    It is Considered a plate with simply supported edges 
 with a distributed load $p(x,y)$ in a domain $\Omega \subset \mathbb{R}^2 : $$\{ (x,y) \in [-1,1] \times [-1,1] \}$.The deflection $w(x,y)$ 
 of a the  plate is governed by the following bi-harmonic equation:
     
            \begin{equation*}
            	\nabla^4 w(x,y) = p(x,y),  %!  4 \pi ^4 \sin(\pi x)\sin(\pi y), 
            \end{equation*}      
where $\nabla ^4 = \nabla^2(\nabla^2) $ is the bi-harmonic operator. The simply supported condition is set by imposed that the displacement 
is zero and bending moments are zero at the boundaries. It can be proven that the zero bending moment condition is equivalent  that the 
Laplacian of the displacement is zero $\nabla^2 w= 0$. 
            
Since the module \verb|Boundary_value_problems|  only takes into account  second-order derivatives, the problem must be  transformed into the 
second-order problem by means of the transformation:
             \begin{equation*}
                \vect{u}(x,y) = [ \  w(x,y), \ v(x,y) \ ],
             \end{equation*}
which leads to the system: 
             \begin{align*}
                     &   \nabla^2 w  = v, \\
                     &   \nabla^2 v = p(x,y). 
            \end{align*}
The above equations are implemented in the function \verb|Elastic_Plate|
            \vspace{0.5cm} 
            \listings{\home/examples/API_example_Boundary_Value_Problem.f90}
            {function Elastic_Plate}
            {end function}
            {API_example_Boundary_Value_Problem.f90}
            
In this example, it is considered a vertical plate in $y$ direction submitted to ambient pressure to one side and a hydro-static pressure to 
the other side. 
Besides, the fluid at $ y=0$ has ambient pressure. With these considerations, the plate is submitted to the following non-dimensional net 
force: 
            $$
                p(x,y) = a y,
            $$
where $ a $ is a non-dimensional parameter. This external load is implemented in the function \verb|load| 
            \vspace{0.5cm} 
            \listings{\home/examples/API_example_Boundary_Value_Problem.f90}
                   {function load}
                   {end function}
                   {API_example_Boundary_Value_Problem.f90}
The boundary conditions  are: 
             \begin{align*}
                                w\big\rvert_{\partial \Omega}=0,  \\
                               \nabla^2 w \big\rvert_{\partial \Omega}=0. 
                   \end{align*}
Since $ v = \nabla^2  w $, these conditions leads to $ w=0, v=0 $ at the boundaries and they are implemented in the following function 
\verb|Plate_BCs|:     
              \vspace{0.5cm} 
            \listings{\home/examples/API_example_Boundary_Value_Problem.f90}
            {function Plate_BCs}
            {end function}
            {API_example_Boundary_Value_Problem.f90}
            
In this example, piecewise polynomial interpolation of degree $ q = 4 $ is chosen. The non-uniform grid points are selected by the subroutine 
\verb|Grid_initialization| by imposing constant truncation error. 
            
The differential operator \verb|Elastic_Plate| and its boundary conditions \verb|Plate_PBCs| are used as input arguments for the subroutine 
\verb|Boundary_value_problem|. 
            \vspace{0.5cm} 
            \listings{\home/examples/API_example_Boundary_Value_Problem.f90}
            {Elastic linear plate}
            {Solution}
            {API_example_Boundary_Value_Problem.f90}
            
In figure \ref{fig:LinearPlateBVP}a, the external load is shown. As it was mentioned, the net force between the hydro-static pressure and the 
ambient pressure takes zero value at the vertical position $ y=0$. For values $ y>0 $, the external load is positive and for values $ y<0$ 
the load is negative. This external load divides the plate vertically into two parts. A depressed lower part and a bulged upper part is shown 
in figure  \ref{fig:LinearPlateBVP}b.   
         
    

      \twographs
      {  \begin{tikzpicture}[]
     \begin{axis}[ height = \textwidth, width = \textwidth,
        title style={at={(0.5,-0.4)}, yshift=-0.05}, title={(a)},
        xlabel={ $x$ },    xmin=   -1.00, xmax =    1.00,
        xlabel={ $y$ },    ymin=   -1.00, ymax =    1.00,
         view={0}{90}, colormap/blackwhite ]
  \addplot3[contour/labels=false, contour gnuplot = { levels={ -100.00, -89.47, -78.95, -68.42, -57.89, -47.37, -36.84, -26.32, -15.79,  -5.26,   5.26,  15.79,  26.32,  36.84,  47.37,  57.89,  68.42,  78.95,  89.47, 100.00 }},
            mesh/ordering = y varies,
            mesh/rows =   21,
            mesh/cols =   21
           ]
           table [skip first n=2] {./doc/chapters/Boundary_Value_Problem/figures/Platea.plt};
     \end{axis}
 \end{tikzpicture}
}
      {  \begin{tikzpicture}[]
     \begin{axis}[ height = \textwidth, width = \textwidth,
        title style={at={(0.5,-0.4)}, yshift=-0.05}, title={(b)},
        xlabel={ $x$ },    xmin=   -1.00, xmax =    1.00,
        xlabel={ $y$ },    ymin=   -1.00, ymax =    1.00,
         view={0}{90}, colormap/blackwhite ]
  \addplot3[contour/labels=false, contour gnuplot = { levels={   -0.51,  -0.45,  -0.40,  -0.35,  -0.29,  -0.24,  -0.19,  -0.13,  -0.08,  -0.03,   0.03,   0.08,   0.13,   0.19,   0.24,   0.29,   0.35,   0.40,   0.45,   0.51 }},
            mesh/ordering = y varies,
            mesh/rows =   21,
            mesh/cols =   21
           ]
           table [skip first n=2] {./doc/chapters/Boundary_Value_Problem/figures/Plateb.plt};
     \end{axis}
 \end{tikzpicture}
}
      {Linear plate solution with $21\times 21$ nodal points and $q=4$. (a) External load $p(x,y)$. (b) Displacement 
      $w(x,y)$.}{fig:LinearPlateBVP}
       
 %__________________________________________________________________________________
 \newpage	  
 \section{Deflection of an elastic non linear plate}
   
A more complex example of a 2D boundary value problem is shown in this section. A nonlinear elastic plate submitted to a distributed load 
simply supported on its four edges is considered. The deflection $w$ of the nonlinear plate is ruled by the bi-harmonic equation plus a non 
linear term which  depends on the stress Airy function $\phi$. As in the section before, the simply supported edges are considered by 
imposing zero displacement and zero Laplacian of the displacement. The problem in a domain $\Omega \subset \mathbb{R}^2 : $ $\{(x,y)\in  
[-1,1]\times [-1,1]\}$ is formulated as:
          \begin{align*}
          & \nabla^4 w = p(x,y)  + \mu \ \mathcal{L}(w,\phi),  \\ 
          & \nabla^4 \phi  + \mathcal{L}(w,w) =0,
          \end{align*}
          where $\mu$ is a non-dimensional parameter and $\mathcal{L}$ is the bi-linear operator:
          \begin{align*}
          	\mathcal{L}(w,\phi)
          	=
          	\pdv[2]{w}{x}\pdv[2]{\phi}{y}
          	+
          	\pdv[2]{w}{y}\pdv[2]{\phi}{x}
          	-
          	2\pdv[2]{w}{x}{y}\pdv[2]{\phi}{x}{y}.
          \end{align*}
          
        
Since the module \verb|Boundary_value_problems|  only takes into account  second-order derivatives, the problem must be  transformed into the 
second-order problem by means of the transformation:
               \begin{equation*}
                  \vect{u}(x,y) = [ \  w, \ v, \ \phi, \ F ],
               \end{equation*}
               which leads to the system: 
               \begin{align*}
                      &    \nabla^2 w  = v, \\
                  %    &    \nabla^2 v = p(x,y)+ \frac{e}{D} \mathcal{L}(w,\phi), \\
                   &    \nabla^2 v = p(x,y)+ \mu \ \mathcal{L}(w,\phi), \\
                      &    \nabla^2 \phi = F, \\
                      &    \nabla^2 F = - \ \mathcal{L}(w,w).  \\
              \end{align*}
The external load applied to the nonlinear plate is the same that it was used in the deflections of the linear plate 
       $$
                       p(x,y) = a y. 
       $$
It allows comparing a linear solution with a nonlinear solution. The plate behaves non-linearly when deflections are of the order of the 
plate thickness. Since the deflections are caused by the external load, the non-dimensional parameter $ a $ can be used to take the plate to 
a nonlinear regime.   
     
The nonlinear plate equations expressed as a second order derivatives system of equations is implemented in the following function 
\verb|NL_Plate|: 
       
      
   \vspace{0.5cm} 
   \listings{\home/examples/API_example_Boundary_Value_Problem.f90}
   {function NL_Plate}
   {end function}
   {API_example_Boundary_Value_Problem.f90}
   In this function the bi-linear operator  $\mathcal{L}$ is implemented in the function \verb|Lb| 
   \vspace{0.5cm} 
   \listings{\home/examples/API_example_Boundary_Value_Problem.f90}
   {function Lb}
   {end function}
   {API_example_Boundary_Value_Problem.f90}
       
       
       
The boundary conditions are implemented in the function \verb|NL_Plate_BCs|
          \vspace{0.5cm} 
          \listings{\home/examples/API_example_Boundary_Value_Problem.f90}
          {function NL_Plate_BCs}
          {end function}
          {API_example_Boundary_Value_Problem.f90}
         
        
     
The differential operator \verb|NL_Plate| and its boundary conditions \verb|NL_Plate_BCs| are used as input arguments for the subroutine 
\verb|Boundary_value_problem|.       
          \vspace{0.5cm} 
          \listings{\home/examples/API_example_Boundary_Value_Problem.f90}
          {Elastic nonlinear plate}
          {Solution}
          {API_example_Boundary_Value_Problem.f90}
          
        
In figure \ref{fig:NonLinearPlateBVP}, the solution of the nonlinear plate model is shown. 
          
             
          
          \twographs
          {  \begin{tikzpicture}[]
     \begin{axis}[ height = \textwidth, width = \textwidth,
        title style={at={(0.5,-0.4)}, yshift=-0.05}, title={(a)},
        xlabel={ $x$ },    xmin=   -1.00, xmax =    1.00,
        xlabel={ $y$ },    ymin=   -1.00, ymax =    1.00,
         view={0}{90}, colormap/blackwhite ]
  \addplot3[contour/labels=false, contour gnuplot = { levels={   -0.30,  -0.27,  -0.24,  -0.21,  -0.17,  -0.14,  -0.11,  -0.08,  -0.05,  -0.02,   0.02,   0.05,   0.08,   0.11,   0.14,   0.17,   0.21,   0.24,   0.27,   0.30 }},
            mesh/ordering = y varies,
            mesh/rows =   21,
            mesh/cols =   21
           ]
           table [skip first n=2] {./doc/chapters/Boundary_Value_Problem/figures/NLPlatea.plt};
     \end{axis}
 \end{tikzpicture}
}
          {  \begin{tikzpicture}[]
     \begin{axis}[ height = \textwidth, width = \textwidth,
        title style={at={(0.5,-0.4)}, yshift=-0.05}, title={(b)},
        xlabel={ $x$ },    xmin=   -1.00, xmax =    1.00,
        xlabel={ $y$ },    ymin=   -1.00, ymax =    1.00,
         view={0}{90}, colormap/blackwhite ]
  \addplot3[contour/labels=false, contour gnuplot = { levels={   -0.06,  -0.05,  -0.05,  -0.05,  -0.04,  -0.04,  -0.04,  -0.04,  -0.03,  -0.03,  -0.03,  -0.02,  -0.02,  -0.02,  -0.01,  -0.01,  -0.01,  -0.01,  -0.00,   0.00 }},
            mesh/ordering = y varies,
            mesh/rows =   21,
            mesh/cols =   21
           ]
           table [skip first n=2] {./doc/chapters/Boundary_Value_Problem/figures/NLPlateb.plt};
     \end{axis}
 \end{tikzpicture}
}
          {Non linear elastic plate solution with $ N_x = 20 $ , $ N_y = 20 $, $q=4$ and $\mu=100$. (a) Displacement $w(x,y)$.
          (b) Solution $\phi(x,y)$}{fig:NonLinearPlateBVP}






