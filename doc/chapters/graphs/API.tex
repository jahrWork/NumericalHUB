\chapter{Plotting graphs with Latex }

\section{Overview}

This library is designed to plot $x-y$ and contour graphs on the screen or to create automatically Latex files with graphs and figures. 
The  module:  \verb|Plots| has two subroutines:  \verb|plot_parametrics| and \verb|plot_contour|.



\section{Plot parametrics}

\begin{lstlisting}[frame=trBL]
  call plot_parametrics(x, y, legends, x_label, y_label, title, & 
                        path, graph_type)
                      
\end{lstlisting}   

  
This subroutine plots a given number of parametric curves $(x,y)$ on the screen and creates a Latex file for optimum quality results. 
This subroutine is overloaded allowing to plot parametric curves sharing $ x$--axis for all curves or with different data booth for $ x $ and $y$ axis. That is, $ x $ can be a vector the same for all parametric curves or a matrix. In this case, $( x_{ij}, y_{ij})$ represents the the point
 $ i $  of the parametric curve $ j$.  The last three arguments are optional. If they are given, this subroutine creates a plot data file (\verb|path.plt|)  and a latex file (\verb|path.tex|) to show the same graphics results by compiling a latex document.

\btable
	x & vector or matrix of reals & in &   First index is the point and second index is the parametric curve.  \\ \hline
	
	y & matrix of reals & in &  First index is the point and second index is the parametric curve.  \\ \hline
	
	legends &  vector of  char strings  & in & These are the legends of the parametric curves.  \\ \hline
	
	x\_label &  character string & in  & $x$ label of the graph.   \\ \hline
	
	y\_label &  character string & in  & $y$ label of the graph.   \\ \hline
	
	title & character string & optional in  &  title of the graph.   \\ \hline
	
	path  &  character string & optional in  & path of Latex and data files.  \\ \hline
	
	graph\_type  &  character string &  optional, in  & graph type   \\ \hline
\etable{Description of \texttt{plot\_parametrics} arguments for Latex graphs}


\vspace{-0.3cm}
  \listings{./sources/my_examples.f90}
  {subroutine myexampleC}
  {end subroutine}{my_examples.f90}
  
The above Fortran example creates automatically four plot files and four latex files. 
By compiling  the following Latex file, the same plots showed on the screen can be included in any latex manuscript. 
 
\vspace{0.5cm} 
\listings{./latex.tex}{documentclass}{exampleCad}{Latex.tex}

\newpage
After compiling the above Latex code, the plot of figure \ref{fig:exampleCad} is obtained. 
\vspace{1cm} 

\fourgraphs
      {  \begin{tikzpicture}[]
     \begin{axis}[ width = \textwidth,
        title style={at={(0.5,-0.45)},anchor=south}, title={(a)},
        xlabel={ $x$ },
        ylabel={ $y$ },
        {}, legend style={font=\small},  {mark repeat =  9, cycle list name = black white, legend columns = 1 }, legend pos = north east  ]
        \foreach \x/\y in { 0/1,2/3,4/5}
        \addplot table [skip first n=2, x index=\x, y index=\y]{./results/myexampleCa.plt   };
        \legend{ $\sin x$, $\cos x$, $\sin 2x$ }
     \end{axis}
  \end{tikzpicture}
 }
      {  \begin{tikzpicture}[]
     \begin{axis}[ width = \textwidth,
        title style={at={(0.5,-0.45)},anchor=south}, title={(b)},
        xlabel={ $y_2$ },
        ylabel={ $y_1$ },
        {}, legend style={font=\small},  {no marks, legend columns = 1 } ]
        \foreach \x/\y in { 0/1,2/3,4/5}
        \addplot table [skip first n=2, x index=\x, y index=\y]{./results/myexampleCb.plt   };
        \legend{ O1, O2, O3 }
     \end{axis}
  \end{tikzpicture}
 }
      {  \begin{tikzpicture}[]
     \begin{axis}[ width = \textwidth,
        title style={at={(0.5,-0.45)},anchor=south}, title={(c)},
        xlabel={ $y_2$ },
        ylabel={ $y_1$ },
        {}, legend style={font=\small},  {no marks, legend columns = 1 } ]
        \foreach \x/\y in { 0/1}
        \addplot table [skip first n=2, x index=\x, y index=\y]{./results/myexampleCc.plt   };
        \legend{ O2 }
     \end{axis}
  \end{tikzpicture}
 }
      {  \begin{tikzpicture}[]
     \begin{axis}[ width = \textwidth,
        title style={at={(0.5,-0.45)},anchor=south}, title={(d)},
        xlabel={ $y_2$ },
        ylabel={ $y_1$ },
        {}, legend style={font=\small},  {no marks, legend columns = 1 } ]
        \foreach \x/\y in { 0/1}
        \addplot table [skip first n=2, x index=\x, y index=\y]{./results/myexampleCd.plt   };
        \legend{ O3 }
     \end{axis}
  \end{tikzpicture}
 }
      {Heinon-Heiles system solution. 
      (a) Trajectory of the star $(x,y)$. 
      (b) Projection $(x,\dot{x})$ of the solution. 
      (c) Projection $(y,\dot{y})$ of the solution. 
      (d) Projection $(\dot{x},\dot{y})$.}
      {fig:exampleCad}   
      
    
\newpage     
\section{Plot contour} 
\begin{lstlisting}[frame=trBL]
  call plot_contour(x, y, z, x_label, y_label, levels, legend, & 
                    path, graph_type)
                      
\end{lstlisting}   

  
This subroutine plots a contour map of  $z = z(x,y)$ on the screen and creates a Latex file for optimum quality results. 
Given a set of values $ x_i $ and $ y_j $ where some function $ z(x,y)$ is evaluated, this subroutine plot a contour map. 
The last three arguments are optional. If they are given, this subroutine creates a plot data file (\verb|path.plt|)  and a latex file (\verb|path.tex|) to show the same graphics results by compiling a latex document.

\btable 
	x & vector  of reals & in &  $x_i$ grid values.  \\ \hline
	
	y & vector of reals & in &  $y_j$ grid values.   \\ \hline
	
	z & matrix of reals & in &  $z_{ij}$ different evaluations of  $z(x,y)$.  \\ \hline
	
	x\_label &  character string & in  & $x$ label of the graph.   \\ \hline
	
	y\_label &  character string & in  & $y$ label of the graph.   \\ \hline
	
	levels &  vector of  reals  & optional in & Levels for the iso--lines.  \\ \hline
	
	legend & character string & optional in  &  title of the graph.   \\ \hline
	
	path  &  character string & optional in  &  Latex and data files.  \\ \hline
	
	graph\_type  &  character string &  optional in  & "color" or "isolines"   \\ \hline
\etable{Description of \texttt{plot\_contour} arguments}


  \newpage
  \listings{./sources/my_examples.f90}
  {subroutine myexampleD}
  {end subroutine}{my_examples.f90}
The above Fortran example creates the following  data files and latex files: 
 
  \verb|./results/myexampleDa.plt|,  \  \verb|./results/myexampleDa.tex|,
  
  \verb|./results/myexampleDb.plt|, \ \verb|./results/myexampleDb.tex|.
  
By compiling  the following Latex file, the same plots showed on the screen are included in any latex manuscript. 
 
\vspace{0.5cm} 
\listings{./latex.tex}{twographs}{:exampleDa}{Latex.tex}

To compile successfully the above code,  \verb|gnuplot| must be installed in the computer. Besides,  during installation, the path environmental variable
should be added. If \verb|TexStudio| is used to compile the Latex file, the \verb|lualatex| and \verb|PDFLatex| orders should be modified as follows: 

\verb|pdflatex -synctex=1 -interaction=nonstopmode -shell-escape %.tex|

\verb|lualatex.exe -synctex=1 -interaction=nonstopmode -shell-escape %.tex|




The results are shown in figure  \ref{fig:exampleDad2}. 
\vspace{1cm} 

%\twographs
%      {  \begin{tikzpicture}[]
     \begin{axis}[ height = 0.8\textwidth, width = 0.8\textwidth,
        title style={at={(0.5,-0.5)}, yshift=-0.1}, title={(a)},
        colorbar, domain=-2:2,
        xlabel={ x }, ylabel={ y },
        view={0}{90}, colormap/bluered ]
  \addplot3[smooth, surf, mesh/rows=21, shader=interp, z buffer=sort] table [skip first n=2]{./results/myexampleDa.plt};
     \end{axis}
 \end{tikzpicture}
 }
%      {  \begin{tikzpicture}[]
     \begin{axis}[ height = 0.8\textwidth, width = 0.8\textwidth,
        title style={at={(0.5,-0.5)}, yshift=-0.1}, title={(a)},
        colorbar, domain=-2:2,
        xlabel={ x }, ylabel={ y },
        view={0}{90}, colormap/bluered ]
  \addplot3[smooth, surf, mesh/rows=21, shader=interp, z buffer=sort] table [skip first n=2]{./results/myexampleDa.plt};
     \end{axis}
 \end{tikzpicture}
 }
%      {Color map. 
%      (a) $ z = \sin x \sin y$. 
%      (a) $ z = \sin x \sin y$. 
%      }
%      {fig:exampleDad1}  
\twographs
      {  \begin{tikzpicture}[]
     \begin{axis}[ height = \textwidth, width = \textwidth,
        title style={at={(0.5,-0.4)}, yshift=-0.05}, title={(b)},
        xlabel={ x },
        ylabel={ y },
         view={0}{90}, colormap/blackwhite ]
  \addplot3[contour gnuplot = { levels={  -1.0, -0.8, -0.6, -0.4, -0.2,  0.0,  0.2,  0.4,  0.6,  0.8,  1.0 }},
            mesh/ordering = y varies,
            mesh/rows =   21,
            mesh/cols =   21
           ]
           table [skip first n=2] {./results/myexampleDb.plt};
     \end{axis}
 \end{tikzpicture}
 }
      {  \begin{tikzpicture}[]
     \begin{axis}[ height = \textwidth, width = \textwidth,
        title style={at={(0.5,-0.4)}, yshift=-0.05}, title={(b)},
        xlabel={ x },
        ylabel={ y },
         view={0}{90}, colormap/blackwhite ]
  \addplot3[contour gnuplot = { levels={  -1.0, -0.8, -0.6, -0.4, -0.2,  0.0,  0.2,  0.4,  0.6,  0.8,  1.0 }},
            mesh/ordering = y varies,
            mesh/rows =   21,
            mesh/cols =   21
           ]
           table [skip first n=2] {./results/myexampleDb.plt};
     \end{axis}
 \end{tikzpicture}
 }
      {Isolines. 
      (a) $ z = \sin x \sin y$. 
      (a) $ z = \sin x \sin y$. 
      }
      {fig:exampleDad2}      
