\chapter{Initial Boundary Value Problem}
\vspace{-0.5cm}
\section{Overview}

This library is intended  to solve an initial value boundary problem. 
This problem is governed by a set time evolving partial differential equations together with boundary conditions and an initial condition. 

\vspace{0.1cm}
 \renewcommand{\home}{./sources/Numerical_Methods/Initial_Boundary_Value_Problem}
  \listings{\home/sources/Initial_Boundary_Value_Problems.f90}
       {module Initial_Boundary_Value_Problems}
       {Linear_operator}{Initial_Boundary_Value_Problems.f90}
\vspace{-0.1cm}       
Since the space domain $ \Omega \subset \R{k} $ with $ k=1,2,3$, initial value boundary problems are stated in 1D, 2D and 3D grids. 
To have the same name interface when dealing with different space dimensions,  
the subroutines 
\verb|Initial_Value_Boundary_Problem|, \verb|Spatial_discretization|,
 \verb|Spatial_truncation_error| and 
\verb|Linear_operator| have been overloaded. 



\newpage
%************************************************************************************
\section{Initial  Boundary Value Problem}
\subsection*{1D Initial Boundary Value Problem for systems of equations}

\begin{lstlisting}[frame=trBL]
call Initial_Boundary_Value_Problem(Time_Domain, x_nodes,                & 
                                    Differential_operator,               & 
                                    Boundary_conditions, Solution, Scheme)    
\end{lstlisting} 
The subroutine \verb|Initial_Value_Boundary_Problem| calculates the solution at 
\verb|x_nodes| of the following problem:

\begin{align*}
\frac{\partial \vect{u}}{\partial t} = 
\vect{\mathcal{L}}(x,t, \vect{u}, \vect{u}_x,  \vect{u}_{xx} ), \qquad 
\eval{\vect{h} ( x,t,\vect{u}, \vect{u}_x)}_{\partial \Omega} = 0.
\end{align*}
Together with an initial condition  $\vect{u}(x,0) = \vect{u}_0 (x)$, the solution is 
simulated in some time domain.
The arguments of the subroutine are described in the following code:
\vspace{0.1cm}
 \renewcommand{\home}{./sources/Numerical_Methods/Initial_Boundary_Value_Problem}
  \listings{\home/sources/Initial_Boundary_Value_Problem1D.f90}
       {:: Time_Domain}
       {Scheme}{Initial_Boundary_Value_Problem1D.f90}
The first index of \verb|Solution| stands for time, the second space and third is the index of variable 
to integrate. 
Besides, the differential operator  $ \vect{\mathcal{L}} $ and the boundary conditions $\vect{h} $ are defined with the following interface: 
  \listings{\home/sources/Initial_Boundary_Value_Problem1D.f90}
         {abstract interface}
         {end interface}{Initial_Boundary_Value_Problem1D.f90}     
       

\newpage
\subsection*{Initial Boundary Value  Problem for 2D systems of equations}

\begin{lstlisting}[frame=trBL]
call Initial_Boundary_Value_Problem(                                &
             Time_Domain, x_nodes, y_nodes,  Differential_operator, & 
             Boundary_conditions,  Solution, Scheme                 )  
\end{lstlisting}   
The subroutine \verb|Initial_Value_Boundary_ProblemS| calculates the solution 
at (\verb|x_nodes|, \verb|y_nodes|) of the following problem: 


\begin{align*}
\frac{\partial \vect{u}}{\partial t} = 
\vect{\mathcal{L}}(x,y,t, \vect{u}, \vect{u}_x, \vect{u}_y, \vect{u}_{xx}, \vect{u}_{yy}, \vect{u}_{xy}), \qquad \eval{\vect{h} ( 
x,y,t,\vect{u}, \vect{u}_x, \vect{u}_y)}_{\partial \Omega} = 0.
\end{align*}
Together with an initial condition  $\vect{u}(x,y,0) = \vect{u}_0 (x,y)$, the solution is 
simulated in some time domain.
The arguments of the subroutine are described in the following code:

\vspace{0.1cm}
 \renewcommand{\home}{./sources/Numerical_Methods/Initial_Boundary_Value_Problem}
  \listings{\home/sources/Initial_Boundary_Value_Problem2D.f90}
       {:: Time_Domain}
       {Scheme}{Initial_Boundary_Value_Problem2D.f90}
The first index of \verb|Solution| stands for time, the second \verb|x_nodes| position, 
the third  \verb|x_nodes| position and the fourth is the index of variable 
to integrate. 
Besides, the differential operator  $ \vect{\mathcal{L}} $ and the boundary conditions $\vect{h} $ are defined with the following interface: 
  \listings{\home/sources/Initial_Boundary_Value_Problem2D.f90}
         {abstract interface}
         {end interface}{Initial_Boundary_Value_Problem1D.f90}     
       


%********************************************************************************************
\newpage
\section{Spatial discretization}
To carry out the numerical integration of the initial boundary value problem, the method of lines 
is used. This method performs the spatial discretization of the differential operator
together with the boundary conditions. 
Once the spatial discretization is accomplished, the following ordinary systems of equations 
is obtained: 
$$
   \frac{ d \vect{U} } { dt } = \vect{F}( \vect{U}, t),
$$
where $ \vect{F}( \vect{U}, t) $ represents the spatial discretization of the problem. 
The following function allows to calculate the space discretization: 
\vspace{0.5cm}
\begin{lstlisting}[frame=trBL]
F = Spatial_discretization(Differential_operator,  & 
                           Boundary_conditions, x, y, U, t) 
\end{lstlisting} 

\subsection*{1D Spatial discretization for systems of equations}

\vspace{0.5cm}
 \renewcommand{\home}{./sources/Numerical_Methods/Initial_Boundary_Value_Problem}
   \listings{\home/sources/Initial_Boundary_Value_Problem1D.f90}
        {function Spatial_discretization1DS}
        {real ::}{Initial_Boundary_Value_Problem1D.f90}
The differential operator and the boundary conditions share the same definition than the subroutine 
\verb|Initial_Boundary_Value_Problem|. The argument \verb|x| represents the grid nodes which 
are previously calculated by \verb|Grid_initialization|. 
The argument \verb|U| represents the solution or some test function evaluated at all grid nodes. 
The first index stands for space and second for the variable.   
The result discretization \verb|F| share the dimensions of \verb|U|. 
It is important to mention that the vector \verb|F| includes inner and boundary points. 
   
%
 
\newpage
\subsection*{Spatial discretization for 2D systems of equations}
\vspace{0.5cm}
 \renewcommand{\home}{./sources/Numerical_Methods/Initial_Boundary_Value_Problem}
   \listings{\home/sources/Initial_Boundary_Value_Problem2D.f90}
        {function Spatial_discretization2DS}
        {real ::}{Initial_Boundary_Value_Problem2D.f90}
The differential operator and the boundary conditions share the same definition than the subroutine 
\verb|Initial_Boundary_Value_Problem|. The argument \verb|x| and \verb|y| represent the grid nodes which 
are previously calculated by \verb|Grid_initialization|. 
The argument \verb|U| represents the solution or some test function evaluated 
at all grid nodes $ (x_i, y_j)$. 
The first and the second index of \verb|U| stand for the grid nodes and the third for the variable.   
The result discretization \verb|F| share the dimensions of \verb|U|. 
The matrix \verb|F| includes inner and boundary points. 
Since, spatial discretization is accomplished for those points in which 
no boundary conditions are imposed, those points that must satisfy boundary conditions have zero 
components of the matrix \verb|F|. 

%********************************************************************************************
\newpage
\section{Spatial truncation error}
The spatial truncation error is defined at different grid points $ (x_i, y_j) $ by the following equation: 
$$
   \vect{R}_{ij} = 
   \vect{\mathcal{L}}(x,y,t, \vect{u}, \vect{u}_x, \vect{u}_y, \vect{u}_{xx}, \vect{u}_{yy}, \vect{u}_{xy}) |_{(x_i, y_j)} - \vect{F}_{ij} 
$$
where $ F_{ij} $ represents the spatial discretization of some test function $ \vect{u}(x,y,t) $. 
The first term of the above equation represents the differential operator together 
with the boundary conditions applied to same  test function evaluated at all different grid points. 
The definition of the spatial truncation error 
requires to differentiate analytically the test function to determine the spatial truncation error.
The following function allows to calculate this error numerically by means of two grids using Richardson 
extrapolation.  
\vspace{0.5cm}
\begin{lstlisting}[frame=trBL]
 R = Spatial_Truncation_Error( Nv, Differential_operator,  & 
                               Boundary_conditions,        & 
                               x, y, Order, Test_U  ) 
\end{lstlisting} 

\subsection*{1D Spatial truncation error for systems of equations}

%\vspace{0.5cm}
 \renewcommand{\home}{./sources/Numerical_Methods/Initial_Boundary_Value_Problem}
   \listings{\home/sources/Initial_Boundary_Value_Problem1D.f90}
        {function Spatial_Truncation_Error1DS}
        {real :: R}{Initial_Boundary_Value_Problem1D.f90}
The differential operator and the boundary conditions share the same definition than the subroutine 
\verb|Initial_Boundary_Value_Problem|. \verb|Nv| stands for the number of variables.  
For example, the heat equation has $ N_v = 1 $ and the wave equation has $ Nv =2$. 
The argument \verb|x| represents the grid nodes which 
are previously calculated by \verb|Grid_initialization| and \verb|Order| stands for the degree of the 
interpolation to accomplish the space discretization. 
The $\verb|Test_function|$ evaluates any test function $u(x)$ at the grid points.   
The first index of the result \verb|R| allows to know the truncation error at every grid node $ x_i $  and 
the second index stands for different variables. 

 

\subsection*{Spatial truncation error for 2D systems of equations}
\vspace{0.5cm}
 \renewcommand{\home}{./sources/Numerical_Methods/Initial_Boundary_Value_Problem}
   \listings{\home/sources/Initial_Boundary_Value_Problem2D.f90}
        {function Spatial_Truncation_Error2DS}
        {real :: R}{Initial_Boundary_Value_Problem2D.f90}
        
The differential operator and the boundary conditions share the same definition than the subroutine 
\verb|Initial_Boundary_Value_Problem|. \verb|Nv| stands for the number of variables. 
The arguments \verb|x| and \verb|y| represent the grid nodes which 
are previously calculated by \verb|Grid_initialization| and \verb|Order| stands for the degree of the 
interpolation to accomplish the space discretization. 
The $\verb|Test_function|$ evaluates any test function $u(x,y)$ at the grid points.   
The first index and the second index of the result \verb|R| 
allows to know the truncation error at every grid node $ (x_i, y_j) $  and 
the third index stands for different variables. 







%********************************************************************************************
\newpage
\section{Linear operator}
If the differential operator and the boundary conditions are linear, 
and once the space discretization is accomplished, an ordinary system of differential equations is obtained.  
$$
   \frac{ dU } { dt } = A \ U  + b,
$$
where $ A $ is the system matrix of size $ M \times M $ where $ M $ is the dimension of the vector $ U $ 
which represents the solution of all variables at all different grid or collocation nodes. 
The following function allows to obtain the matrix $ A$ associated to the space discretization problem: 
\vspace{0.5cm}
\begin{lstlisting}[frame=trBL]
A = Linear_operator( Nv, x, y, Order, Differential_operator, & 
                     Boundary_conditions) 
\end{lstlisting}  

\subsection*{Linear operator 1D for systems of equations}

\vspace{0.5cm}
 \renewcommand{\home}{./sources/Numerical_Methods/Initial_Boundary_Value_Problem}
   \listings{\home/sources/Initial_Boundary_Value_Problem1D.f90}
        {function Linear_operator1DS}
        {real ::}{Initial_Boundary_Value_Problem1D.f90}
The differential operator and the boundary conditions share the same definition than the subroutine 
\verb|Initial_Boundary_Value_Problem|. 
\verb|Nv| stands for the number of variables. 
The argument \verb|x| represents the grid nodes which 
are previously calculated by \verb|Grid_initialization| and \verb|Order| stands for the degree of the 
interpolation to accomplish the space discretization.   
The function returns a square matrix \verb|A|  of size $ M \times M $ with $ M = (N_x+1)  Nv $
which includes inner and boundary points.  
%
 
\newpage
\subsection*{Linear operator 2D for systems of equations}
\vspace{0.5cm}
 \renewcommand{\home}{./sources/Numerical_Methods/Initial_Boundary_Value_Problem}
   \listings{\home/sources/Initial_Boundary_Value_Problem2D.f90}
        {function Linear_operator2DS}
        {real ::}{Initial_Boundary_Value_Problem2D.f90}
The differential operator and the boundary conditions share the same definition than the subroutine 
\verb|Initial_Boundary_Value_Problem|. 
\verb|Nv| stands for the number of variables. 
The arguments  \verb|x| and \verb|y| represens the grid nodes which 
are previously calculated by \verb|Grid_initialization| and \verb|Order| stands for the degree of the 
interpolation to accomplish the space discretization.   
The function returns a square matrix \verb|A|  of size $ M \times M $ with $ M = (N_x+1) (N_y+1)  Nv $
which includes inner and boundary points. 
Those rows associated to boundary points have all components equal to zero. 


