     
 %*************************************************************************
\chapter{Mixed Boundary and Initial Value Problems   }\label{user:IBVP_BVP}
 %*************************************************************************  
\vspace{-1cm}
\section{Overview}
 In this chapter, a mixed problem coupled with elliptic and parabolic equations  is solved making use of the module \verb|IBVP_and_BVP|. 
 Briefly and not rigorously, these problems are governed by a parabolic time dependent problem for $\vect{u}(\vect{x}, t)$ and an elliptic problem or a boundary value problem for $\vect{v}(\vect{x}, t)$. 
 Let $\Omega $  
 be an open and connected set and $\partial \Omega$ its boundary. 
 These problems are formulated with the following set of equations: 
           \begin{align*}
           & \frac{\partial \vect{u} }{\partial t}(\vect{x},t) =\vect{\mathcal{L}}_u (\vect{x},t,\vect{u}(\vect{x},t),\vect{v}(\vect{x},t)) ,   & \forall & \ \vect{x} \in  \Omega, \\
           & \vect{h}_u (\vect{x},t,\vect{u}(\vect{x},t))\big\rvert_{\partial \Omega}=0 , & \forall & \ \vect{x} \in \partial \Omega, \\ 
           & \vect{u}(\vect{x},t_0)=\vect{u}_0(\vect{x}),  & \forall & \ \vect{x} \in  D, \\ \\
           & \vect{\mathcal{L}}_v (\vect{x},t,\vect{v}(\vect{x},t),\vect{u}(\vect{x},t)) = 0 ,  & \forall & \ \vect{x} \in  \Omega, \\
           & \vect{h}_v (\vect{x},t,\vect{v}(\vect{x},t))\big\rvert_{\partial \Omega}=0 , & \forall & \ \vect{x} \in \partial \Omega,
           \end{align*}
 where $\vect{\mathcal{L}}_u$ is the spatial differential operator of the initial value problem of $N_u$ equations, $\vect{u}_0(\vect{x})$ is the initial value, $\vect{h}_u$ represents the boundary conditions operator for the solution at the boundary points $\vect{u} \big\rvert_{\partial \Omega}$,  $\vect{\mathcal{L}}_v$ is the spatial differential operator of the boundary value problem of $N_v$ equations and $\vect{h}_v$ represents the boundary conditions operator for $\vect{v}$  at the boundary points $\vect{v} \big\rvert_{\partial \Omega}$. 
       
     
       
\newpage      
\section{Non Linear Plate Vibration}
The vibrations $ w(x,y,t)$  of an nonlinear plate subjected to a transversal load $ p(x,y,t) $ are governed by the following set of equations: 
\begin{align*}
\frac{\partial^2{w}}{\partial{t}^2}+ \nabla^4{w}
&= p(x,y,t) + \mu \ \mathcal{B}(w,\phi), \\ 
\nabla^4 \phi & + \mathcal{B}(w,w) =0 ,
\end{align*}
where $\mathcal{B}$ is the bilinear operator:
%
\begin{align*}
	\mathcal{B}(w,\phi) 
	= 
	\pdv[2]{w}{x}\pdv[2]{\phi}{y}
	+
	\pdv[2]{\phi}{x}\pdv[2]{w}{y}
	-
	2\pdv[2]{\phi}{x}{y}\pdv[2]{w}{x}{y}.
\end{align*}
These equations together with boundary and initial conditions allow predicting the oscillations of the plate. Since second order derivatives of the displacement $ w(x,y,t) $ are involved, the initial position and the initial velocity are given. In this example, 
\begin{align*}
& w(x,y,0)  =   e^{-10(x^2+y^2)}. \\ 
& \frac{\partial w}{\partial t} (x,y,0) = 0. 
\end{align*} 
Besides, simple supported edges are considered which means that the displacement  $ w(x,y,t) $ and the bending moments $ \nabla^2  w $ are zero at boundaries. 
In this example, the  spatial domain $\Omega \equiv \{ (x,y) \in [-1,1] \times [-1,1] \}$ and the time domain  $t \in [0,1]$.
Since the module \verb|IBVP_and_BVP| is written for systems of second order derivatives in space and first order in time, the problem is rewritten by means of the following transformation:
        \begin{equation*}
           \vect{u} = [ \  w, \ w_2, \ w_3 \ ],
        \end{equation*}
        which leads to the evolution system of equations for $ \vect{u}(x,y,t) $ with: 
        \begin{align*}
                    \vect{\mathcal{L}}_u = [ \  w_2, \  -\nabla^2 w_3 + p + \mu \ \mathcal{B}(w, \phi),  \ \nabla^2 w_2 \ ],
       \end{align*} 
To implement the elliptic boundary problem in terms of second order derivatives, the following transformation is used:
                       \begin{equation*}
                          \vect{v} = [ \ \phi, \ F \ ],
                       \end{equation*}
which leads to the system  $ \vect{\mathcal{L}}_v (\vect{x},t,\vect{v},\vect{u}) = 0 $ with: 
        \begin{align*}
                   \vect{\mathcal{L}}_v = [ \  \nabla^2 \phi - F, \  \nabla^2 F +  \mathcal{B}(w, w) \ ].
        \end{align*}
        
The evolution differential operator $ \vect{\mathcal{L}}_u $ together with the elliptic differential $ \vect{\mathcal{L}}_v $ are implemented in the following vector functions: 
\vspace{0.5cm} 
\listings{\home/examples/API_Example_IBVP_and_BVP.f90}
{function Lu}
{end function}{API_Example_IBVP_and_BVP.f90}
   
\vspace{0.5cm} 
\listings{\home/examples/API_Example_IBVP_and_BVP.f90}
{function Lv}
{end function}{API_Example_IBVP_and_BVP.f90}


To impose simple supported edges, the values of all components of  $ \vect{u}(x,y,t) $ and 
 $ \vect{v}(x,y,t) $ must be determined analytically at boundaries. Since $ w(x,y,t) $ is zero at boundaries for all time and $  w_2  = \partial w / \partial t$ then, $ w_2 $ is zero at boundaries. Since $ \nabla^2 w $ is zero at boundaries for all time and 
 $$ 
 \frac{\partial w_3 }{ \partial t }  = \frac{\partial  \nabla^2 w }{  \partial t}
 $$ 
 then, $ w_3 $ is zero at boundaries. The same reasoning is applied to determine
 $ \vect{v}(x,y,t) $ components at boundaries.% For the initial conditions 
With these considerations, the boundary conditions $ \vect{h}_u $ and  $ \vect{h}_v $ are implemented by:
\vspace{0.5cm} 
\listings{\home/examples/API_Example_IBVP_and_BVP.f90}
{function BCu}
{end function}{API_Example_IBVP_and_BVP.f90}


\vspace{0.5cm} 
\listings{\home/examples/API_Example_IBVP_and_BVP.f90}
{function BCv}
{end function}{API_Example_IBVP_and_BVP.f90}


These differential operators \verb|Lu| and \verb|Lv| together with their boundary conditions \verb|BCu| and \verb|BCv|  are used as input arguments for the subroutine \verb|IBVP_and_BVP|

\vspace{0.5cm} 
\listings{\home/examples/API_Example_IBVP_and_BVP.f90}
{Nonlinear Plate Vibration}
{IBVP}{API_Example_IBVP_and_BVP.f90}


In figure \ref{fig:NLPlate_Vibrations_Pulse}, the oscillations of a plate with zero external loads starting from an elongated position with zero velocity  are shown. 

\fourgraphs
{  \begin{tikzpicture}[]
     \begin{axis}[ height = \textwidth, width = \textwidth,
        title style={at={(0.5,-0.4)}, yshift=-0.05}, title={(a)},
        xlabel={ $x$ },    xmin=   -1.00, xmax =    1.00,
        xlabel={ $y$ },    ymin=   -1.00, ymax =    1.00,
         view={0}{90}, colormap/blackwhite ]
  \addplot3[contour/labels=false, contour gnuplot = { levels={    0.00,   0.05,   0.11,   0.16,   0.21,   0.26,   0.32,   0.37,   0.42,   0.47,   0.53,   0.58,   0.63,   0.68,   0.74,   0.79,   0.84,   0.89,   0.95,   1.00 }},
            mesh/ordering = y varies,
            mesh/rows =   11,
            mesh/cols =   11
           ]
           table [skip first n=2] {./doc/chapters/IBVP_BVP/figures/NLvibrationsa.plt};
     \end{axis}
 \end{tikzpicture}
}
{  \begin{tikzpicture}[]
     \begin{axis}[ height = \textwidth, width = \textwidth,
        title style={at={(0.5,-0.4)}, yshift=-0.05}, title={(b)},
        xlabel={ $x$ },    xmin=   -1.00, xmax =    1.00,
        xlabel={ $y$ },    ymin=   -1.00, ymax =    1.00,
         view={0}{90}, colormap/blackwhite ]
  \addplot3[contour/labels=false, contour gnuplot = { levels={   -0.04,  -0.02,  -0.00,   0.01,   0.03,   0.04,   0.06,   0.07,   0.09,   0.11,   0.12,   0.14,   0.15,   0.17,   0.18,   0.20,   0.22,   0.23,   0.25,   0.26 }},
            mesh/ordering = y varies,
            mesh/rows =   11,
            mesh/cols =   11
           ]
           table [skip first n=2] {./doc/chapters/IBVP_BVP/figures/NLvibrationsb.plt};
     \end{axis}
 \end{tikzpicture}
}
{  \begin{tikzpicture}[]
     \begin{axis}[ height = \textwidth, width = \textwidth,
        title style={at={(0.5,-0.4)}, yshift=-0.05}, title={(c)},
        xlabel={ $x$ },    xmin=   -1.00, xmax =    1.00,
        xlabel={ $y$ },    ymin=   -1.00, ymax =    1.00,
         view={0}{90}, colormap/blackwhite ]
  \addplot3[contour/labels=false, contour gnuplot = { levels={   -0.36,  -0.34,  -0.31,  -0.29,  -0.27,  -0.25,  -0.22,  -0.20,  -0.18,  -0.16,  -0.14,  -0.11,  -0.09,  -0.07,  -0.05,  -0.02,  -0.00,   0.02,   0.04,   0.07 }},
            mesh/ordering = y varies,
            mesh/rows =   11,
            mesh/cols =   11
           ]
           table [skip first n=2] {./doc/chapters/IBVP_BVP/figures/NLvibrationsc.plt};
     \end{axis}
 \end{tikzpicture}
}
{  \begin{tikzpicture}[]
     \begin{axis}[ height = \textwidth, width = \textwidth,
        title style={at={(0.5,-0.4)}, yshift=-0.05}, title={(d)},
        xlabel={ $x$ },    xmin=   -1.00, xmax =    1.00,
        xlabel={ $y$ },    ymin=   -1.00, ymax =    1.00,
         view={0}{90}, colormap/blackwhite ]
  \addplot3[contour/labels=false, contour gnuplot = { levels={   -0.40,  -0.38,  -0.36,  -0.33,  -0.31,  -0.29,  -0.26,  -0.24,  -0.22,  -0.20,  -0.17,  -0.15,  -0.13,  -0.10,  -0.08,  -0.06,  -0.03,  -0.01,   0.01,   0.03 }},
            mesh/ordering = y varies,
            mesh/rows =   11,
            mesh/cols =   11
           ]
           table [skip first n=2] {./doc/chapters/IBVP_BVP/figures/NLvibrationsd.plt};
     \end{axis}
 \end{tikzpicture}
}
{Time evolution of nonlinear vibrations $w(x,y,t) $ with  $11 \times 11$ nodal points and order $q=6$. (a)  $w(x,y,0.25)$. (b)  $w(x,y,0.5)$. (c) Numerical solution at $w(x,y,0.75)$. (d)Numerical solution at $w(x,y,1)$.}{fig:NLPlate_Vibrations_Pulse}














